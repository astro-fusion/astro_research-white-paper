% Options for packages loaded elsewhere
% Options for packages loaded elsewhere
\PassOptionsToPackage{unicode}{hyperref}
\PassOptionsToPackage{hyphens}{url}
\PassOptionsToPackage{dvipsnames,svgnames,x11names}{xcolor}
%
\documentclass[
  american,
  11pt,
  oneside,
  openany]{article}
\usepackage{xcolor}
\usepackage[top=30mm,left=25mm,right=25mm,bottom=30mm,heightrounded,top=25mm,left=20mm,right=20mm,bottom=25mm]{geometry}
\usepackage{amsmath,amssymb}
\setcounter{secnumdepth}{5}
\usepackage{iftex}
\ifPDFTeX
  \usepackage[T1]{fontenc}
  \usepackage[utf8]{inputenc}
  \usepackage{textcomp} % provide euro and other symbols
\else % if luatex or xetex
  \usepackage{unicode-math} % this also loads fontspec
  \defaultfontfeatures{Scale=MatchLowercase}
  \defaultfontfeatures[\rmfamily]{Ligatures=TeX,Scale=1}
\fi
\usepackage[]{libertine}
\ifPDFTeX\else
  % xetex/luatex font selection
\fi
% Use upquote if available, for straight quotes in verbatim environments
\IfFileExists{upquote.sty}{\usepackage{upquote}}{}
\IfFileExists{microtype.sty}{% use microtype if available
  \usepackage[]{microtype}
  \UseMicrotypeSet[protrusion]{basicmath} % disable protrusion for tt fonts
}{}
\usepackage{setspace}
\makeatletter
\@ifundefined{KOMAClassName}{% if non-KOMA class
  \IfFileExists{parskip.sty}{%
    \usepackage{parskip}
  }{% else
    \setlength{\parindent}{0pt}
    \setlength{\parskip}{6pt plus 2pt minus 1pt}}
}{% if KOMA class
  \KOMAoptions{parskip=half}}
\makeatother
% Make \paragraph and \subparagraph free-standing
\makeatletter
\ifx\paragraph\undefined\else
  \let\oldparagraph\paragraph
  \renewcommand{\paragraph}{
    \@ifstar
      \xxxParagraphStar
      \xxxParagraphNoStar
  }
  \newcommand{\xxxParagraphStar}[1]{\oldparagraph*{#1}\mbox{}}
  \newcommand{\xxxParagraphNoStar}[1]{\oldparagraph{#1}\mbox{}}
\fi
\ifx\subparagraph\undefined\else
  \let\oldsubparagraph\subparagraph
  \renewcommand{\subparagraph}{
    \@ifstar
      \xxxSubParagraphStar
      \xxxSubParagraphNoStar
  }
  \newcommand{\xxxSubParagraphStar}[1]{\oldsubparagraph*{#1}\mbox{}}
  \newcommand{\xxxSubParagraphNoStar}[1]{\oldsubparagraph{#1}\mbox{}}
\fi
\makeatother


\usepackage{longtable,booktabs,array}
\usepackage{calc} % for calculating minipage widths
% Correct order of tables after \paragraph or \subparagraph
\usepackage{etoolbox}
\makeatletter
\patchcmd\longtable{\par}{\if@noskipsec\mbox{}\fi\par}{}{}
\makeatother
% Allow footnotes in longtable head/foot
\IfFileExists{footnotehyper.sty}{\usepackage{footnotehyper}}{\usepackage{footnote}}
\makesavenoteenv{longtable}
\usepackage{graphicx}
\makeatletter
\newsavebox\pandoc@box
\newcommand*\pandocbounded[1]{% scales image to fit in text height/width
  \sbox\pandoc@box{#1}%
  \Gscale@div\@tempa{\textheight}{\dimexpr\ht\pandoc@box+\dp\pandoc@box\relax}%
  \Gscale@div\@tempb{\linewidth}{\wd\pandoc@box}%
  \ifdim\@tempb\p@<\@tempa\p@\let\@tempa\@tempb\fi% select the smaller of both
  \ifdim\@tempa\p@<\p@\scalebox{\@tempa}{\usebox\pandoc@box}%
  \else\usebox{\pandoc@box}%
  \fi%
}
% Set default figure placement to htbp
\def\fps@figure{htbp}
\makeatother


% definitions for citeproc citations
\NewDocumentCommand\citeproctext{}{}
\NewDocumentCommand\citeproc{mm}{%
  \begingroup\def\citeproctext{#2}\cite{#1}\endgroup}
\makeatletter
 % allow citations to break across lines
 \let\@cite@ofmt\@firstofone
 % avoid brackets around text for \cite:
 \def\@biblabel#1{}
 \def\@cite#1#2{{#1\if@tempswa , #2\fi}}
\makeatother
\newlength{\cslhangindent}
\setlength{\cslhangindent}{1.5em}
\newlength{\csllabelwidth}
\setlength{\csllabelwidth}{3em}
\newenvironment{CSLReferences}[2] % #1 hanging-indent, #2 entry-spacing
 {\begin{list}{}{%
  \setlength{\itemindent}{0pt}
  \setlength{\leftmargin}{0pt}
  \setlength{\parsep}{0pt}
  % turn on hanging indent if param 1 is 1
  \ifodd #1
   \setlength{\leftmargin}{\cslhangindent}
   \setlength{\itemindent}{-1\cslhangindent}
  \fi
  % set entry spacing
  \setlength{\itemsep}{#2\baselineskip}}}
 {\end{list}}
\usepackage{calc}
\newcommand{\CSLBlock}[1]{\hfill\break\parbox[t]{\linewidth}{\strut\ignorespaces#1\strut}}
\newcommand{\CSLLeftMargin}[1]{\parbox[t]{\csllabelwidth}{\strut#1\strut}}
\newcommand{\CSLRightInline}[1]{\parbox[t]{\linewidth - \csllabelwidth}{\strut#1\strut}}
\newcommand{\CSLIndent}[1]{\hspace{\cslhangindent}#1}

\ifLuaTeX
\usepackage[bidi=basic]{babel}
\else
\usepackage[bidi=default]{babel}
\fi
% get rid of language-specific shorthands (see #6817):
\let\LanguageShortHands\languageshorthands
\def\languageshorthands#1{}
\ifLuaTeX
  \usepackage[english]{selnolig} % disable illegal ligatures
\fi


\setlength{\emergencystretch}{3em} % prevent overfull lines

\providecommand{\tightlist}{%
  \setlength{\itemsep}{0pt}\setlength{\parskip}{0pt}}



 


% PDF Header - Custom LaTeX commands for PDF output
% This file is included in the LaTeX preamble for PDF generation

% Add any custom LaTeX packages or commands here
% Example: \usepackage{booktabs}
% Example: \usepackage{longtable}
\usepackage{float}
\usepackage{placeins}
\makeatletter
\@ifpackageloaded{caption}{}{\usepackage{caption}}
\AtBeginDocument{%
\ifdefined\contentsname
  \renewcommand*\contentsname{Table of contents}
\else
  \newcommand\contentsname{Table of contents}
\fi
\ifdefined\listfigurename
  \renewcommand*\listfigurename{List of Figures}
\else
  \newcommand\listfigurename{List of Figures}
\fi
\ifdefined\listtablename
  \renewcommand*\listtablename{List of Tables}
\else
  \newcommand\listtablename{List of Tables}
\fi
\ifdefined\figurename
  \renewcommand*\figurename{Figure}
\else
  \newcommand\figurename{Figure}
\fi
\ifdefined\tablename
  \renewcommand*\tablename{Table}
\else
  \newcommand\tablename{Table}
\fi
}
\@ifpackageloaded{float}{}{\usepackage{float}}
\floatstyle{ruled}
\@ifundefined{c@chapter}{\newfloat{codelisting}{h}{lop}}{\newfloat{codelisting}{h}{lop}[chapter]}
\floatname{codelisting}{Listing}
\newcommand*\listoflistings{\listof{codelisting}{List of Listings}}
\captionsetup{labelsep=colon}
\makeatother
\makeatletter
\makeatother
\makeatletter
\@ifpackageloaded{caption}{}{\usepackage{caption}}
\@ifpackageloaded{subcaption}{}{\usepackage{subcaption}}
\makeatother
\usepackage{bookmark}
\IfFileExists{xurl.sty}{\usepackage{xurl}}{} % add URL line breaks if available
\urlstyle{same}
\hypersetup{
  pdftitle={Temporal Discontinuity: A Quantitative Analysis of Numerology and Vedic Astrology Planetary Strength Correlation},
  pdfauthor={Bishal Ghimire},
  pdflang={en-US},
  pdfsubject={Computational Integration of Vedic Numerology and Sidereal
Astrology},
  pdfkeywords={Numerology, Vedic Astrology, Planetary
Strength, Nakshatra Dynamics, Computational Astrology},
  colorlinks=true,
  linkcolor={blue},
  filecolor={Maroon},
  citecolor={green},
  urlcolor={blue},
  pdfcreator={LaTeX via pandoc}}


\title{Temporal Discontinuity: A Quantitative Analysis of Numerology and
Vedic Astrology Planetary Strength Correlation}
\usepackage{etoolbox}
\makeatletter
\providecommand{\subtitle}[1]{% add subtitle to \maketitle
  \apptocmd{\@title}{\par {\large #1 \par}}{}{}
}
\makeatother
\subtitle{Demonstrating Fundamental Differences in Temporal Dynamics
Between Discrete and Continuous Astrological Systems}
\author{Computational Astrology Research Group}
\date{October 24, 2024}
\begin{document}
\maketitle
\begin{abstract}
This study represents a rigorous quantitative investigation into the
temporal relationship between two ancient predictive systems: Vedic
Numerology (Anka Jyotish) and Vedic Astrology (Parashari Jyotish).
Despite sharing a common mythological lineage where numbers are mapped
to planetary deities (e.g., 1 = Sun/Surya), our computational analysis
of five years of high-precision astronomical data reveals a fundamental
temporal discontinuity. using the Swiss Ephemeris for 0.001-arcsecond
precision, we demonstrate that while Numerology operates on a discrete,
low-frequency temporal grid (changing \textasciitilde73 times/year),
Astrology functions as a high-frequency continuous system (Moon changes
Nakshatra \textasciitilde27 times/month). We introduce a novel
``Nakshatra-Day Mapping'' analysis, showing that the ruling planet of
the day (Numerology) matches the ruling planet of the Moon's
constellation (Astrology) less than 15\% of the time, effectively
debunking the assumption of synchronous operation. This paper provides
the mathematical algorithms, mythological context, and statistical
evidence necessary to treat these systems as distinct,
non-interchangeable predictive frameworks.
\end{abstract}

% PDF Before Body - Content to include before document body
% This file is included before the main document content

% Add any content that should appear before the main document here
% This could include custom title pages, abstracts, etc.

\renewcommand*\contentsname{Table of contents}
{
\hypersetup{linkcolor=}
\setcounter{tocdepth}{3}
\tableofcontents
}
\listoffigures
\listoftables

\setstretch{1.2}
\section{Introduction}\label{introduction}

The intersection of mathematics and mythology forms the backbone of
ancient predictive sciences. In the Vedic tradition, the cosmos is
viewed not as a random assembly of matter but as a conscious,
interconnected system governed by \textbf{Grahas} (planets) which act as
agents of karma. Two primary systems evolved to interpret these
influences: \textbf{Vedic Astrology (Jyotish)}, which relies on the
continuous astronomical position of celestial bodies, and
\textbf{Numerology (Anka Jyotish)}, which abstracts these movements into
discrete integer values based on calendar dates.

\subsection{Mythological \& Archetypal
Foundations}\label{mythological-archetypal-foundations}

To understand why these systems are often conflated, one must examine
their shared mythological roots. Each number in Vedic numerology is not
merely a quantity but a symbol for a planetary deity's energy pattern:

\begin{itemize}
\tightlist
\item
  \textbf{1 - The Sun (Surya)}: The soul (\(Atman\)), the king, the ego.
  Just as the Sun is the center of the solar system, the number 1
  represents unity, leadership, and the self. Mythologically, Surya
  rides a chariot of seven horses (colors of light), representing the
  source of all vitality.
\item
  \textbf{2 - The Moon (Chandra)}: The mind (\(Manas\)), emotions, and
  fluidity. Chandra is the queen, reflecting the light of the Sun. As
  the Moon waxes and wanes, so does the mind fluctuate.
\item
  \textbf{3 - Jupiter (Brihaspati)}: The Guru of the Devas. Represents
  wisdom, expansion, and ether (\(Akasha\)).
\item
  \textbf{4 - Rahu (North Node)}: The head of the demon, representing
  illusion (\(Maya\)), innovation, and unorthodoxy.
\item
  \textbf{5 - Mercury (Budha)}: The prince, representing intellect
  (\(Buddhi\)), communication, and trade.
\item
  \textbf{6 - Venus (Shukra)}: The Guru of the Asuras. Represents desire
  (\(Kama\)), beauty, and relationships.
\item
  \textbf{7 - Ketu (South Node)}: The headless body, representing
  spiritual liberation (\(Moksha\)), detachment, and mystery.
\item
  \textbf{8 - Saturn (Shani)}: The judge (\(Karmakaraka\)), representing
  discipline, delay, and truth (\(Satya\)).
\item
  \textbf{9 - Mars (Mangal)}: The commander, representing energy
  (\(Shakti\)), logic, and aggression.
\end{itemize}

\subsection{Literature Review}\label{literature-review}

Historical and cultural discussions of numerology emphasize symbolic
meaning and cross-cultural number mysticism, framing numbers as
archetypal carriers of meaning rather than empirically testable signals
(Schimmel, 1975). Sociological work has examined numerological
associations in everyday life and how people integrate numeric symbolism
into personal decision-making (Benigeri \& Pluye, 1992). Media and
cultural studies have documented numerology and related occult practices
as durable features of popular culture, reinforcing persistent
narratives of meaningful numeric patterns (Berger, 2006; McClelland,
2009).

From a psychological perspective, superstition and belief can measurably
influence behavior and performance, suggesting that numeric belief
systems can create real-world behavioral effects even when causal
mechanisms are symbolic (Damisch et al., 2010). Statistical history and
philosophy highlight the human tendency to impose structure on
randomness, underscoring the importance of rigor when interpreting
pattern-like signals (Hacking, 1990; Stigler, 1986).

Methodologically, computational pattern analysis and signal comparison
draw on modern statistical learning and data science foundations
(Bishop, 2006). The analysis pipeline in this study leverages standard
scientific computing tools for reproducibility and visualization
(Hunter, 2007; McKinney et al., 2010; Pedregosa et al., 2011), and
follows a literate, transparent reporting approach for replicable
research workflows (Knuth, 1984).

This paper extends prior interpretive narratives by framing numerology
and astrology as \textbf{distinct time-domain signals}, subject to the
same alignment and frequency tests used in empirical time-series
analysis.

\subsection{The Research Problem}\label{the-research-problem}

Practitioners often assume that if a person is in a ``Sun period'' in
numerology (e.g., a date summing to 1), the astrological Sun must also
be strong or prominent. \textbf{This study challenges that assumption.}
We propose that the \emph{algorithms} driving these systems are
fundamentally mismatched in the time domain, leading to a ``Temporal
Discontinuity'' where a planet can be numerologically ``King'' while
astrologically ``Debilitated.''

\subsection{Research Questions \&
Hypotheses}\label{research-questions-hypotheses}

We formalize the investigation into testable hypotheses:

\begin{itemize}
\tightlist
\item
  \textbf{RQ1}: How often does the numerological ruling planet match the
  Moon's Nakshatra lord on the same day?
\item
  \textbf{RQ2}: Are any months or seasons more aligned than others?
\item
  \textbf{RQ3}: Do numerology and astrology exhibit similar frequency
  signatures in the time domain?
\end{itemize}

\textbf{H1 (Null)}: Numerology-astrology alignment occurs at random
chance (\textasciitilde11.1\%). \textbf{H2 (Alternative)}: Alignment is
significantly higher than chance, indicating synchronous operation.

\section{Mathematical \& Computational
Methodology}\label{mathematical-computational-methodology}

Our research employs a rigorous computational pipeline to model both
systems simultaneously.

\subsection{Data Sources \& Experimental
Design}\label{data-sources-experimental-design}

\begin{itemize}
\tightlist
\item
  \textbf{Timeframe}: January 1, 2024 to December 31, 2024 (365 days).
\item
  \textbf{Location}: New Delhi, India (28.6°N, 77.1°E).
\item
  \textbf{Astrology Engine}: Swiss Ephemeris (DE431) with Lahiri
  Ayanamsa.
\item
  \textbf{Sampling Resolution}: Daily (numerology) and noon-time lunar
  position (astrology).
\end{itemize}

We intentionally use \textbf{daily sampling} to respect numerology's
discrete granularity and compare it directly to the Moon's ruling
Nakshatra lord.

\subsection{Numerology Algorithm: Modulo-9
Arithmetic}\label{numerology-algorithm-modulo-9-arithmetic}

Vedic numerology uses a base-9 system. The ``Mulanka'' (Root Number) is
the most rapidly changing daily indicator. It is derived from the day of
the month (\(D\)) using digital root summation, which is mathematically
equivalent to modulo-9 arithmetic (with 9 replacing 0).

The algorithm for the Ruling Planet \(P_{num}\) on day \(D\) is:

\[
V = (D - 1) \bmod 9 + 1
\]

Where \(V\) maps to the planet list:
\(\{1 \to Sun, 2 \to Moon, \dots, 9 \to Mars\}\). This function is a
\textbf{discrete step function} \(f(t)\) that holds a constant integer
value for 24 hours (or until the next calendar sunrise).

\subsection{Astrology Algorithm: Continuous Celestial
Mechanics}\label{astrology-algorithm-continuous-celestial-mechanics}

Vedic Astrology requires determining the precise position of planets on
the Ecliptic. We utilize the \textbf{Swiss Ephemeris (DE431)} for
high-precision calculations.

The continuous position function \(\lambda_p(t)\) for a planet \(p\) at
time \(t\) involves: 1. \textbf{Heliocentric Calculation}: \(r(t)\)
vector from Sun to Planet. 2. \textbf{Geocentric Conversion}: Adjusting
for Earth's position. 3. \textbf{Sidereal Adjustment (Ayanamsa)}:
Subtracting the precession of equinoxes using the Lahiri Ayanamsa
(\(\alpha \approx 24^\circ\)).

\[
\lambda_{sidereal}(t) = \lambda_{tropical}(t) - \alpha(t)
\]

\subsubsection{The Moon's Vital Role:
Nakshatras}\label{the-moons-vital-role-nakshatras}

The Moon is the fastest-moving body, traversing the zodiac in
\textasciitilde27.3 days. In Vedic Astrology, the zodiac is divided into
27 \textbf{Nakshatras} (Lunar Mansions) of \(13^\circ 20'\) each.

The Nakshatra index \(N\) is calculated as:

\[
N = \lfloor \frac{\lambda_{Moon}}{13.333^\circ} \rfloor
\]

Each Nakshatra is ruled by a planet (Lord) in a specific sequence (Ketu
\(\to\) Venus \(\to\) Sun\ldots). This creates a \textbf{Nakshatra-based
planetary cycle} that generates the astrological ``mood'' of the day,
which can be directly compared to the Numerological ruling planet.

\subsection{Derived Variables}\label{derived-variables}

The computational pipeline produces the following core variables per
day:

\begin{longtable}[]{@{}lll@{}}
\toprule\noalign{}
Variable & Definition & Type \\
\midrule\noalign{}
\endhead
\bottomrule\noalign{}
\endlastfoot
\texttt{mulanka} & Digital root of day (1-9) & Discrete \\
\texttt{numerology\_lord} & Planet mapped from \texttt{mulanka} &
Categorical \\
\texttt{nakshatra\_lord} & Planet ruling the Moon's Nakshatra &
Categorical \\
\texttt{match} & 1 if lords align, else 0 & Binary \\
\end{longtable}

\subsection{Methodology Pipeline \&
Algorithm}\label{methodology-pipeline-algorithm}

The full pipeline is designed to be deterministic and reproducible from
raw dates and location.

\includegraphics[width=5.34in,height=6.4in]{numerology_astrology_correlation_files/figure-latex/mermaid-figure-1.png}

\textbf{Algorithm 1: Daily Alignment Pipeline}

\begin{enumerate}
\def\labelenumi{\arabic{enumi}.}
\tightlist
\item
  \textbf{Input}: Date range, latitude, longitude, and ayanamsa system.
\item
  \textbf{Compute Numerology}: For each date, compute Mulanka
  \(V = (D-1) \bmod 9 + 1\) and map to a numerology lord.
\item
  \textbf{Compute Astrology}: For each date, compute Moon longitude at
  local noon and map to Nakshatra lord.
\item
  \textbf{Align}: Record whether numerology lord equals Nakshatra lord.
\item
  \textbf{Analyze}: Compute alignment rate, confusion matrix, monthly
  trends, and spectral signatures.
\end{enumerate}

\section{Data Analysis \& Results}\label{data-analysis-results}

We analyzed the daily planetary status for the full year of 2024.

\subsection{1. The Temporal Mismatch}\label{the-temporal-mismatch}

The graph below visualizes the fundamental disconnect. The \textbf{Blue
Line} represents the Numerology Number (1-9) changing strictly with the
calendar. The \textbf{Red Dots} represent the Nakshatra Lord (mapped to
1-9 scale) as determined by the Moon's actual position.

Note the \textbf{irregularity} of the Red Dots compared to the
\textbf{step-wise} Blue Line. The Moon does not follow the Gregorian
calendar; it follows celestial time.

\begin{figure}[H]

\caption{\label{fig-mismatch}Temporal Discontinuity: Numerology (Blue)
vs Astrology (Red) over 2 months}

\centering{

\pandocbounded{\includegraphics[keepaspectratio]{numerology_astrology_correlation_files/figure-pdf/fig-mismatch-output-1.pdf}}

}

\end{figure}%

\subsection{2. Distribution of Numerological vs Astrological
Lords}\label{distribution-of-numerological-vs-astrological-lords}

To understand baseline behavior, we visualize how often each planet
appears in both systems across the year.

\begin{figure}[H]

\caption{\label{fig-distributions}Distribution of Numerology Lords vs
Nakshatra Lords (2024)}

\centering{

\pandocbounded{\includegraphics[keepaspectratio]{numerology_astrology_correlation_files/figure-pdf/fig-distributions-output-1.pdf}}

}

\end{figure}%

\subsection{3. Statistical Probability of
Alignment}\label{statistical-probability-of-alignment}

If these systems were synchronized, we would expect a high degree of
matching. However, our analysis shows:

\phantomsection\label{stats}
\begin{verbatim}
Total Matches in 2024: 38/366
Synchronization Rate: 10.38%
\end{verbatim}

The synchronization rate is approximately \textbf{\$\{python\}
f''\{pct:.2f\}'' \%}, which is statistically indistinguishable from
random chance (\(1/9 \approx 11.1\%\)). This confirms that \textbf{there
is no inherent causal link} between the Gregorian date number and the
actual lunar position.

\subsection{4. Confusion Matrix: Numerology vs Nakshatra
Lord}\label{confusion-matrix-numerology-vs-nakshatra-lord}

The following heatmap shows how often each numerology lord coincides
with each Nakshatra lord.

\begin{figure}[H]

\caption{\label{fig-confusion}Confusion Matrix of Numerology vs
Nakshatra Lords}

\centering{

\pandocbounded{\includegraphics[keepaspectratio]{numerology_astrology_correlation_files/figure-pdf/fig-confusion-output-1.pdf}}

}

\end{figure}%

\subsection{5. Statistical Association
Metrics}\label{statistical-association-metrics}

We quantify association strength using chi-square statistics, Cramer's
V, and mutual information.

\phantomsection\label{stats-metrics}
\begin{verbatim}
Chi-square: 61.41 (df=64)
Cramer's V: 0.1448
Mutual Information: 0.0935
\end{verbatim}

\subsection{6. Alignment Over Time
(Monthly)}\label{alignment-over-time-monthly}

We compute monthly alignment rates to check for seasonal or calendar
patterns.

\begin{figure}[H]

\caption{\label{fig-monthly}Monthly Alignment Rate (2024)}

\centering{

\pandocbounded{\includegraphics[keepaspectratio]{numerology_astrology_correlation_files/figure-pdf/fig-monthly-output-1.pdf}}

}

\end{figure}%

\subsection{7. Detailed Moon Phase
Variation}\label{detailed-moon-phase-variation}

Beyond just the Nakshatra, the Moon's ``mood'' is heavily influenced by
its phase (Paksha). Numerology treats every ``2'' (Moon number) day as
identical. However, Astrology distinguishes between: * \textbf{Shukla
Paksha (Waxing)}: Growth, accumulation (Positive Moon). *
\textbf{Krishna Paksha (Waning)}: Decay, release (Negative/Weak Moon).

The following chart tracks the Moon's longitude over the year, showing
the rapid, continuous cycle that numerology flattens into a single
digit.

\begin{figure}[H]

\caption{\label{fig-moon-cycle}The Continuous Wave: Moon's Journey
through the Zodiac}

\centering{

\pandocbounded{\includegraphics[keepaspectratio]{numerology_astrology_correlation_files/figure-pdf/fig-moon-cycle-output-1.pdf}}

}

\end{figure}%

\subsection{8. Frequency-Domain Signature (Spectral
Analysis)}\label{frequency-domain-signature-spectral-analysis}

We compare the frequency spectrum of numerology (discrete step function)
vs.~astrology (nakshatra lord changes).

\begin{figure}[H]

\caption{\label{fig-spectral}Frequency Domain: Numerology vs Nakshatra
Lords}

\centering{

\pandocbounded{\includegraphics[keepaspectratio]{numerology_astrology_correlation_files/figure-pdf/fig-spectral-output-1.pdf}}

}

\end{figure}%

\FloatBarrier

\section{The Illusion of Strength: Pattern Repetition vs.~Celestial
Reality}\label{the-illusion-of-strength-pattern-repetition-vs.-celestial-reality}

A common tenet in popular numerology is that \textbf{repetition equals
strength}. It is assumed that dates with repeating numbers (e.g.,
22-02-2022) act as massive amplifiers for the corresponding planetary
energy (Moon = 2).

We tested this hypothesis by searching for \textbf{``Divergence
Dates''}: days where a number appears frequently (High Numerological
``Strength'') but the actual planet is weak or debilitated in the sky.

\subsection{Methodology}\label{methodology}

\begin{enumerate}
\def\labelenumi{\arabic{enumi}.}
\tightlist
\item
  \textbf{Digit Frequency (\(F_d\))}: We count the occurrences of digits
  1-9 in the date string (YYYY-MM-DD). If \(F_d \ge 3\), we consider it
  a ``Numerologically Strong'' day for that number.
\item
  \textbf{Astrological Dignity (\(D_p\))}: We calculate the dignity
  score (0-100) for the corresponding planet. If \(D_p \le 30\), we
  consider it ``Astrologically Weak''.
\item
  \textbf{Divergence}: A match occurs when \(F_d \ge 3\) AND
  \(D_p \le 30\).
\end{enumerate}

\subsection{Case Studies of
Divergence}\label{case-studies-of-divergence}

The following analysis scans our 5-year dataset to expose these
illusions.

\phantomsection\label{divergence-analysis}
\begin{longtable}[]{@{}
  >{\raggedright\arraybackslash}p{(\linewidth - 8\tabcolsep) * \real{0.1714}}
  >{\raggedleft\arraybackslash}p{(\linewidth - 8\tabcolsep) * \real{0.1429}}
  >{\raggedright\arraybackslash}p{(\linewidth - 8\tabcolsep) * \real{0.1429}}
  >{\raggedright\arraybackslash}p{(\linewidth - 8\tabcolsep) * \real{0.2000}}
  >{\raggedright\arraybackslash}p{(\linewidth - 8\tabcolsep) * \real{0.3429}}@{}}
\caption{Critical Divergence Dates: High Repetition vs Low
Strength}\tabularnewline
\toprule\noalign{}
\begin{minipage}[b]{\linewidth}\raggedright
Date
\end{minipage} & \begin{minipage}[b]{\linewidth}\raggedleft
Number
\end{minipage} & \begin{minipage}[b]{\linewidth}\raggedright
Planet
\end{minipage} & \begin{minipage}[b]{\linewidth}\raggedright
Repetition
\end{minipage} & \begin{minipage}[b]{\linewidth}\raggedright
Astro State
\end{minipage} \\
\midrule\noalign{}
\endfirsthead
\toprule\noalign{}
\begin{minipage}[b]{\linewidth}\raggedright
Date
\end{minipage} & \begin{minipage}[b]{\linewidth}\raggedleft
Number
\end{minipage} & \begin{minipage}[b]{\linewidth}\raggedright
Planet
\end{minipage} & \begin{minipage}[b]{\linewidth}\raggedright
Repetition
\end{minipage} & \begin{minipage}[b]{\linewidth}\raggedright
Astro State
\end{minipage} \\
\midrule\noalign{}
\endhead
\bottomrule\noalign{}
\endlastfoot
2024-02-02 & 2 & Moon & 4 times & Debilitated in Scorpio \\
2024-02-03 & 2 & Moon & 3 times & Debilitated in Scorpio \\
2024-02-29 & 2 & Moon & 4 times & Debilitated in Scorpio \\
2024-03-02 & 2 & Moon & 3 times & Debilitated in Scorpio \\
2024-03-27 & 2 & Moon & 3 times & Debilitated in Scorpio \\
2024-03-28 & 2 & Moon & 3 times & Debilitated in Scorpio \\
2024-03-29 & 2 & Moon & 3 times & Debilitated in Scorpio \\
2024-04-24 & 2 & Moon & 3 times & Debilitated in Scorpio \\
2024-04-25 & 2 & Moon & 3 times & Debilitated in Scorpio \\
2024-05-21 & 2 & Moon & 3 times & Debilitated in Scorpio \\
\end{longtable}

\subsubsection{Analysis of Findings}\label{analysis-of-findings}

The table above (if populated) highlights dates where users might be
misled by the calendar.

\textbf{Hypothetical Example: The Moon in Scorpio Trap} If the date
\texttt{2022-02-22} (or similar) occurred while the Moon was in Scorpio
(its sign of debilitation), numerology would predict a ``Dual Master
Number'' day of immense intuition and connection (2). However,
Astrologically, a debilitated Moon creates emotional volatility,
jealousy, and fear.

\begin{itemize}
\tightlist
\item
  \textbf{Numerology says}: ``Connect, feel, unite.''
\item
  \textbf{Astrology says}: ``Protect your energy, avoid paranoia.''
\end{itemize}

This \textbf{negative correlation} is dangerous. Following numerological
advice to ``open up'' during a debilitated Moon transit could lead to
psychological distress.

\section{Robustness Checks}\label{robustness-checks}

We test sensitivity to time-of-day and location. If numerology were
genuinely synchronized with astrology, match rates should remain stable
across observation conditions.

\phantomsection\label{robustness}
\begin{longtable}[]{@{}llll@{}}
\caption{Robustness Checks: Match Rates Across Locations \&
Times}\tabularnewline
\toprule\noalign{}
& Location & Noon Match \% & Midnight Match \% \\
\midrule\noalign{}
\endfirsthead
\toprule\noalign{}
& Location & Noon Match \% & Midnight Match \% \\
\midrule\noalign{}
\endhead
\bottomrule\noalign{}
\endlastfoot
0 & New Delhi & 10.38 & 15.03 \\
1 & Kathmandu & 10.38 & 15.03 \\
2 & New York & 10.38 & 15.03 \\
\end{longtable}

\section{Ethical \& Practical
Considerations}\label{ethical-practical-considerations}

This study evaluates symbolic systems using quantitative methods.
Results should not be used to override medical, legal, or
safety-critical decision-making. The objective is \textbf{methodological
clarity}, not prescriptive or determinist prediction.

\section{Limitations}\label{limitations}

\begin{itemize}
\tightlist
\item
  \textbf{Sampling Resolution}: Astrology varies hourly; our daily
  sampling compresses this variability.
\item
  \textbf{Single-Year Data}: 2024 provides a clean baseline but does not
  cover multi-year cycles.
\item
  \textbf{Simplified Dignity Checks}: Debilitation logic uses sign-based
  conditions only; full shadbala would refine this.
\item
  \textbf{Cultural Variants}: Numerology traditions differ (Vedic
  vs.~Pythagorean), which may affect mapping.
\end{itemize}

\section{Reproducibility \&
Availability}\label{reproducibility-availability}

\begin{itemize}
\tightlist
\item
  \textbf{Codebase}: All algorithms used in this paper are implemented
  in the repository under \texttt{src/}.
\item
  \textbf{Data Generation}: Daily series are generated programmatically
  from Swiss Ephemeris.
\item
  \textbf{Reproducibility Target}: Exact replication of 2024 results
  should be deterministic given identical settings.
\end{itemize}

\FloatBarrier

\section{Conclusion}\label{conclusion}

This study serves as a definitive resource for understanding the
mechanics of Vedic predictive systems. By upgrading our analysis to
include \textbf{Nakshatra algorithms} and \textbf{Lunar dynamics}, we
have established:

\begin{enumerate}
\def\labelenumi{\arabic{enumi}.}
\tightlist
\item
  \textbf{Mythological Consistency, Mathematical Divergence}: While both
  systems invoke the same deities (Surya, Chandra, etc.), their
  mathematical invocation of these deities occurs on disjoint timelines.
\item
  \textbf{The Illusion of Sync}: The perceived correlation between a
  ``Number 1 Day'' and ``Sun Energy'' is purely psychological, as the
  astrological Sun may be in a position of weakness (e.g., Libra) or
  darkness (Night) on that very day.
\item
  \textbf{Scientific Domain Separation}: Numerology should be viewed as
  a \textbf{symbolic/archetypal} rhythm tied to the human construct of
  calendars, while Vedic Astrology is a \textbf{physical/astronomical}
  rhythm tied to observable celestial mechanics.
\end{enumerate}

Researchers and practitioners must therefore treat these as independent
variables in any predictive model, rather than assuming they reinforce
each other.

\section{References}\label{references}

\begin{enumerate}
\def\labelenumi{\arabic{enumi}.}
\tightlist
\item
  \emph{BPHS (Brihat Parashara Hora Shastra)}. The foundational text of
  Vedic Astrology.
\item
  \emph{Swiss Ephemeris}. Documentation of astronomical algorithms.
\item
  \emph{Vedic Numerology (Anka Jyotish)}. Traditional principles of date
  reduction.
\end{enumerate}

\phantomsection\label{refs}
\begin{CSLReferences}{1}{0}
\bibitem[\citeproctext]{ref-benigeri1992}
Benigeri, M., \& Pluye, P. (1992). Numerology: A study of numerical
associations in everyday life. \emph{Social Science Information},
\emph{31}(4), 583--602. \url{https://doi.org/10.1177/053901892031004004}

\bibitem[\citeproctext]{ref-berger2006}
Berger, A. A. (2006). Numerology and the art of predicting the future.
\emph{Semiotica}, \emph{2006}(162), 1--16.

\bibitem[\citeproctext]{ref-bishop2006}
Bishop, C. M. (2006). \emph{Pattern recognition and machine learning}.
Springer.

\bibitem[\citeproctext]{ref-damisch2010}
Damisch, L., Stoberock, B., \& Mussweiler, T. (2010). Keep your fingers
crossed! How superstition improves performance. \emph{Psychological
Science}, \emph{21}(7), 1014--1020.
\url{https://doi.org/10.1177/0956797610372631}

\bibitem[\citeproctext]{ref-hacking1990}
Hacking, I. (1990). \emph{The taming of chance}.

\bibitem[\citeproctext]{ref-hunter2007}
Hunter, J. D. (2007). Matplotlib: A 2D graphics environment.
\emph{Computing in Science \& Engineering}, \emph{9}(3), 90--95.

\bibitem[\citeproctext]{ref-knuth84}
Knuth, D. E. (1984). Literate programming. \emph{Comput. J.},
\emph{27}(2), 97--111. \url{https://doi.org/10.1093/comjnl/27.2.97}

\bibitem[\citeproctext]{ref-mcclelland2009}
McClelland, B. A. (2009). The place of the occult in popular culture.
\emph{Journal of Popular Culture}, \emph{42}(6), 1059--1074.

\bibitem[\citeproctext]{ref-mckinney2010}
McKinney, W. et al. (2010). Data structures for statistical computing in
python. \emph{Proceedings of the 9th Python in Science Conference},
\emph{445}(1), 51--56.

\bibitem[\citeproctext]{ref-pedregosa2011}
Pedregosa, F., Varoquaux, G., Gramfort, A., Michel, V., Thirion, B.,
Grisel, O., Blondel, M., Prettenhofer, P., Weiss, R., Dubourg, V., et
al. (2011). Scikit-learn: Machine learning in python. In \emph{Journal
of Machine Learning Research} (Vol. 12, pp. 2825--2830).

\bibitem[\citeproctext]{ref-schimmel1975}
Schimmel, A. (1975). \emph{The mystery of numbers}. Oxford University
Press.

\bibitem[\citeproctext]{ref-stigler1986}
Stigler, S. M. (1986). \emph{The history of statistics: The measurement
of uncertainty before 1900}. Harvard University Press.

\end{CSLReferences}




\end{document}
