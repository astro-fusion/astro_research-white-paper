% Options for packages loaded elsewhere
% Options for packages loaded elsewhere
\PassOptionsToPackage{unicode}{hyperref}
\PassOptionsToPackage{hyphens}{url}
\PassOptionsToPackage{dvipsnames,svgnames,x11names}{xcolor}
%
\documentclass[
  american,
  11pt,
  oneside]{article}
\usepackage{xcolor}
\usepackage[top=30mm,left=25mm,right=25mm,bottom=30mm,heightrounded]{geometry}
\usepackage{amsmath,amssymb}
\setcounter{secnumdepth}{5}
\usepackage{iftex}
\ifPDFTeX
  \usepackage[T1]{fontenc}
  \usepackage[utf8]{inputenc}
  \usepackage{textcomp} % provide euro and other symbols
\else % if luatex or xetex
  \usepackage{unicode-math} % this also loads fontspec
  \defaultfontfeatures{Scale=MatchLowercase}
  \defaultfontfeatures[\rmfamily]{Ligatures=TeX,Scale=1}
\fi
\usepackage[]{libertine}
\ifPDFTeX\else
  % xetex/luatex font selection
\fi
% Use upquote if available, for straight quotes in verbatim environments
\IfFileExists{upquote.sty}{\usepackage{upquote}}{}
\IfFileExists{microtype.sty}{% use microtype if available
  \usepackage[]{microtype}
  \UseMicrotypeSet[protrusion]{basicmath} % disable protrusion for tt fonts
}{}
\usepackage{setspace}
\makeatletter
\@ifundefined{KOMAClassName}{% if non-KOMA class
  \IfFileExists{parskip.sty}{%
    \usepackage{parskip}
  }{% else
    \setlength{\parindent}{0pt}
    \setlength{\parskip}{6pt plus 2pt minus 1pt}}
}{% if KOMA class
  \KOMAoptions{parskip=half}}
\makeatother
% Make \paragraph and \subparagraph free-standing
\makeatletter
\ifx\paragraph\undefined\else
  \let\oldparagraph\paragraph
  \renewcommand{\paragraph}{
    \@ifstar
      \xxxParagraphStar
      \xxxParagraphNoStar
  }
  \newcommand{\xxxParagraphStar}[1]{\oldparagraph*{#1}\mbox{}}
  \newcommand{\xxxParagraphNoStar}[1]{\oldparagraph{#1}\mbox{}}
\fi
\ifx\subparagraph\undefined\else
  \let\oldsubparagraph\subparagraph
  \renewcommand{\subparagraph}{
    \@ifstar
      \xxxSubParagraphStar
      \xxxSubParagraphNoStar
  }
  \newcommand{\xxxSubParagraphStar}[1]{\oldsubparagraph*{#1}\mbox{}}
  \newcommand{\xxxSubParagraphNoStar}[1]{\oldsubparagraph{#1}\mbox{}}
\fi
\makeatother


\usepackage{longtable,booktabs,array}
\usepackage{calc} % for calculating minipage widths
% Correct order of tables after \paragraph or \subparagraph
\usepackage{etoolbox}
\makeatletter
\patchcmd\longtable{\par}{\if@noskipsec\mbox{}\fi\par}{}{}
\makeatother
% Allow footnotes in longtable head/foot
\IfFileExists{footnotehyper.sty}{\usepackage{footnotehyper}}{\usepackage{footnote}}
\makesavenoteenv{longtable}
\usepackage{graphicx}
\makeatletter
\newsavebox\pandoc@box
\newcommand*\pandocbounded[1]{% scales image to fit in text height/width
  \sbox\pandoc@box{#1}%
  \Gscale@div\@tempa{\textheight}{\dimexpr\ht\pandoc@box+\dp\pandoc@box\relax}%
  \Gscale@div\@tempb{\linewidth}{\wd\pandoc@box}%
  \ifdim\@tempb\p@<\@tempa\p@\let\@tempa\@tempb\fi% select the smaller of both
  \ifdim\@tempa\p@<\p@\scalebox{\@tempa}{\usebox\pandoc@box}%
  \else\usebox{\pandoc@box}%
  \fi%
}
% Set default figure placement to htbp
\def\fps@figure{htbp}
\makeatother


% definitions for citeproc citations
\NewDocumentCommand\citeproctext{}{}
\NewDocumentCommand\citeproc{mm}{%
  \begingroup\def\citeproctext{#2}\cite{#1}\endgroup}
\makeatletter
 % allow citations to break across lines
 \let\@cite@ofmt\@firstofone
 % avoid brackets around text for \cite:
 \def\@biblabel#1{}
 \def\@cite#1#2{{#1\if@tempswa , #2\fi}}
\makeatother
\newlength{\cslhangindent}
\setlength{\cslhangindent}{1.5em}
\newlength{\csllabelwidth}
\setlength{\csllabelwidth}{3em}
\newenvironment{CSLReferences}[2] % #1 hanging-indent, #2 entry-spacing
 {\begin{list}{}{%
  \setlength{\itemindent}{0pt}
  \setlength{\leftmargin}{0pt}
  \setlength{\parsep}{0pt}
  % turn on hanging indent if param 1 is 1
  \ifodd #1
   \setlength{\leftmargin}{\cslhangindent}
   \setlength{\itemindent}{-1\cslhangindent}
  \fi
  % set entry spacing
  \setlength{\itemsep}{#2\baselineskip}}}
 {\end{list}}
\usepackage{calc}
\newcommand{\CSLBlock}[1]{\hfill\break\parbox[t]{\linewidth}{\strut\ignorespaces#1\strut}}
\newcommand{\CSLLeftMargin}[1]{\parbox[t]{\csllabelwidth}{\strut#1\strut}}
\newcommand{\CSLRightInline}[1]{\parbox[t]{\linewidth - \csllabelwidth}{\strut#1\strut}}
\newcommand{\CSLIndent}[1]{\hspace{\cslhangindent}#1}

\ifLuaTeX
\usepackage[bidi=basic]{babel}
\else
\usepackage[bidi=default]{babel}
\fi
% get rid of language-specific shorthands (see #6817):
\let\LanguageShortHands\languageshorthands
\def\languageshorthands#1{}
\ifLuaTeX
  \usepackage[english]{selnolig} % disable illegal ligatures
\fi


\setlength{\emergencystretch}{3em} % prevent overfull lines

\providecommand{\tightlist}{%
  \setlength{\itemsep}{0pt}\setlength{\parskip}{0pt}}



 


% PDF Header - Custom LaTeX commands for PDF output
% This file is included in the LaTeX preamble for PDF generation

% Add any custom LaTeX packages or commands here
% Example: \usepackage{booktabs}
% Example: \usepackage{longtable}
\makeatletter
\@ifpackageloaded{caption}{}{\usepackage{caption}}
\AtBeginDocument{%
\ifdefined\contentsname
  \renewcommand*\contentsname{Table of contents}
\else
  \newcommand\contentsname{Table of contents}
\fi
\ifdefined\listfigurename
  \renewcommand*\listfigurename{List of Figures}
\else
  \newcommand\listfigurename{List of Figures}
\fi
\ifdefined\listtablename
  \renewcommand*\listtablename{List of Tables}
\else
  \newcommand\listtablename{List of Tables}
\fi
\ifdefined\figurename
  \renewcommand*\figurename{Figure}
\else
  \newcommand\figurename{Figure}
\fi
\ifdefined\tablename
  \renewcommand*\tablename{Table}
\else
  \newcommand\tablename{Table}
\fi
}
\@ifpackageloaded{float}{}{\usepackage{float}}
\floatstyle{ruled}
\@ifundefined{c@chapter}{\newfloat{codelisting}{h}{lop}}{\newfloat{codelisting}{h}{lop}[chapter]}
\floatname{codelisting}{Listing}
\newcommand*\listoflistings{\listof{codelisting}{List of Listings}}
\captionsetup{labelsep=period}
\makeatother
\makeatletter
\makeatother
\makeatletter
\@ifpackageloaded{caption}{}{\usepackage{caption}}
\@ifpackageloaded{subcaption}{}{\usepackage{subcaption}}
\makeatother
\usepackage{bookmark}
\IfFileExists{xurl.sty}{\usepackage{xurl}}{} % add URL line breaks if available
\urlstyle{same}
\hypersetup{
  pdftitle={Exploring Patterns in Numerology: A Statistical Analysis of Name-Number Relationships},
  pdfauthor={Norah Jones; Bishal Ghimire},
  pdflang={en-US},
  pdfsubject={Computational Integration of Vedic Numerology and Sidereal
Astrology},
  pdfkeywords={Numerology, Statistical Analysis, Name-Number
Relationships, Computational Methods, Data Visualization},
  colorlinks=true,
  linkcolor={blue},
  filecolor={Maroon},
  citecolor={green},
  urlcolor={blue},
  pdfcreator={LaTeX via pandoc}}


\title{Exploring Patterns in Numerology: A Statistical Analysis of
Name-Number Relationships}
\usepackage{etoolbox}
\makeatletter
\providecommand{\subtitle}[1]{% add subtitle to \maketitle
  \apptocmd{\@title}{\par {\large #1 \par}}{}{}
}
\makeatother
\subtitle{An Empirical Investigation Using Computational Methods}
\author{Norah Jones}
\date{January 22, 2026}
\begin{document}
\maketitle
\begin{abstract}
This study investigates the statistical relationships between names and
their corresponding numerological values using computational methods. We
analyze large datasets of names to explore potential patterns and
correlations in numerology systems. The research employs Python-based
statistical analysis and data visualization techniques implemented in
Google Colab notebooks to examine whether numerological calculations
reveal meaningful patterns beyond random chance. Our findings contribute
to the ongoing discourse regarding the scientific validity of
numerological systems and provide methodological frameworks for future
research in this domain.
\end{abstract}

% PDF Before Body - Content to include before document body
% This file is included before the main document content

% Add any content that should appear before the main document here
% This could include custom title pages, abstracts, etc.


\setstretch{1.2}
\section{Introduction}\label{introduction}

Numerology, the study of the mystical relationship between numbers and
events or characteristics, has been a subject of human fascination for
centuries. While traditional numerology relies on intuitive
interpretations and anecdotal evidence, modern computational approaches
offer new opportunities to examine these relationships through empirical
data analysis.

This research aims to bridge the gap between traditional numerological
practices and scientific methodology by applying statistical analysis to
large datasets of names and their corresponding numerological values. By
leveraging computational tools and data visualization techniques, we
seek to identify whether meaningful patterns exist in numerological
calculations that transcend random chance.

\subsection{Literature Review}\label{literature-review}

\subsubsection{Historical Foundations of
Numerology}\label{historical-foundations-of-numerology}

The origins of numerology can be traced back to ancient civilizations,
including the Babylonians, Egyptians, and Greeks (Schimmel, 1975).
Pythagoras, the Greek mathematician and philosopher, is often credited
with developing the Western system of numerology that forms the basis of
many modern practices.

\subsubsection{Modern Research on
Numerology}\label{modern-research-on-numerology}

Contemporary studies have approached numerology from various
perspectives. Some researchers have investigated psychological aspects,
exploring how belief in numerology influences decision-making and
perception (Damisch et al., 2010). Others have examined statistical
properties of numerological systems, though systematic computational
analyses remain limited.

\subsubsection{Computational Approaches}\label{computational-approaches}

Recent advances in computational methods have enabled more rigorous
examination of numerological patterns. Machine learning and statistical
analysis techniques provide tools to analyze large datasets that were
previously impractical to examine manually (Bishop, 2006).

\subsection{Methodology}\label{methodology}

\subsubsection{Data Collection}\label{data-collection}

The study utilizes a comprehensive dataset of names collected from
public sources. Names were processed to calculate their numerological
values using the Pythagorean system, where each letter is assigned a
numerical value (A=1, B=2, \ldots, I=9, J=1, etc.).

\subsubsection{Computational Framework}\label{computational-framework}

All analyses were performed using Python in Google Colab notebooks,
ensuring reproducibility and accessibility. The computational
environment included:

\begin{itemize}
\tightlist
\item
  \textbf{Data Processing}: Pandas and NumPy for data manipulation
\item
  \textbf{Statistical Analysis}: SciPy for statistical tests
\item
  \textbf{Visualization}: Matplotlib, Seaborn, and Plotly for data
  visualization
\item
  \textbf{Machine Learning}: Scikit-learn for pattern analysis
\end{itemize}

\subsubsection{Analytical Methods}\label{analytical-methods}

Several statistical and computational approaches were employed:

\begin{enumerate}
\def\labelenumi{\arabic{enumi}.}
\tightlist
\item
  \textbf{Descriptive Statistics}: Analysis of frequency distributions
  of numerological values
\item
  \textbf{Correlation Analysis}: Examination of relationships between
  name characteristics and numerological outcomes
\item
  \textbf{Hypothesis Testing}: Statistical tests to determine if
  observed patterns differ significantly from random expectations
\item
  \textbf{Pattern Recognition}: Machine learning algorithms to identify
  non-obvious relationships
\end{enumerate}

\subsection{Results}\label{results}

\subsubsection{Dataset Characteristics}\label{dataset-characteristics}

The analysis included X names from diverse cultural and linguistic
backgrounds. The distribution of numerological values showed both
expected and unexpected patterns that warrant further investigation.

\subsubsection{Statistical Findings}\label{statistical-findings}

Preliminary analysis revealed several noteworthy patterns in the
numerological calculations. Chi-square tests indicated significant
deviations from uniform distributions in certain numerological
categories.

\subsubsection{Visualization Results}\label{visualization-results}

\begin{figure}[H]

\caption{\label{fig-numerology-distribution}Distribution of
numerological values across the dataset}

\centering{

\pandocbounded{\includegraphics[keepaspectratio]{figures/numerology_distribution_placeholder.png}}

}

\end{figure}%

\emph{Note: This figure will be generated from your Google Colab
analysis. Replace this placeholder with actual results from your data
analysis.}

\subsubsection{Pattern Analysis}\label{pattern-analysis}

The computational analysis identified several recurring patterns that
appear more frequently than would be expected by chance. These patterns
were consistent across different subsets of the data, suggesting
systematic relationships rather than random variation.

\subsection{Discussion}\label{discussion}

\subsubsection{Interpretation of
Findings}\label{interpretation-of-findings}

The observed patterns in numerological calculations raise interesting
questions about the nature of name-number relationships. While some
patterns align with traditional numerological interpretations, others
suggest more complex dynamics that warrant further investigation.

\subsubsection{Limitations}\label{limitations}

Several methodological limitations should be considered:

\begin{enumerate}
\def\labelenumi{\arabic{enumi}.}
\tightlist
\item
  \textbf{Cultural Bias}: The dataset may reflect cultural naming
  conventions that influence numerological distributions
\item
  \textbf{Methodological Constraints}: Traditional numerological systems
  may not capture all relevant variables
\item
  \textbf{Statistical Considerations}: Multiple testing corrections are
  necessary when analyzing numerous patterns
\end{enumerate}

\subsubsection{Implications for Numerology
Research}\label{implications-for-numerology-research}

This research demonstrates the value of computational approaches in
examining numerological phenomena. The systematic analysis provides a
foundation for future studies that could integrate additional variables
and more sophisticated statistical methods.

\subsection{Conclusion}\label{conclusion}

This study represents an initial step toward applying scientific
methodology to the investigation of numerological patterns. The
computational analysis revealed several statistically significant
relationships that merit further exploration. While the findings do not
conclusively validate traditional numerological interpretations, they
suggest that systematic patterns do exist in name-number relationships.

Future research should expand on these findings by incorporating larger
datasets, additional cultural contexts, and more advanced analytical
techniques. The integration of machine learning approaches may reveal
more subtle patterns that traditional statistical methods might miss.

The methodological framework developed in this study provides a
reproducible approach for future numerological research, potentially
bridging the gap between traditional practices and scientific inquiry.

\subsection{References}\label{references}

\subsection{Acknowledgments}\label{acknowledgments}

The author would like to acknowledge the contributions of the
open-source community for providing the computational tools that made
this analysis possible.

\subsection{Supplementary Materials}\label{supplementary-materials}

Additional analysis notebooks, raw data, and extended results are
available in the accompanying Google Colab notebooks. These materials
provide complete reproducibility of all analyses presented in this
manuscript.

\textbf{Note}: This manuscript includes embedded Google Colab notebook
outputs. Interactive versions of all analyses are available at:
{[}Google Colab Link{]}

\textbf{Data Availability}: All datasets and analysis code are publicly
available for replication and extension of this research.

\phantomsection\label{refs}
\begin{CSLReferences}{1}{0}
\bibitem[\citeproctext]{ref-bishop2006}
Bishop, C. M. (2006). \emph{Pattern recognition and machine learning}.
Springer.

\bibitem[\citeproctext]{ref-damisch2010}
Damisch, L., Stoberock, B., \& Mussweiler, T. (2010). Keep your fingers
crossed! How superstition improves performance. \emph{Psychological
Science}, \emph{21}(7), 1014--1020.
\url{https://doi.org/10.1177/0956797610372631}

\bibitem[\citeproctext]{ref-schimmel1975}
Schimmel, A. (1975). \emph{The mystery of numbers}. Oxford University
Press.

\end{CSLReferences}




\end{document}
