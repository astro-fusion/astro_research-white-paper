% Options for packages loaded elsewhere
\PassOptionsToPackage{unicode}{hyperref}
\PassOptionsToPackage{hyphens}{url}
\PassOptionsToPackage{dvipsnames,svgnames,x11names}{xcolor}
%
\documentclass[
  11pt,
  oneside,
  openany]{article}

\usepackage{amsmath,amssymb}
\usepackage{setspace}
\usepackage{iftex}
\ifPDFTeX
  \usepackage[T1]{fontenc}
  \usepackage[utf8]{inputenc}
  \usepackage{textcomp} % provide euro and other symbols
\else % if luatex or xetex
  \usepackage{unicode-math}
  \defaultfontfeatures{Scale=MatchLowercase}
  \defaultfontfeatures[\rmfamily]{Ligatures=TeX,Scale=1}
\fi
\usepackage[]{libertine}
\ifPDFTeX\else  
    % xetex/luatex font selection
\fi
% Use upquote if available, for straight quotes in verbatim environments
\IfFileExists{upquote.sty}{\usepackage{upquote}}{}
\IfFileExists{microtype.sty}{% use microtype if available
  \usepackage[]{microtype}
  \UseMicrotypeSet[protrusion]{basicmath} % disable protrusion for tt fonts
}{}
\makeatletter
\@ifundefined{KOMAClassName}{% if non-KOMA class
  \IfFileExists{parskip.sty}{%
    \usepackage{parskip}
  }{% else
    \setlength{\parindent}{0pt}
    \setlength{\parskip}{6pt plus 2pt minus 1pt}}
}{% if KOMA class
  \KOMAoptions{parskip=half}}
\makeatother
\usepackage{xcolor}
\usepackage[top=30mm,left=25mm,right=25mm,bottom=30mm,heightrounded]{geometry}
\setlength{\emergencystretch}{3em} % prevent overfull lines
\setcounter{secnumdepth}{5}
% Make \paragraph and \subparagraph free-standing
\ifx\paragraph\undefined\else
  \let\oldparagraph\paragraph
  \renewcommand{\paragraph}[1]{\oldparagraph{#1}\mbox{}}
\fi
\ifx\subparagraph\undefined\else
  \let\oldsubparagraph\subparagraph
  \renewcommand{\subparagraph}[1]{\oldsubparagraph{#1}\mbox{}}
\fi


\providecommand{\tightlist}{%
  \setlength{\itemsep}{0pt}\setlength{\parskip}{0pt}}\usepackage{longtable,booktabs,array}
\usepackage{calc} % for calculating minipage widths
% Correct order of tables after \paragraph or \subparagraph
\usepackage{etoolbox}
\makeatletter
\patchcmd\longtable{\par}{\if@noskipsec\mbox{}\fi\par}{}{}
\makeatother
% Allow footnotes in longtable head/foot
\IfFileExists{footnotehyper.sty}{\usepackage{footnotehyper}}{\usepackage{footnote}}
\makesavenoteenv{longtable}
\usepackage{graphicx}
\makeatletter
\def\maxwidth{\ifdim\Gin@nat@width>\linewidth\linewidth\else\Gin@nat@width\fi}
\def\maxheight{\ifdim\Gin@nat@height>\textheight\textheight\else\Gin@nat@height\fi}
\makeatother
% Scale images if necessary, so that they will not overflow the page
% margins by default, and it is still possible to overwrite the defaults
% using explicit options in \includegraphics[width, height, ...]{}
\setkeys{Gin}{width=\maxwidth,height=\maxheight,keepaspectratio}
% Set default figure placement to htbp
\makeatletter
\def\fps@figure{htbp}
\makeatother

% PDF Header - Custom LaTeX commands for PDF output
% This file is included in the LaTeX preamble for PDF generation

% Add any custom LaTeX packages or commands here
% Example: \usepackage{booktabs}
% Example: \usepackage{longtable}
\makeatletter
\@ifpackageloaded{caption}{}{\usepackage{caption}}
\AtBeginDocument{%
\ifdefined\contentsname
  \renewcommand*\contentsname{Table of contents}
\else
  \newcommand\contentsname{Table of contents}
\fi
\ifdefined\listfigurename
  \renewcommand*\listfigurename{List of Figures}
\else
  \newcommand\listfigurename{List of Figures}
\fi
\ifdefined\listtablename
  \renewcommand*\listtablename{List of Tables}
\else
  \newcommand\listtablename{List of Tables}
\fi
\ifdefined\figurename
  \renewcommand*\figurename{Figure}
\else
  \newcommand\figurename{Figure}
\fi
\ifdefined\tablename
  \renewcommand*\tablename{Table}
\else
  \newcommand\tablename{Table}
\fi
}
\@ifpackageloaded{float}{}{\usepackage{float}}
\floatstyle{ruled}
\@ifundefined{c@chapter}{\newfloat{codelisting}{h}{lop}}{\newfloat{codelisting}{h}{lop}[chapter]}
\floatname{codelisting}{Listing}
\newcommand*\listoflistings{\listof{codelisting}{List of Listings}}
\captionsetup{labelsep=colon}
\makeatother
\makeatletter
\makeatother
\makeatletter
\@ifpackageloaded{caption}{}{\usepackage{caption}}
\@ifpackageloaded{subcaption}{}{\usepackage{subcaption}}
\makeatother
\ifLuaTeX
\usepackage[bidi=basic]{babel}
\else
\usepackage[bidi=default]{babel}
\fi
\babelprovide[main,import]{american}
% get rid of language-specific shorthands (see #6817):
\let\LanguageShortHands\languageshorthands
\def\languageshorthands#1{}
\ifLuaTeX
  \usepackage{selnolig}  % disable illegal ligatures
\fi
\usepackage{bookmark}

\IfFileExists{xurl.sty}{\usepackage{xurl}}{} % add URL line breaks if available
\urlstyle{same} % disable monospaced font for URLs
\hypersetup{
  pdftitle={Statistical Validation of Astrological and Numerological Precursors in Global Seismic Activity},
  pdfauthor={Bishal Ghimire},
  pdflang={en-US},
  pdfsubject={Computational Integration of Vedic Numerology and Sidereal
Astrology},
  pdfkeywords={Seismology, Generalized Linear Models, Numerology, Monte
Carlo Verification, Anomaly Detection},
  colorlinks=true,
  linkcolor={blue},
  filecolor={Maroon},
  citecolor={green},
  urlcolor={blue},
  pdfcreator={LaTeX via pandoc}}

\title{Statistical Validation of Astrological and Numerological
Precursors in Global Seismic Activity}
\usepackage{etoolbox}
\makeatletter
\providecommand{\subtitle}[1]{% add subtitle to \maketitle
  \apptocmd{\@title}{\par {\large #1 \par}}{}{}
}
\makeatother
\subtitle{A Negative Binomial Regression Analysis of the USGS Catalog
(1900-2023)}
\author{Astro-Fusion Research Team}
\date{2026-01-30}

\begin{document}
\maketitle
\begin{abstract}
This study conducts a rigorous statistical audit of the ``Astro-Fusion''
hypothesis, which posits that discrete numerological time cycles and
planetary configurations (specifically Vedic \emph{Shadbala}) act as
exogenous triggers for global seismicity. Using a Negative Binomial
Generalized Linear Model (GLM) to account for overdispersion in the USGS
earthquake catalog, we test whether these variables provide information
gain over a standard Poisson baseline of seismicity (Trend +
Seasonality). Our analysis focuses on the prediction of daily earthquake
counts (M5.0+) and employs a ``Look-Elsewhere'' effect methodology
(Monte Carlo randomization) to control for false discoveries. The
results indicate that while certain high-magnitude events align with
classical astrological ``Yogas,'' the global statistical signal for
numerological precursors remains weak when controlled for multiple
hypothesis testing.
\end{abstract}

% PDF Before Body - Content to include before document body
% This file is included before the main document content

% Add any content that should appear before the main document here
% This could include custom title pages, abstracts, etc.

\renewcommand*\contentsname{Table of contents}
{
\hypersetup{linkcolor=}
\setcounter{tocdepth}{3}
\tableofcontents
}
\setstretch{1.2}
\section{Introduction}\label{introduction}

The prediction of earthquakes remains one of the most persistent
challenges in geophysical science. The ``Prediction Gap''---the
inability to deterministically forecast the time, location, and
magnitude of rupture---stems from the non-linear, chaotic nature of
crustal stress accumulation.

This paper investigates a novel set of candidate features derived from
``Astro-Fusion'' variables, specifically: 1. \textbf{Universal Day
Numbers (UDN):} A cyclic base-9 numerological time series. 2.
\textbf{Planetary Force Vectors:} Quantified ``Shadbala'' strength
scores for major planets (Mars, Saturn) derived from the Swiss
Ephemeris.

We do not presuppose the physical mechanism but treat these as potential
exogenous regressors in an Exploratory Data Analysis (EDA) framework.

\section{Data \& Methodology}\label{data-methodology}

\subsection{Data Preprocessing}\label{data-preprocessing}

We utilized the USGS Comprehensive Earthquake Catalog (ComCat),
filtering for events with \(Magnitude \ge 5.0\) from 1900 to 2023. Key
preprocessing steps included: - \textbf{Magnitude Homogenization:}
Converting all magnitudes to Moment Magnitude (\(M_w\)) to ensure
consistency. - \textbf{Declustering:} Applying the Gardner-Knopoff
algorithm to remove aftershocks, isolating independent mainshocks.

\subsection{Statistical Framework: Negative Binomial
Regression}\label{statistical-framework-negative-binomial-regression}

Earthquake counts are discrete, non-negative integers. Traditional OLS
regression is inappropriate. We employ a \textbf{Generalized Linear
Model (GLM)}. Because earthquake counts exhibit overdispersion
(\(Variance > Mean\)) due to clustering, we utilize the Negative
Binomial distribution rather than the Poisson distribution.

The model specification is:
\[ \ln(\mu_t) = \beta_0 + \beta_{trend} t + \beta_{season} \sin(\frac{2\pi t}{365}) + \beta_{astro} X_{astro} \]

\subsection{\texorpdfstring{The Null Hypothesis
(\(H_0\))}{The Null Hypothesis (H\_0)}}\label{the-null-hypothesis-h_0}

The Null Hypothesis states that earthquake occurrence is a random
Poisson process influenced only by: - Long-term catalog detection
improvement (Trend). - Seasonal tidal stress (Annual Seasonality). -
Random noise.

Any significant deviation in \(\beta_{astro}\) would suggest a signal.

\section{Results}\label{results}

\subsection{Regression Analysis}\label{regression-analysis}

Table 1 presents the estimated coefficients from the GLM.

\begin{longtable}[]{@{}
  >{\raggedright\arraybackslash}p{(\columnwidth - 6\tabcolsep) * \real{0.4000}}
  >{\raggedright\arraybackslash}p{(\columnwidth - 6\tabcolsep) * \real{0.2000}}
  >{\raggedright\arraybackslash}p{(\columnwidth - 6\tabcolsep) * \real{0.2000}}
  >{\raggedright\arraybackslash}p{(\columnwidth - 6\tabcolsep) * \real{0.2000}}@{}}
\caption{Table 1: Regression Coefficients}\tabularnewline
\toprule\noalign{}
\begin{minipage}[b]{\linewidth}\raggedright
Variable
\end{minipage} & \begin{minipage}[b]{\linewidth}\raggedright
Coefficient
\end{minipage} & \begin{minipage}[b]{\linewidth}\raggedright
Standard Error
\end{minipage} & \begin{minipage}[b]{\linewidth}\raggedright
P-Value
\end{minipage} \\
\midrule\noalign{}
\endfirsthead
\toprule\noalign{}
\begin{minipage}[b]{\linewidth}\raggedright
Variable
\end{minipage} & \begin{minipage}[b]{\linewidth}\raggedright
Coefficient
\end{minipage} & \begin{minipage}[b]{\linewidth}\raggedright
Standard Error
\end{minipage} & \begin{minipage}[b]{\linewidth}\raggedright
P-Value
\end{minipage} \\
\midrule\noalign{}
\endhead
\bottomrule\noalign{}
\endlastfoot
Intercept & -12.45 & 0.05 & \textless{} 0.001 \\
Year Index & 0.012 & 0.001 & \textless{} 0.001 \\
Universal Day 8 & 0.045 & 0.03 & 0.15 \\
Mars Strength & -0.002 & 0.001 & 0.04 \\
Saturn Strength & 0.001 & 0.001 & 0.32 \\
\end{longtable}

\emph{(Note: See \texttt{tables/regression\_coefficients.csv} for full
model output)}

\subsection{Temporal Prediction}\label{temporal-prediction}

Figure 1 below illustrates the model's ability to track the aggregate
monthly seismic rate. The ``Prediction Gap'' is visible where the model
captures the trend but misses extreme outliers (Mega-Quakes).

\begin{figure}[H]

\caption{Figure 1: Time series of Predicted Rate vs.~Actual Rate
(Monthly Aggregated). The red dashed line represents the Negative
Binomial model fit.}

{\centering \includegraphics{../figures/predicted_vs_actual.png}

}

\end{figure}%

\subsection{``Look-Elsewhere'' Analysis}\label{look-elsewhere-analysis}

To ensure any detected signal is not a result of p-hacking (finding a
pattern by checking thousands of combinations), we performed a Monte
Carlo Shuffle Test (N=1000). We destroyed the time-link between
astrology and earthquakes while preserving the internal structure of
both series.

\begin{figure}[H]

\caption{Figure 2: The ``Look-Elsewhere'' Analysis. The histogram shows
the distribution of Delta-AIC values for random shuffles. The red line
marks the Real Data performance. If the red line is within the gray
mass, the result is indistinguishable from noise.}

{\centering \includegraphics{../figures/monte_carlo_distribution.png}

}

\end{figure}%

\section{Discussion}\label{discussion}

\subsection{Interpretation}\label{interpretation}

The coefficient for \textbf{Mars Strength} shows a marginally
significant negative correlation (p=0.04), suggesting that periods of
\emph{lower} Mars strength slightly correlate with higher seismicity in
this specific model. However, the \textbf{Universal Day Number}
variables did not reach the threshold of statistical significance
(\(p < 0.05\)).

\subsection{Confounders}\label{confounders}

Potential confounders include: - \textbf{Solar Cycle:} The 11-year solar
maximum cycle may overlap with planetary periods (Jupiter). -
\textbf{Tidal Stress:} Lunar phases (syzygy) are known physical triggers
but are often conflated with ``astrology.''

\subsection{Critical Self-Assessment}\label{critical-self-assessment}

The Monte Carlo analysis (Figure 2) reveals that the improvement in
model fit (Delta-AIC) provided by the Astro-Fusion variables falls
within the 95\% confidence interval of random noise. Therefore, strictly
speaking, \textbf{we fail to reject the Null Hypothesis} for the global
dataset. The signal provided by these variables is not strong enough to
be used as a standalone predictor for global seismicity.

\section{Conclusion}\label{conclusion}

This study provides a reproducible computational framework for testing
``fringe'' scientific hypotheses. While we found anecdotal correlations
in individual high-magnitude events, the aggregate statistical analysis
suggests that ``Universal Day Numbers'' and ``Global Planetary
Strength'' do not possess significant predictive power for global
earthquake rates. Future work should focus on regionalized hypotheses
(Koorma Chakra) rather than global aggregates.



\end{document}
