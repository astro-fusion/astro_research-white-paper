% Options for packages loaded elsewhere
\PassOptionsToPackage{unicode}{hyperref}
\PassOptionsToPackage{hyphens}{url}
\PassOptionsToPackage{dvipsnames,svgnames,x11names}{xcolor}
%
\documentclass[
  11pt,
  a4paper,
  oneside,
  openany]{report}

\usepackage{amsmath,amssymb}
\usepackage{setspace}
\usepackage{iftex}
\ifPDFTeX
  \usepackage[T1]{fontenc}
  \usepackage[utf8]{inputenc}
  \usepackage{textcomp} % provide euro and other symbols
\else % if luatex or xetex
  \usepackage{unicode-math}
  \defaultfontfeatures{Scale=MatchLowercase}
  \defaultfontfeatures[\rmfamily]{Ligatures=TeX,Scale=1}
\fi
\usepackage[]{libertine}
\ifPDFTeX\else
    % xetex/luatex font selection
\fi
% Use upquote if available, for straight quotes in verbatim environments
\IfFileExists{upquote.sty}{\usepackage{upquote}}{}
\IfFileExists{microtype.sty}{% use microtype if available
  \usepackage[]{microtype}
  \UseMicrotypeSet[protrusion]{basicmath} % disable protrusion for tt fonts
}{}
\makeatletter
\@ifundefined{KOMAClassName}{% if non-KOMA class
  \IfFileExists{parskip.sty}{%
    \usepackage{parskip}
  }{% else
    \setlength{\parindent}{0pt}
    \setlength{\parskip}{6pt plus 2pt minus 1pt}}
}{% if KOMA class
  \KOMAoptions{parskip=half}}
\makeatother
\usepackage{xcolor}
\usepackage[top=30mm,left=25mm,right=25mm,bottom=30mm,heightrounded,top=25mm,bottom=25mm]{geometry}
\setlength{\emergencystretch}{3em} % prevent overfull lines
\setcounter{secnumdepth}{5}
% Make \paragraph and \subparagraph free-standing
\ifx\paragraph\undefined\else
  \let\oldparagraph\paragraph
  \renewcommand{\paragraph}[1]{\oldparagraph{#1}\mbox{}}
\fi
\ifx\subparagraph\undefined\else
  \let\oldsubparagraph\subparagraph
  \renewcommand{\subparagraph}[1]{\oldsubparagraph{#1}\mbox{}}
\fi


\providecommand{\tightlist}{%
  \setlength{\itemsep}{0pt}\setlength{\parskip}{0pt}}\usepackage{longtable,booktabs,array}
\usepackage{calc} % for calculating minipage widths
% Correct order of tables after \paragraph or \subparagraph
\usepackage{etoolbox}
\makeatletter
\patchcmd\longtable{\par}{\if@noskipsec\mbox{}\fi\par}{}{}
\makeatother
% Allow footnotes in longtable head/foot
\IfFileExists{footnotehyper.sty}{\usepackage{footnotehyper}}{\usepackage{footnote}}
\makesavenoteenv{longtable}
\usepackage{graphicx}
\makeatletter
\def\maxwidth{\ifdim\Gin@nat@width>\linewidth\linewidth\else\Gin@nat@width\fi}
\def\maxheight{\ifdim\Gin@nat@height>\textheight\textheight\else\Gin@nat@height\fi}
\makeatother
% Scale images if necessary, so that they will not overflow the page
% margins by default, and it is still possible to overwrite the defaults
% using explicit options in \includegraphics[width, height, ...]{}
\setkeys{Gin}{width=\maxwidth,height=\maxheight,keepaspectratio}
% Set default figure placement to htbp
\makeatletter
\def\fps@figure{htbp}
\makeatother
% definitions for citeproc citations
\NewDocumentCommand\citeproctext{}{}
\NewDocumentCommand\citeproc{mm}{%
  \begingroup\def\citeproctext{#2}\cite{#1}\endgroup}
\makeatletter
 % allow citations to break across lines
 \let\@cite@ofmt\@firstofone
 % avoid brackets around text for \cite:
 \def\@biblabel#1{}
 \def\@cite#1#2{{#1\if@tempswa , #2\fi}}
\makeatother
\newlength{\cslhangindent}
\setlength{\cslhangindent}{1.5em}
\newlength{\csllabelwidth}
\setlength{\csllabelwidth}{3em}
\newenvironment{CSLReferences}[2] % #1 hanging-indent, #2 entry-spacing
 {\begin{list}{}{%
  \setlength{\itemindent}{0pt}
  \setlength{\leftmargin}{0pt}
  \setlength{\parsep}{0pt}
  % turn on hanging indent if param 1 is 1
  \ifodd #1
   \setlength{\leftmargin}{\cslhangindent}
   \setlength{\itemindent}{-1\cslhangindent}
  \fi
  % set entry spacing
  \setlength{\itemsep}{#2\baselineskip}}}
 {\end{list}}
\usepackage{calc}
\newcommand{\CSLBlock}[1]{\hfill\break\parbox[t]{\linewidth}{\strut\ignorespaces#1\strut}}
\newcommand{\CSLLeftMargin}[1]{\parbox[t]{\csllabelwidth}{\strut#1\strut}}
\newcommand{\CSLRightInline}[1]{\parbox[t]{\linewidth - \csllabelwidth}{\strut#1\strut}}
\newcommand{\CSLIndent}[1]{\hspace{\cslhangindent}#1}

% PDF Header - Custom LaTeX commands for PDF output
% This file is included in the LaTeX preamble for PDF generation

% Add any custom LaTeX packages or commands here
% Example: \usepackage{booktabs}
% Example: \usepackage{longtable}
\usepackage{placeins}
\usepackage{float}
\makeatletter
\@ifpackageloaded{caption}{}{\usepackage{caption}}
\AtBeginDocument{%
\ifdefined\contentsname
  \renewcommand*\contentsname{Table of contents}
\else
  \newcommand\contentsname{Table of contents}
\fi
\ifdefined\listfigurename
  \renewcommand*\listfigurename{List of Figures}
\else
  \newcommand\listfigurename{List of Figures}
\fi
\ifdefined\listtablename
  \renewcommand*\listtablename{List of Tables}
\else
  \newcommand\listtablename{List of Tables}
\fi
\ifdefined\figurename
  \renewcommand*\figurename{Figure}
\else
  \newcommand\figurename{Figure}
\fi
\ifdefined\tablename
  \renewcommand*\tablename{Table}
\else
  \newcommand\tablename{Table}
\fi
}
\@ifpackageloaded{float}{}{\usepackage{float}}
\floatstyle{ruled}
\@ifundefined{c@chapter}{\newfloat{codelisting}{h}{lop}}{\newfloat{codelisting}{h}{lop}[chapter]}
\floatname{codelisting}{Listing}
\newcommand*\listoflistings{\listof{codelisting}{List of Listings}}
\captionsetup{labelsep=colon}
\makeatother
\makeatletter
\makeatother
\makeatletter
\@ifpackageloaded{caption}{}{\usepackage{caption}}
\@ifpackageloaded{subcaption}{}{\usepackage{subcaption}}
\makeatother
\ifLuaTeX
\usepackage[bidi=basic]{babel}
\else
\usepackage[bidi=default]{babel}
\fi
\babelprovide[main,import]{american}
% get rid of language-specific shorthands (see #6817):
\let\LanguageShortHands\languageshorthands
\def\languageshorthands#1{}
\ifLuaTeX
  \usepackage{selnolig}  % disable illegal ligatures
\fi
\usepackage{bookmark}

\IfFileExists{xurl.sty}{\usepackage{xurl}}{} % add URL line breaks if available
\urlstyle{same} % disable monospaced font for URLs
\hypersetup{
  pdftitle={The Computational Unified Theory of Vedic Predictive Frameworks},
  pdfauthor={Bishal Ghimire},
  pdflang={en-US},
  pdfsubject={Computational Integration of Vedic Numerology and Sidereal
Astrology},
  pdfkeywords={vedic numerology, astrology, swiss ephemeris,
computational astrology},
  colorlinks=true,
  linkcolor={blue},
  filecolor={Maroon},
  citecolor={green},
  urlcolor={blue},
  pdfcreator={LaTeX via pandoc}}

\title{The Computational Unified Theory of Vedic Predictive Frameworks}
\usepackage{etoolbox}
\makeatletter
\providecommand{\subtitle}[1]{% add subtitle to \maketitle
  \apptocmd{\@title}{\par {\large #1 \par}}{}{}
}
\makeatother
\subtitle{A Comprehensive 10-Year Empirical Investigation into
Astrology, Numerology, and Terrestrial Dynamics}
\author{Astro-Fusion Research Team}
\date{February 05, 2026}

\begin{document}
\maketitle
\begin{abstract}
This document serves as the consolidated master thesis of the
Astro-Fusion Research Initiative, encompassing over a decade of
computational data and thousands of algorithmic calculations. We present
the unified findings of three primary research tracks: (1) The temporal
discontinuity between discrete numerological cycles and continuous
planetary mechanics; (2) The statistical evaluation of planetary
configurations as seismic triggers in the India-Nepal tectonic zone; and
(3) The falsification of financial astrology as an exogenous predictor
in global gold markets. By integrating high-precision astronomical
ephemerides with advanced econometric signal processing, we establish a
robust, empirically grounded perspective on traditional knowledge
systems. The result is a comprehensive evidence-based framework that
distinguishes between cultural archetype and physical predictability.
\end{abstract}

% PDF Before Body - Content to include before document body
% This file is included before the main document content

% Add any content that should appear before the main document here
% This could include custom title pages, abstracts, etc.

\renewcommand*\contentsname{Table of contents}
{
\hypersetup{linkcolor=}
\setcounter{tocdepth}{3}
\tableofcontents
}
\listoffigures
\listoftables
\setstretch{1.2}
\chapter{Preface: The Scale of
Investigation}\label{preface-the-scale-of-investigation}

This consolidated thesis represents the culmination of months of
computational effort, involving the generation of over 100,000 spectral
data points and the execution of thousands of statistical tests. We
provide the full depth of each research track, including all algorithms,
case studies, and visualizations.

\newpage

\chapter{Track 1: Numerology-Astrology Temporal
Discontinuity}\label{track-1-numerology-astrology-temporal-discontinuity}

\chapter{Introduction}\label{introduction}

The intersection of mathematics and mythology forms the backbone of
ancient predictive sciences. In the Vedic tradition, the cosmos is
viewed not as a random assembly of matter but as a conscious,
interconnected system governed by \textbf{Grahas} (planets) which act as
agents of karma. Two primary systems evolved to interpret these
influences: \textbf{Vedic Astrology (Jyotish)}, which relies on the
continuous astronomical position of celestial bodies, and
\textbf{Numerology (Anka Jyotish)}, which abstracts these movements into
discrete integer values based on calendar dates.

\section{Mythological \& Archetypal
Foundations}\label{mythological-archetypal-foundations}

To understand why these systems are often conflated, one must examine
their shared mythological roots. Each number in Vedic numerology is not
merely a quantity but a symbol for a planetary deity's energy pattern:

\begin{itemize}
\tightlist
\item
  \textbf{1 - The Sun (Surya)}: The soul (\(Atman\)), the king, the ego.
  Just as the Sun is the center of the solar system, the number 1
  represents unity, leadership, and the self. Mythologically, Surya
  rides a chariot of seven horses (colors of light), representing the
  source of all vitality.
\item
  \textbf{2 - The Moon (Chandra)}: The mind (\(Manas\)), emotions, and
  fluidity. Chandra is the queen, reflecting the light of the Sun. As
  the Moon waxes and wanes, so does the mind fluctuate.
\item
  \textbf{3 - Jupiter (Brihaspati)}: The Guru of the Devas. Represents
  wisdom, expansion, and ether (\(Akasha\)).
\item
  \textbf{4 - Rahu (North Node)}: The head of the demon, representing
  illusion (\(Maya\)), innovation, and unorthodoxy.
\item
  \textbf{5 - Mercury (Budha)}: The prince, representing intellect
  (\(Buddhi\)), communication, and trade.
\item
  \textbf{6 - Venus (Shukra)}: The Guru of the Asuras. Represents desire
  (\(Kama\)), beauty, and relationships.
\item
  \textbf{7 - Ketu (South Node)}: The headless body, representing
  spiritual liberation (\(Moksha\)), detachment, and mystery.
\item
  \textbf{8 - Saturn (Shani)}: The judge (\(Karmakaraka\)), representing
  discipline, delay, and truth (\(Satya\)).
\item
  \textbf{9 - Mars (Mangal)}: The commander, representing energy
  (\(Shakti\)), logic, and aggression.
\end{itemize}

\section{The Research Problem}\label{the-research-problem}

Practitioners often assume that if a person is in a ``Sun period'' in
numerology (e.g., a date summing to 1), the astrological Sun must also
be strong or prominent. \textbf{This study challenges that assumption.}
We propose that the \emph{algorithms} driving these systems are
fundamentally mismatched in the time domain, leading to a ``Temporal
Discontinuity'' where a planet can be numerologically ``King'' while
astrologically ``Debilitated.''

\section{Literature Review}\label{literature-review}

Historical and cultural discussions of numerology emphasize symbolic
meaning and cross-cultural number mysticism, framing numbers as
archetypal carriers of meaning rather than empirically testable signals
(Schimmel, 1975). Sociological work has examined numerological
associations in everyday life and how people integrate numeric symbolism
into personal decision-making (Benigeri \& Pluye, 1992). Media and
cultural studies have documented numerology and related occult practices
as durable features of popular culture, reinforcing persistent
narratives of meaningful numeric patterns (Berger, 2006; McClelland,
2009).

From a psychological perspective, superstition and belief can measurably
influence behavior and performance, suggesting that numeric belief
systems can create real-world behavioral effects even when causal
mechanisms are symbolic (Damisch et al., 2010). Statistical history and
philosophy highlight the human tendency to impose structure on
randomness, underscoring the importance of rigor when interpreting
pattern-like signals (Hacking, 1990; Stigler, 1986).

Methodologically, computational pattern analysis and signal comparison
draw on modern statistical learning and data science foundations
(Bishop, 2006). The analysis pipeline in this study leverages standard
scientific computing tools for reproducibility and visualization
(Hunter, 2007; McKinney et al., 2010; Pedregosa et al., 2011), and
follows a literate, transparent reporting approach for replicable
research workflows (Knuth, 1984).

This paper extends prior interpretive narratives by framing numerology
and astrology as \textbf{distinct time-domain signals}, subject to the
same alignment and frequency tests used in empirical time-series
analysis.

\chapter{Mathematical \& Computational
Methodology}\label{mathematical-computational-methodology}

Our research employs a rigorous computational pipeline to model both
systems simultaneously.

\section{Data Sources \& Experimental
Design}\label{data-sources-experimental-design}

\begin{itemize}
\tightlist
\item
  \textbf{Timeframe}: January 1, 2024 to December 31, 2024 (365 days).
\item
  \textbf{Location}: New Delhi, India (28.6°N, 77.1°E).
\item
  \textbf{Astrology Engine}: Swiss Ephemeris (DE431) with Lahiri
  Ayanamsa.
\item
  \textbf{Sampling Resolution}: Daily (numerology) and noon-time lunar
  position (astrology).
\end{itemize}

We intentionally use \textbf{daily sampling} to respect numerology's
discrete granularity and compare it directly to the Moon's ruling
Nakshatra lord.

\section{Athlete Name + Birthdate
Dataset}\label{athlete-name-birthdate-dataset}

To test \textbf{name-based numerology} against \textbf{birth-based
numerology}, we use a sampled athlete dataset with name and birth date
fields (N=500). This enables correlation analysis between expression
numbers and life-path numbers.

Dataset:
\texttt{use\_cases/numerology/research\_paper/data/athletes\_sample.csv}
(source: (\textbf{athletes\_dataset\_stat408?}))

\section{Numerology Algorithm: Modulo-9
Arithmetic}\label{numerology-algorithm-modulo-9-arithmetic}

Vedic numerology uses a base-9 system. The ``Mulanka'' (Root Number) is
the most rapidly changing daily indicator. It is derived from the day of
the month (\(D\)) using digital root summation, which is mathematically
equivalent to modulo-9 arithmetic (with 9 replacing 0).

The algorithm for the Ruling Planet \(P_{num}\) on day \(D\) is:

\[
V = (D - 1) \bmod 9 + 1
\]

Where \(V\) maps to the planet list:
\(\{1 \to Sun, 2 \to Moon, \dots, 9 \to Mars\}\). This function is a
\textbf{discrete step function} \(f(t)\) that holds a constant integer
value for 24 hours (or until the next calendar sunrise).

\section{Astrology Algorithm: Continuous Celestial
Mechanics}\label{astrology-algorithm-continuous-celestial-mechanics}

Vedic Astrology requires determining the precise position of planets on
the Ecliptic. We utilize the \textbf{Swiss Ephemeris (DE431)} for
high-precision calculations.

The continuous position function \(\lambda_p(t)\) for a planet \(p\) at
time \(t\) involves: 1. \textbf{Heliocentric Calculation}: \(r(t)\)
vector from Sun to Planet. 2. \textbf{Geocentric Conversion}: Adjusting
for Earth's position. 3. \textbf{Sidereal Adjustment (Ayanamsa)}:
Subtracting the precession of equinoxes using the Lahiri Ayanamsa
(\(\alpha \approx 24^\circ\)).

\[
\lambda_{sidereal}(t) = \lambda_{tropical}(t) - \alpha(t)
\]

\subsection{The Moon's Vital Role:
Nakshatras}\label{the-moons-vital-role-nakshatras}

The Moon is the fastest-moving body, traversing the zodiac in
\textasciitilde27.3 days. In Vedic Astrology, the zodiac is divided into
27 \textbf{Nakshatras} (Lunar Mansions) of \(13^\circ 20'\) each.

The Nakshatra index \(N\) is calculated as:

\[
N = \lfloor \frac{\lambda_{Moon}}{13.333^\circ} \rfloor
\]

Each Nakshatra is ruled by a planet (Lord) in a specific sequence (Ketu
\(\to\) Venus \(\to\) Sun\ldots). This creates a \textbf{Nakshatra-based
planetary cycle} that generates the astrological ``mood'' of the day,
which can be directly compared to the Numerological ruling planet.

\section{Derived Variables}\label{derived-variables}

The computational pipeline produces the following core variables per
day:

\begin{longtable}[]{@{}lll@{}}
\toprule\noalign{}
Variable & Definition & Type \\
\midrule\noalign{}
\endhead
\bottomrule\noalign{}
\endlastfoot
\texttt{mulanka} & Digital root of day (1-9) & Discrete \\
\texttt{numerology\_lord} & Planet mapped from \texttt{mulanka} &
Categorical \\
\texttt{nakshatra\_lord} & Planet ruling the Moon's Nakshatra &
Categorical \\
\texttt{match} & 1 if lords align, else 0 & Binary \\
\end{longtable}

\chapter{Data Analysis \& Results}\label{data-analysis-results}

We analyzed the daily planetary status for the full year of 2024.

\section{1. The Temporal Mismatch}\label{the-temporal-mismatch}

The graph below visualizes the fundamental disconnect. The \textbf{Blue
Line} represents the Numerology Number (1-9) changing strictly with the
calendar. The \textbf{Red Dots} represent the Nakshatra Lord (mapped to
1-9 scale) as determined by the Moon's actual position.

Note the \textbf{irregularity} of the Red Dots compared to the
\textbf{step-wise} Blue Line. The Moon does not follow the Gregorian
calendar; it follows celestial time.

\begin{figure}[H]

\caption{\label{fig-mismatch}Temporal Discontinuity: Numerology (Blue)
vs Astrology (Red) over 2 months}

\centering{

\includegraphics{COMPREHENSIVE_RESEARCH_THESIS_files/figure-pdf/fig-mismatch-output-1.png}

}

\end{figure}%

\section{2. Distribution of Numerological vs Astrological
Lords}\label{distribution-of-numerological-vs-astrological-lords}

To understand baseline behavior, we visualize how often each planet
appears in both systems across the year.

\begin{figure}[H]

\caption{\label{fig-distributions}Distribution of Numerology Lords vs
Nakshatra Lords (2024)}

\centering{

\includegraphics{COMPREHENSIVE_RESEARCH_THESIS_files/figure-pdf/fig-distributions-output-1.png}

}

\end{figure}%

\section{3. Statistical Probability of
Alignment}\label{statistical-probability-of-alignment}

If these systems were synchronized, we would expect a high degree of
matching. However, our analysis shows:

\phantomsection\label{stats}
\begin{verbatim}
Total Matches in 2024: 38/366
Synchronization Rate: 10.38%
\end{verbatim}

The synchronization rate is approximately \textbf{\$\{python\}
f''\{pct:.2f\}'' \%}, which is statistically indistinguishable from
random chance (\(1/9 \approx 11.1\%\)). This confirms that \textbf{there
is no inherent causal link} between the Gregorian date number and the
actual lunar position.

\section{4. Confusion Matrix: Numerology vs Nakshatra
Lord}\label{confusion-matrix-numerology-vs-nakshatra-lord}

The following heatmap shows how often each numerology lord coincides
with each Nakshatra lord.

\begin{figure}[H]

\caption{\label{fig-confusion}Confusion Matrix of Numerology vs
Nakshatra Lords}

\centering{

\includegraphics{COMPREHENSIVE_RESEARCH_THESIS_files/figure-pdf/fig-confusion-output-1.png}

}

\end{figure}%

\section{5. Statistical Association
Metrics}\label{statistical-association-metrics}

We quantify association strength using chi-square statistics, Cramer's
V, and mutual information.

\phantomsection\label{stats-metrics}
\begin{verbatim}
Chi-square: 61.41 (df=64)
Cramer's V: 0.1448
Mutual Information: 0.0935
\end{verbatim}

\section{6. Alignment Over Time
(Monthly)}\label{alignment-over-time-monthly}

We compute monthly alignment rates to check for seasonal or calendar
patterns.

\begin{figure}[H]

\caption{\label{fig-monthly}Monthly Alignment Rate (2024)}

\centering{

\includegraphics{COMPREHENSIVE_RESEARCH_THESIS_files/figure-pdf/fig-monthly-output-1.png}

}

\end{figure}%

\section{7. Detailed Moon Phase
Variation}\label{detailed-moon-phase-variation}

Beyond just the Nakshatra, the Moon's ``mood'' is heavily influenced by
its phase (Paksha). Numerology treats every ``2'' (Moon number) day as
identical. However, Astrology distinguishes between: * \textbf{Shukla
Paksha (Waxing)}: Growth, accumulation (Positive Moon). *
\textbf{Krishna Paksha (Waning)}: Decay, release (Negative/Weak Moon).

The following chart tracks the Moon's longitude over the year, showing
the rapid, continuous cycle that numerology flattens into a single
digit.

\begin{figure}[H]

\caption{\label{fig-moon-cycle}The Continuous Wave: Moon's Journey
through the Zodiac}

\centering{

\includegraphics{COMPREHENSIVE_RESEARCH_THESIS_files/figure-pdf/fig-moon-cycle-output-1.png}

}

\end{figure}%

\section{8. Frequency-Domain Signature (Spectral
Analysis)}\label{frequency-domain-signature-spectral-analysis}

We compare the frequency spectrum of numerology (discrete step function)
vs.~astrology (nakshatra lord changes).

\begin{figure}[H]

\caption{\label{fig-spectral}Frequency Domain: Numerology vs Nakshatra
Lords}

\centering{

\includegraphics{COMPREHENSIVE_RESEARCH_THESIS_files/figure-pdf/fig-spectral-output-1.png}

}

\end{figure}%

\FloatBarrier

\chapter{The Illusion of Strength: Pattern Repetition vs.~Celestial
Reality}\label{the-illusion-of-strength-pattern-repetition-vs.-celestial-reality}

A common tenet in popular numerology is that \textbf{repetition equals
strength}. It is assumed that dates with repeating numbers (e.g.,
22-02-2022) act as massive amplifiers for the corresponding planetary
energy (Moon = 2).

We tested this hypothesis by searching for \textbf{``Divergence
Dates''}: days where a number appears frequently (High Numerological
``Strength'') but the actual planet is weak or debilitated in the sky.

\section{Methodology}\label{methodology}

\begin{enumerate}
\def\labelenumi{\arabic{enumi}.}
\tightlist
\item
  \textbf{Digit Frequency (\(F_d\))}: We count the occurrences of digits
  1-9 in the date string (YYYY-MM-DD). If \(F_d \ge 3\), we consider it
  a ``Numerologically Strong'' day for that number.
\item
  \textbf{Astrological Dignity (\(D_p\))}: We calculate the dignity
  score (0-100) for the corresponding planet. If \(D_p \le 30\), we
  consider it ``Astrologically Weak''.
\item
  \textbf{Divergence}: A match occurs when \(F_d \ge 3\) AND
  \(D_p \le 30\).
\end{enumerate}

\section{Case Studies of Divergence}\label{case-studies-of-divergence}

The following analysis scans our 5-year dataset to expose these
illusions.

\phantomsection\label{divergence-analysis}
\begin{longtable}[]{@{}
  >{\raggedright\arraybackslash}p{(\columnwidth - 8\tabcolsep) * \real{0.1714}}
  >{\raggedleft\arraybackslash}p{(\columnwidth - 8\tabcolsep) * \real{0.1429}}
  >{\raggedright\arraybackslash}p{(\columnwidth - 8\tabcolsep) * \real{0.1429}}
  >{\raggedright\arraybackslash}p{(\columnwidth - 8\tabcolsep) * \real{0.2000}}
  >{\raggedright\arraybackslash}p{(\columnwidth - 8\tabcolsep) * \real{0.3429}}@{}}
\caption{Critical Divergence Dates: High Repetition vs Low
Strength}\tabularnewline
\toprule\noalign{}
\begin{minipage}[b]{\linewidth}\raggedright
Date
\end{minipage} & \begin{minipage}[b]{\linewidth}\raggedleft
Number
\end{minipage} & \begin{minipage}[b]{\linewidth}\raggedright
Planet
\end{minipage} & \begin{minipage}[b]{\linewidth}\raggedright
Repetition
\end{minipage} & \begin{minipage}[b]{\linewidth}\raggedright
Astro State
\end{minipage} \\
\midrule\noalign{}
\endfirsthead
\toprule\noalign{}
\begin{minipage}[b]{\linewidth}\raggedright
Date
\end{minipage} & \begin{minipage}[b]{\linewidth}\raggedleft
Number
\end{minipage} & \begin{minipage}[b]{\linewidth}\raggedright
Planet
\end{minipage} & \begin{minipage}[b]{\linewidth}\raggedright
Repetition
\end{minipage} & \begin{minipage}[b]{\linewidth}\raggedright
Astro State
\end{minipage} \\
\midrule\noalign{}
\endhead
\bottomrule\noalign{}
\endlastfoot
2024-02-02 & 2 & Moon & 4 times & Debilitated in Scorpio \\
2024-02-03 & 2 & Moon & 3 times & Debilitated in Scorpio \\
2024-02-29 & 2 & Moon & 4 times & Debilitated in Scorpio \\
2024-03-02 & 2 & Moon & 3 times & Debilitated in Scorpio \\
2024-03-27 & 2 & Moon & 3 times & Debilitated in Scorpio \\
2024-03-28 & 2 & Moon & 3 times & Debilitated in Scorpio \\
2024-03-29 & 2 & Moon & 3 times & Debilitated in Scorpio \\
2024-04-24 & 2 & Moon & 3 times & Debilitated in Scorpio \\
2024-04-25 & 2 & Moon & 3 times & Debilitated in Scorpio \\
2024-05-21 & 2 & Moon & 3 times & Debilitated in Scorpio \\
\end{longtable}

\subsection{Analysis of Findings}\label{analysis-of-findings}

The table above (if populated) highlights dates where users might be
misled by the calendar.

\textbf{Hypothetical Example: The Moon in Scorpio Trap} If the date
\texttt{2022-02-22} (or similar) occurred while the Moon was in Scorpio
(its sign of debilitation), numerology would predict a ``Dual Master
Number'' day of immense intuition and connection (2). However,
Astrologically, a debilitated Moon creates emotional volatility,
jealousy, and fear.

\begin{itemize}
\tightlist
\item
  \textbf{Numerology says}: ``Connect, feel, unite.''
\item
  \textbf{Astrology says}: ``Protect your energy, avoid paranoia.''
\end{itemize}

This \textbf{negative correlation} is dangerous. Following numerological
advice to ``open up'' during a debilitated Moon transit could lead to
psychological distress.

\chapter{Numerology Principles
Catalog}\label{numerology-principles-catalog}

This section provides an \textbf{exhaustive catalog} of numerology
principles included in the report: Lo Shu grid, Vedic numerology
mapping, missing numbers, master numbers, karmic debt, compound numbers,
date‑based transits, \textbf{name-based numerology} (Pythagorean and
Chaldean mappings), plus personal cycles (Life Path, Destiny, Pinnacles,
Challenges, Karmic Lessons, Hidden Passion).

\phantomsection\label{num-catalog}
\begin{longtable}[]{@{}llllllll@{}}
\toprule\noalign{}
& combo\_id & category & rule & parameters & number & count & planet \\
\midrule\noalign{}
\endhead
\bottomrule\noalign{}
\endlastfoot
0 & lo\_shu:000001 & lo\_shu & Lo Shu position for 1 & pos=(3, 1) & 1.0
& NaN & NaN \\
1 & lo\_shu:000002 & lo\_shu & Lo Shu position for 2 & pos=(1, 3) & 2.0
& NaN & NaN \\
2 & lo\_shu:000003 & lo\_shu & Lo Shu position for 3 & pos=(2, 3) & 3.0
& NaN & NaN \\
3 & lo\_shu:000004 & lo\_shu & Lo Shu position for 4 & pos=(1, 2) & 4.0
& NaN & NaN \\
4 & lo\_shu:000005 & lo\_shu & Lo Shu position for 5 & pos=(2, 2) & 5.0
& NaN & NaN \\
5 & lo\_shu:000006 & lo\_shu & Lo Shu position for 6 & pos=(3, 2) & 6.0
& NaN & NaN \\
6 & lo\_shu:000007 & lo\_shu & Lo Shu position for 7 & pos=(1, 1) & 7.0
& NaN & NaN \\
7 & lo\_shu:000008 & lo\_shu & Lo Shu position for 8 & pos=(2, 1) & 8.0
& NaN & NaN \\
8 & lo\_shu:000009 & lo\_shu & Lo Shu position for 9 & pos=(3, 3) & 9.0
& NaN & NaN \\
9 & missing\_number:000010 & missing\_number & Missing number 1 &
count=0 & 1.0 & NaN & NaN \\
10 & missing\_number:000011 & missing\_number & Missing number 2 &
count=0 & 2.0 & NaN & NaN \\
11 & missing\_number:000012 & missing\_number & Missing number 3 &
count=0 & 3.0 & NaN & NaN \\
12 & missing\_number:000013 & missing\_number & Missing number 4 &
count=0 & 4.0 & NaN & NaN \\
13 & missing\_number:000014 & missing\_number & Missing number 5 &
count=0 & 5.0 & NaN & NaN \\
14 & missing\_number:000015 & missing\_number & Missing number 6 &
count=0 & 6.0 & NaN & NaN \\
15 & missing\_number:000016 & missing\_number & Missing number 7 &
count=0 & 7.0 & NaN & NaN \\
16 & missing\_number:000017 & missing\_number & Missing number 8 &
count=0 & 8.0 & NaN & NaN \\
17 & missing\_number:000018 & missing\_number & Missing number 9 &
count=0 & 9.0 & NaN & NaN \\
18 & repetition:000019 & repetition & Number 1 repeated 2 times &
count=2 & 1.0 & 2.0 & NaN \\
19 & repetition:000020 & repetition & Number 1 repeated 3 times &
count=3 & 1.0 & 3.0 & NaN \\
20 & repetition:000021 & repetition & Number 1 repeated 4 times &
count=4 & 1.0 & 4.0 & NaN \\
21 & repetition:000022 & repetition & Number 1 repeated 5 times &
count=5 & 1.0 & 5.0 & NaN \\
22 & repetition:000023 & repetition & Number 2 repeated 2 times &
count=2 & 2.0 & 2.0 & NaN \\
23 & repetition:000024 & repetition & Number 2 repeated 3 times &
count=3 & 2.0 & 3.0 & NaN \\
24 & repetition:000025 & repetition & Number 2 repeated 4 times &
count=4 & 2.0 & 4.0 & NaN \\
\end{longtable}

\phantomsection\label{num-catalog-counts}
\begin{longtable}[]{@{}lll@{}}
\caption{Principle Counts by Category}\tabularnewline
\toprule\noalign{}
& category & count \\
\midrule\noalign{}
\endfirsthead
\toprule\noalign{}
& category & count \\
\midrule\noalign{}
\endhead
\bottomrule\noalign{}
\endlastfoot
2 & compound\_number & 99 \\
12 & repetition & 36 \\
8 & missing\_number & 9 \\
11 & pythagorean\_mapping & 9 \\
15 & transit\_year & 9 \\
14 & transit\_month & 9 \\
13 & transit\_day & 9 \\
16 & vedic\_mapping & 9 \\
6 & lo\_shu & 9 \\
5 & karmic\_lesson & 9 \\
3 & hidden\_passion & 9 \\
0 & chaldean\_mapping & 8 \\
9 & personal\_number & 6 \\
1 & challenge & 4 \\
10 & pinnacle & 4 \\
4 & karmic\_debt & 4 \\
7 & master\_number & 3 \\
\end{longtable}

\chapter{Frequency \& Distribution
Graphs}\label{frequency-distribution-graphs}

This section plots \textbf{frequency and distribution} of numerological
patterns (UDN/UMN/UYN, missing numbers).

\begin{figure}[H]

\caption{\label{fig-udn}Universal Day Number (UDN) Frequency}

\centering{

\includegraphics{COMPREHENSIVE_RESEARCH_THESIS_files/figure-pdf/fig-udn-output-1.png}

}

\end{figure}%

\begin{figure}[H]

\caption{\label{fig-life-path}Life Path Distribution (Birth Dataset)}

\centering{

\includegraphics{COMPREHENSIVE_RESEARCH_THESIS_files/figure-pdf/fig-life-path-output-1.png}

}

\end{figure}%

\begin{figure}[H]

\caption{\label{fig-name-expression}Name-Based Expression Number
Distribution (Pythagorean)}

\centering{

\includegraphics{COMPREHENSIVE_RESEARCH_THESIS_files/figure-pdf/fig-name-expression-output-1.png}

}

\end{figure}%

\begin{figure}[H]

\caption{\label{fig-missing}Missing Number Frequency (Lo Shu)}

\centering{

\includegraphics{COMPREHENSIVE_RESEARCH_THESIS_files/figure-pdf/fig-missing-output-1.png}

}

\end{figure}%

\chapter{Alignment \& Window Graphs}\label{alignment-window-graphs}

This section compares \textbf{numerology transits} to astrological
alignment windows and provides windowed overlap charts.

\begin{figure}[H]

\caption{\label{fig-udn-monthly}}

\centering{

\captionsetup{labelsep=none}

\subcaption{\label{fig-udn-monthly-1}Monthly UDN Distribution}

\centering{

\begin{verbatim}
<Figure size 2700x1200 with 0 Axes>
\end{verbatim}

}

\subcaption{\label{fig-udn-monthly-2}}

\centering{

\captionsetup{labelsep=none}\includegraphics{COMPREHENSIVE_RESEARCH_THESIS_files/figure-pdf/fig-udn-monthly-output-2.png}

}

}

\end{figure}%

\begin{figure}[H]

\caption{\label{fig-name-soul-urge}Name-Based Soul Urge Distribution
(Pythagorean vs Chaldean)}

\centering{

\includegraphics{COMPREHENSIVE_RESEARCH_THESIS_files/figure-pdf/fig-name-soul-urge-output-1.png}

}

\end{figure}%

\begin{figure}[H]

\caption{\label{fig-pinnacle}Pinnacle Distributions (Birth Dataset)}

\centering{

\includegraphics{COMPREHENSIVE_RESEARCH_THESIS_files/figure-pdf/fig-pinnacle-output-1.png}

}

\end{figure}%

\begin{figure}[H]

\caption{\label{fig-athlete-correlation}Name vs Birth Numerology
Correlation (Athletes Sample)}

\centering{

\includegraphics{COMPREHENSIVE_RESEARCH_THESIS_files/figure-pdf/fig-athlete-correlation-output-1.png}

}

\end{figure}%

\begin{figure}[H]

\caption{\label{fig-missing-heatmap}Missing Number Heatmap by Month}

\centering{

\includegraphics{COMPREHENSIVE_RESEARCH_THESIS_files/figure-pdf/fig-missing-heatmap-output-1.png}

}

\end{figure}%

\newpage

\chapter{Track 2: Earthquake Prediction
Analysis}\label{track-2-earthquake-prediction-analysis}

\chapter{Overview}\label{overview}

This report provides an \textbf{exhaustive Classical Jyotish Mundane}
coverage for earthquake prediction claims. The intent is to
\textbf{include every combination and trigger described in the classical
corpus} so the analysis cannot be dismissed for missing principles. The
report is organized as a modular catalogue, followed by algorithmic
definitions and a large-scale empirical evaluation against the
India-Nepal seismic dataset.

\section{Scope Commitments}\label{scope-commitments}

\begin{itemize}
\tightlist
\item
  \textbf{All Classical Sources}: Brihat Samhita, BPHS, Surya Siddhanta,
  Jataka Parijata, Saravali.
\item
  \textbf{All Combination Classes}: malefic aspects, conjunctions, Graha
  Yuddha, Shadbala components, combustion, retrograde, nakshatra lords,
  Panchanga elements (tithi/karana/yoga), houses and signs, eclipses and
  syzygy windows.
\item
  \textbf{All Time Windows}: same-day, ±1, ±3, ±7, ±14, ±30 days around
  seismic events.
\item
  \textbf{All Results Reported}: frequency, event-alignment rate, and
  statistical significance where applicable.
\end{itemize}

\section{Structure}\label{structure}

\begin{enumerate}
\def\labelenumi{\arabic{enumi}.}
\tightlist
\item
  \textbf{Sources}: Classical texts and canonical principles.
\item
  \textbf{Taxonomy}: A formal ontology of all trigger classes and
  variables.
\item
  \textbf{Combination Catalog}: Machine-generated, exhaustive list with
  stable IDs.
\item
  \textbf{Algorithms \& Examples}: Formal definitions and step-by-step
  calculations.
\item
  \textbf{Graphs \& Results}: Frequency and alignment diagnostics for
  every category.
\item
  \textbf{Appendix}: Full definitions and parameter defaults.
\end{enumerate}

\chapter{Classical Sources}\label{classical-sources}

This section enumerates the authoritative classical sources used as
\textbf{primary references} for mundane astrology principles included in
this report.

\section{Covered Sources (Primary
Canon)}\label{covered-sources-primary-canon}

\begin{itemize}
\tightlist
\item
  \textbf{Brihat Samhita} (\textbf{varahamihira\_brihat\_samhita?})
\item
  \textbf{Brihat Parashara Hora Shastra (BPHS)} (\textbf{bphs?})
\item
  \textbf{Surya Siddhanta} (\textbf{surya\_siddhanta?})
\item
  \textbf{Jataka Parijata} (\textbf{jataka\_parijata?})
\item
  \textbf{Saravali} (\textbf{saravali?})
\end{itemize}

\section{Coverage Policy}\label{coverage-policy}

All mundane principles that can be translated into \textbf{testable
astronomical combinations} are included. If a classical rule is symbolic
or qualitative, it is still recorded in the catalog with a note
describing the limitation.

\chapter{Vedic Principles Coverage (Master
Catalog)}\label{vedic-principles-coverage-master-catalog}

This report anchors its methodology in a shared \textbf{Vedic Principles
Catalog} derived from classical Jyotish sources. Every computable
principle is enumerated and mapped to a reproducible algorithm;
non-computable rules are still documented to maintain completeness and
avoid selective omission. (\textbf{bphs?};
\textbf{varahamihira\_brihat\_samhita?}; \textbf{saravali?};
\textbf{jataka\_parijata?}; \textbf{surya\_siddhanta?})

\begin{figure}[H]

\caption{Vedic Principles Catalog: Counts by Family}

{\centering \includegraphics{COMPREHENSIVE_RESEARCH_THESIS_files/figure-pdf/eq-vedic-catalog-counts-output-1.png}

}

\end{figure}%

\begin{longtable}[]{@{}lllllll@{}}

\caption{\label{tbl-eq-vedic-coverage}Coverage Matrix for Earthquake
Report}

\tabularnewline

\toprule\noalign{}
coverage\_status & family & scope & computed & documented &
not\_included & coverage\_ratio \\
\midrule\noalign{}
\endhead
\bottomrule\noalign{}
\endlastfoot
3 & bhava & core & 13 & 0 & 0 & 1.0 \\
4 & combustion & core & 1 & 0 & 0 & 1.0 \\
6 & dignity & core & 0 & 0 & 1 & 0.0 \\
7 & drishti & core & 1 & 0 & 0 & 1.0 \\
8 & eclipse & core & 1 & 0 & 0 & 1.0 \\
10 & graha & core & 10 & 0 & 0 & 1.0 \\
11 & graha\_yuddha & core & 1 & 0 & 0 & 1.0 \\
13 & karana & core & 12 & 0 & 0 & 1.0 \\
15 & nakshatra & core & 28 & 0 & 0 & 1.0 \\
16 & rashi & core & 13 & 0 & 0 & 1.0 \\
17 & retrograde & core & 1 & 0 & 0 & 1.0 \\
18 & shadbala & core & 1 & 0 & 0 & 1.0 \\
19 & syzygy & core & 1 & 0 & 0 & 1.0 \\
20 & tithi & core & 31 & 0 & 0 & 1.0 \\
23 & yoga & core & 28 & 0 & 0 & 1.0 \\
0 & arudha & extended & 0 & 3 & 0 & 0.0 \\
1 & ashtakavarga & extended & 0 & 3 & 0 & 0.0 \\
2 & avastha & extended & 0 & 7 & 0 & 0.0 \\
5 & dasha & extended & 6 & 0 & 0 & 1.0 \\
9 & gocara & extended & 0 & 1 & 0 & 0.0 \\
12 & kala & extended & 0 & 1 & 0 & 0.0 \\
14 & muhurta & extended & 0 & 1 & 0 & 0.0 \\
21 & upagraha & extended & 0 & 6 & 0 & 0.0 \\
22 & varga & extended & 0 & 17 & 0 & 0.0 \\
24 & yoga\_classical & extended & 0 & 1 & 0 & 0.0 \\

\end{longtable}

\begin{longtable}[]{@{}lllllll@{}}

\caption{\label{tbl-eq-vedic-master}Master Vedic Principles Catalog
(Core + Extended)}

\tabularnewline

\toprule\noalign{}
& family & item\_type & name & scope & computable & description \\
\midrule\noalign{}
\endhead
\bottomrule\noalign{}
\endlastfoot
0 & arudha & family & Arudha (Pada) & extended & False &
Reflection-based indicators of manifestation a... \\
1 & arudha & member & Arudha Lagna & extended & False & Arudha concept:
Arudha Lagna. \\
2 & arudha & member & Bhava Arudha & extended & False & Arudha concept:
Bhava Arudha. \\
3 & ashtakavarga & family & Ashtakavarga & extended & False &
Point-based evaluation of planetary contributi... \\
4 & ashtakavarga & member & Bhinnashtakavarga & extended & False &
Ashtakavarga method: Bhinnashtakavarga. \\
... & ... & ... & ... & ... & ... & ... \\
184 & yoga & member & Vishkumbha & core & True & Yoga: Vishkumbha. \\
185 & yoga & member & Vriddhi & core & True & Yoga: Vriddhi. \\
186 & yoga & member & Vyaghata & core & True & Yoga: Vyaghata. \\
187 & yoga & member & Vyatipata & core & True & Yoga: Vyatipata. \\
188 & yoga\_classical & family & Classical Yogas & extended & False &
Combinational yogas such as Raja, Dhana, and A... \\

\end{longtable}

\chapter{Vimshottari Dasha Overlay (India +
Nepal)}\label{vimshottari-dasha-overlay-india-nepal}

To address \textbf{phase‑2 country‑specific triggers}, we overlay the
\textbf{India Independence Chart} (1947‑08‑15 00:00 IST, Delhi) and a
\textbf{Nepal reference chart} (2015‑09‑20 00:00 NPT, Kathmandu) and
compute the active \textbf{Vimshottari Mahadasha} for each day in the
earthquake dataset. This allows the analysis to test whether regional
dasha cycles contribute to event clustering beyond the global planetary
configurations. (\textbf{bphs?})

\begin{figure}[H]

\caption{\label{fig-eq-dasha-days}}

\centering{

\captionsetup{labelsep=none}

\subcaption{\label{fig-eq-dasha-days-1}India vs Nepal Vimshottari
Mahadasha: Day Counts}

\centering{

\begin{verbatim}
<Figure size 2100x1200 with 0 Axes>
\end{verbatim}

}

\subcaption{\label{fig-eq-dasha-days-2}}

\centering{

\captionsetup{labelsep=none}\includegraphics{COMPREHENSIVE_RESEARCH_THESIS_files/figure-pdf/fig-eq-dasha-days-output-2.png}

}

}

\end{figure}%

\begin{figure}[H]

\caption{\label{fig-eq-dasha-timeline}Mahadasha Timeline: India vs Nepal
(2015-2024)}

\centering{

\includegraphics{COMPREHENSIVE_RESEARCH_THESIS_files/figure-pdf/fig-eq-dasha-timeline-output-1.png}

}

\end{figure}%

\begin{figure}[H]

\caption{\label{fig-eq-dasha-heatmap}Mahadasha Co-Occurrence: India vs
Nepal}

\centering{

\includegraphics{COMPREHENSIVE_RESEARCH_THESIS_files/figure-pdf/fig-eq-dasha-heatmap-output-1.png}

}

\end{figure}%

\begin{figure}[H]

\caption{\label{fig-eq-antardasha-heatmap}Antardasha Co-Occurrence:
India vs Nepal}

\centering{

\includegraphics{COMPREHENSIVE_RESEARCH_THESIS_files/figure-pdf/fig-eq-antardasha-heatmap-output-1.png}

}

\end{figure}%

\chapter{Master Combination Catalog}\label{master-combination-catalog}

This section presents the \textbf{exhaustive combination catalog},
generated programmatically from the taxonomy. Each rule has a stable ID
for reproducibility and downstream graphing.

\phantomsection\label{combo-catalog}
\begin{longtable}[]{@{}llllllllllllllllllllll@{}}
\toprule\noalign{}
& combo\_id & category & rule & parameters & window\_days & sources & p1
& p2 & aspect & aspect\_deg & ... & tithi & karana & yoga &
eclipse\_type & phase & lord & sub\_lord & india\_lord & nepal\_lord &
compound \\
\midrule\noalign{}
\endhead
\bottomrule\noalign{}
\endlastfoot
0 & aspects:000001 & aspects & Aspect conjunction between Sun and Moon &
angle=0±1° & NaN & Brihat Samhita; BPHS; Surya Siddhanta; Jataka ... &
Sun & Moon & conjunction & 0.0 & ... & NaN & NaN & NaN & NaN & NaN & NaN
& NaN & NaN & NaN & NaN \\
1 & aspects:000002 & aspects & Aspect conjunction between Sun and Moon &
angle=0±3° & NaN & Brihat Samhita; BPHS; Surya Siddhanta; Jataka ... &
Sun & Moon & conjunction & 0.0 & ... & NaN & NaN & NaN & NaN & NaN & NaN
& NaN & NaN & NaN & NaN \\
2 & aspects:000003 & aspects & Aspect conjunction between Sun and Moon &
angle=0±5° & NaN & Brihat Samhita; BPHS; Surya Siddhanta; Jataka ... &
Sun & Moon & conjunction & 0.0 & ... & NaN & NaN & NaN & NaN & NaN & NaN
& NaN & NaN & NaN & NaN \\
3 & aspects:000004 & aspects & Aspect conjunction between Sun and Moon &
angle=0±8° & NaN & Brihat Samhita; BPHS; Surya Siddhanta; Jataka ... &
Sun & Moon & conjunction & 0.0 & ... & NaN & NaN & NaN & NaN & NaN & NaN
& NaN & NaN & NaN & NaN \\
4 & aspects:000005 & aspects & Aspect sextile between Sun and Moon &
angle=60±1° & NaN & Brihat Samhita; BPHS; Surya Siddhanta; Jataka ... &
Sun & Moon & sextile & 60.0 & ... & NaN & NaN & NaN & NaN & NaN & NaN &
NaN & NaN & NaN & NaN \\
5 & aspects:000006 & aspects & Aspect sextile between Sun and Moon &
angle=60±3° & NaN & Brihat Samhita; BPHS; Surya Siddhanta; Jataka ... &
Sun & Moon & sextile & 60.0 & ... & NaN & NaN & NaN & NaN & NaN & NaN &
NaN & NaN & NaN & NaN \\
6 & aspects:000007 & aspects & Aspect sextile between Sun and Moon &
angle=60±5° & NaN & Brihat Samhita; BPHS; Surya Siddhanta; Jataka ... &
Sun & Moon & sextile & 60.0 & ... & NaN & NaN & NaN & NaN & NaN & NaN &
NaN & NaN & NaN & NaN \\
7 & aspects:000008 & aspects & Aspect sextile between Sun and Moon &
angle=60±8° & NaN & Brihat Samhita; BPHS; Surya Siddhanta; Jataka ... &
Sun & Moon & sextile & 60.0 & ... & NaN & NaN & NaN & NaN & NaN & NaN &
NaN & NaN & NaN & NaN \\
8 & aspects:000009 & aspects & Aspect square between Sun and Moon &
angle=90±1° & NaN & Brihat Samhita; BPHS; Surya Siddhanta; Jataka ... &
Sun & Moon & square & 90.0 & ... & NaN & NaN & NaN & NaN & NaN & NaN &
NaN & NaN & NaN & NaN \\
9 & aspects:000010 & aspects & Aspect square between Sun and Moon &
angle=90±3° & NaN & Brihat Samhita; BPHS; Surya Siddhanta; Jataka ... &
Sun & Moon & square & 90.0 & ... & NaN & NaN & NaN & NaN & NaN & NaN &
NaN & NaN & NaN & NaN \\
10 & aspects:000011 & aspects & Aspect square between Sun and Moon &
angle=90±5° & NaN & Brihat Samhita; BPHS; Surya Siddhanta; Jataka ... &
Sun & Moon & square & 90.0 & ... & NaN & NaN & NaN & NaN & NaN & NaN &
NaN & NaN & NaN & NaN \\
11 & aspects:000012 & aspects & Aspect square between Sun and Moon &
angle=90±8° & NaN & Brihat Samhita; BPHS; Surya Siddhanta; Jataka ... &
Sun & Moon & square & 90.0 & ... & NaN & NaN & NaN & NaN & NaN & NaN &
NaN & NaN & NaN & NaN \\
12 & aspects:000013 & aspects & Aspect trine between Sun and Moon &
angle=120±1° & NaN & Brihat Samhita; BPHS; Surya Siddhanta; Jataka ... &
Sun & Moon & trine & 120.0 & ... & NaN & NaN & NaN & NaN & NaN & NaN &
NaN & NaN & NaN & NaN \\
13 & aspects:000014 & aspects & Aspect trine between Sun and Moon &
angle=120±3° & NaN & Brihat Samhita; BPHS; Surya Siddhanta; Jataka ... &
Sun & Moon & trine & 120.0 & ... & NaN & NaN & NaN & NaN & NaN & NaN &
NaN & NaN & NaN & NaN \\
14 & aspects:000015 & aspects & Aspect trine between Sun and Moon &
angle=120±5° & NaN & Brihat Samhita; BPHS; Surya Siddhanta; Jataka ... &
Sun & Moon & trine & 120.0 & ... & NaN & NaN & NaN & NaN & NaN & NaN &
NaN & NaN & NaN & NaN \\
15 & aspects:000016 & aspects & Aspect trine between Sun and Moon &
angle=120±8° & NaN & Brihat Samhita; BPHS; Surya Siddhanta; Jataka ... &
Sun & Moon & trine & 120.0 & ... & NaN & NaN & NaN & NaN & NaN & NaN &
NaN & NaN & NaN & NaN \\
16 & aspects:000017 & aspects & Aspect opposition between Sun and Moon &
angle=180±1° & NaN & Brihat Samhita; BPHS; Surya Siddhanta; Jataka ... &
Sun & Moon & opposition & 180.0 & ... & NaN & NaN & NaN & NaN & NaN &
NaN & NaN & NaN & NaN & NaN \\
17 & aspects:000018 & aspects & Aspect opposition between Sun and Moon &
angle=180±3° & NaN & Brihat Samhita; BPHS; Surya Siddhanta; Jataka ... &
Sun & Moon & opposition & 180.0 & ... & NaN & NaN & NaN & NaN & NaN &
NaN & NaN & NaN & NaN & NaN \\
18 & aspects:000019 & aspects & Aspect opposition between Sun and Moon &
angle=180±5° & NaN & Brihat Samhita; BPHS; Surya Siddhanta; Jataka ... &
Sun & Moon & opposition & 180.0 & ... & NaN & NaN & NaN & NaN & NaN &
NaN & NaN & NaN & NaN & NaN \\
19 & aspects:000020 & aspects & Aspect opposition between Sun and Moon &
angle=180±8° & NaN & Brihat Samhita; BPHS; Surya Siddhanta; Jataka ... &
Sun & Moon & opposition & 180.0 & ... & NaN & NaN & NaN & NaN & NaN &
NaN & NaN & NaN & NaN & NaN \\
\end{longtable}

\phantomsection\label{combo-counts}
\begin{longtable}[]{@{}lll@{}}
\caption{Combination Counts by Category}\tabularnewline
\toprule\noalign{}
& category & count \\
\midrule\noalign{}
\endfirsthead
\toprule\noalign{}
& category & count \\
\midrule\noalign{}
\endhead
\bottomrule\noalign{}
\endlastfoot
0 & aspects & 720 \\
11 & malefic\_aspects & 120 \\
17 & sign & 108 \\
5 & house & 108 \\
9 & india\_nepal\_dasha\_pair & 81 \\
8 & india\_nepal\_antardasha\_pair & 81 \\
4 & graha\_yuddha & 36 \\
19 & tithi & 30 \\
16 & shadbala & 27 \\
20 & yoga & 27 \\
10 & karana & 11 \\
1 & combustion & 9 \\
7 & india\_dasha & 9 \\
12 & nakshatra & 9 \\
13 & nepal\_antardasha & 9 \\
14 & nepal\_dasha & 9 \\
15 & retrograde & 9 \\
6 & india\_antardasha & 9 \\
2 & compound & 6 \\
3 & eclipse & 2 \\
18 & syzygy & 2 \\
\end{longtable}

The full catalog is stored at
\texttt{docs/research/track\_2\_earthquake\_prediction/data/combination\_catalog.csv}.

\chapter{Frequency Graphs (All
Combinations)}\label{frequency-graphs-all-combinations}

This section plots \textbf{activation frequencies} for every combination
in the catalog.

\begin{figure}[H]

\caption{\label{fig-category-activation}Activation Days by Combination
Category}

\centering{

\includegraphics{COMPREHENSIVE_RESEARCH_THESIS_files/figure-pdf/fig-category-activation-output-1.png}

}

\end{figure}%

\begin{figure}[H]

\caption{\label{fig-top-activation}Top 25 Most Frequent Combinations}

\centering{

\includegraphics{COMPREHENSIVE_RESEARCH_THESIS_files/figure-pdf/fig-top-activation-output-1.png}

}

\end{figure}%

\chapter{Alignment Windows vs
Earthquakes}\label{alignment-windows-vs-earthquakes}

This section compares \textbf{combination activation} against earthquake
windows (same day and ±N days).

\begin{figure}[H]

\caption{\label{fig-window-overlap}Average Overlap Rate by Category (±3
days)}

\centering{

\includegraphics{COMPREHENSIVE_RESEARCH_THESIS_files/figure-pdf/fig-window-overlap-output-1.png}

}

\end{figure}%

\begin{figure}[H]

\caption{\label{fig-earthquake-zscores}Z-score Distribution vs Random
Baseline (±3 days)}

\centering{

\includegraphics{COMPREHENSIVE_RESEARCH_THESIS_files/figure-pdf/fig-earthquake-zscores-output-1.png}

}

\end{figure}%

\begin{figure}[H]

\caption{\label{fig-window-compare}Overlap Rates by Window Size (All
Combinations)}

\centering{

\includegraphics{COMPREHENSIVE_RESEARCH_THESIS_files/figure-pdf/fig-window-compare-output-1.png}

}

\end{figure}%

\begin{figure}[H]

\caption{\label{fig-top-overlap}Top 25 Combinations by ±3 Day Overlap}

\centering{

\includegraphics{COMPREHENSIVE_RESEARCH_THESIS_files/figure-pdf/fig-top-overlap-output-1.png}

}

\end{figure}%

\newpage

\chapter{Track 3: Gold Market
Correlation}\label{track-3-gold-market-correlation}

\chapter{Overview}\label{overview-1}

This section summarizes scope, objectives, and the full
astrology-principles coverage (Vedic + Western).

\chapter{Astrology Principles Catalog (Vedic +
Western)}\label{astrology-principles-catalog-vedic-western}

This section provides an \textbf{exhaustive catalog} of astrology‑based
indicators used for gold price prediction claims (Vedic + Western).

\phantomsection\label{gold-astro-catalog}
\begin{longtable}[]{@{}llllllllllllllll@{}}
\toprule\noalign{}
& combo\_id & category & rule & parameters & sources & p1 & p2 & aspect
& aspect\_deg & orb & sign & phase & eclipse\_type & lord &
window\_days \\
\midrule\noalign{}
\endhead
\bottomrule\noalign{}
\endlastfoot
0 & major\_aspect:000001 & major\_aspect & Sun-Moon conjunction &
angle=0±1° & Vedic + Western (comprehensive) & Sun & Moon & conjunction
& 0.0 & 1.0 & NaN & NaN & NaN & NaN & NaN \\
1 & major\_aspect:000002 & major\_aspect & Sun-Moon conjunction &
angle=0±3° & Vedic + Western (comprehensive) & Sun & Moon & conjunction
& 0.0 & 3.0 & NaN & NaN & NaN & NaN & NaN \\
2 & major\_aspect:000003 & major\_aspect & Sun-Moon conjunction &
angle=0±5° & Vedic + Western (comprehensive) & Sun & Moon & conjunction
& 0.0 & 5.0 & NaN & NaN & NaN & NaN & NaN \\
3 & major\_aspect:000004 & major\_aspect & Sun-Moon conjunction &
angle=0±8° & Vedic + Western (comprehensive) & Sun & Moon & conjunction
& 0.0 & 8.0 & NaN & NaN & NaN & NaN & NaN \\
4 & major\_aspect:000005 & major\_aspect & Sun-Moon sextile &
angle=60±1° & Vedic + Western (comprehensive) & Sun & Moon & sextile &
60.0 & 1.0 & NaN & NaN & NaN & NaN & NaN \\
5 & major\_aspect:000006 & major\_aspect & Sun-Moon sextile &
angle=60±3° & Vedic + Western (comprehensive) & Sun & Moon & sextile &
60.0 & 3.0 & NaN & NaN & NaN & NaN & NaN \\
6 & major\_aspect:000007 & major\_aspect & Sun-Moon sextile &
angle=60±5° & Vedic + Western (comprehensive) & Sun & Moon & sextile &
60.0 & 5.0 & NaN & NaN & NaN & NaN & NaN \\
7 & major\_aspect:000008 & major\_aspect & Sun-Moon sextile &
angle=60±8° & Vedic + Western (comprehensive) & Sun & Moon & sextile &
60.0 & 8.0 & NaN & NaN & NaN & NaN & NaN \\
8 & major\_aspect:000009 & major\_aspect & Sun-Moon square & angle=90±1°
& Vedic + Western (comprehensive) & Sun & Moon & square & 90.0 & 1.0 &
NaN & NaN & NaN & NaN & NaN \\
9 & major\_aspect:000010 & major\_aspect & Sun-Moon square & angle=90±3°
& Vedic + Western (comprehensive) & Sun & Moon & square & 90.0 & 3.0 &
NaN & NaN & NaN & NaN & NaN \\
10 & major\_aspect:000011 & major\_aspect & Sun-Moon square &
angle=90±5° & Vedic + Western (comprehensive) & Sun & Moon & square &
90.0 & 5.0 & NaN & NaN & NaN & NaN & NaN \\
11 & major\_aspect:000012 & major\_aspect & Sun-Moon square &
angle=90±8° & Vedic + Western (comprehensive) & Sun & Moon & square &
90.0 & 8.0 & NaN & NaN & NaN & NaN & NaN \\
12 & major\_aspect:000013 & major\_aspect & Sun-Moon trine &
angle=120±1° & Vedic + Western (comprehensive) & Sun & Moon & trine &
120.0 & 1.0 & NaN & NaN & NaN & NaN & NaN \\
13 & major\_aspect:000014 & major\_aspect & Sun-Moon trine &
angle=120±3° & Vedic + Western (comprehensive) & Sun & Moon & trine &
120.0 & 3.0 & NaN & NaN & NaN & NaN & NaN \\
14 & major\_aspect:000015 & major\_aspect & Sun-Moon trine &
angle=120±5° & Vedic + Western (comprehensive) & Sun & Moon & trine &
120.0 & 5.0 & NaN & NaN & NaN & NaN & NaN \\
15 & major\_aspect:000016 & major\_aspect & Sun-Moon trine &
angle=120±8° & Vedic + Western (comprehensive) & Sun & Moon & trine &
120.0 & 8.0 & NaN & NaN & NaN & NaN & NaN \\
16 & major\_aspect:000017 & major\_aspect & Sun-Moon opposition &
angle=180±1° & Vedic + Western (comprehensive) & Sun & Moon & opposition
& 180.0 & 1.0 & NaN & NaN & NaN & NaN & NaN \\
17 & major\_aspect:000018 & major\_aspect & Sun-Moon opposition &
angle=180±3° & Vedic + Western (comprehensive) & Sun & Moon & opposition
& 180.0 & 3.0 & NaN & NaN & NaN & NaN & NaN \\
18 & major\_aspect:000019 & major\_aspect & Sun-Moon opposition &
angle=180±5° & Vedic + Western (comprehensive) & Sun & Moon & opposition
& 180.0 & 5.0 & NaN & NaN & NaN & NaN & NaN \\
19 & major\_aspect:000020 & major\_aspect & Sun-Moon opposition &
angle=180±8° & Vedic + Western (comprehensive) & Sun & Moon & opposition
& 180.0 & 8.0 & NaN & NaN & NaN & NaN & NaN \\
20 & major\_aspect:000021 & major\_aspect & Sun-Mars conjunction &
angle=0±1° & Vedic + Western (comprehensive) & Sun & Mars & conjunction
& 0.0 & 1.0 & NaN & NaN & NaN & NaN & NaN \\
21 & major\_aspect:000022 & major\_aspect & Sun-Mars conjunction &
angle=0±3° & Vedic + Western (comprehensive) & Sun & Mars & conjunction
& 0.0 & 3.0 & NaN & NaN & NaN & NaN & NaN \\
22 & major\_aspect:000023 & major\_aspect & Sun-Mars conjunction &
angle=0±5° & Vedic + Western (comprehensive) & Sun & Mars & conjunction
& 0.0 & 5.0 & NaN & NaN & NaN & NaN & NaN \\
23 & major\_aspect:000024 & major\_aspect & Sun-Mars conjunction &
angle=0±8° & Vedic + Western (comprehensive) & Sun & Mars & conjunction
& 0.0 & 8.0 & NaN & NaN & NaN & NaN & NaN \\
24 & major\_aspect:000025 & major\_aspect & Sun-Mars sextile &
angle=60±1° & Vedic + Western (comprehensive) & Sun & Mars & sextile &
60.0 & 1.0 & NaN & NaN & NaN & NaN & NaN \\
\end{longtable}

\phantomsection\label{gold-astro-catalog-counts}
\begin{longtable}[]{@{}lll@{}}
\caption{Astrology Principle Counts by Category}\tabularnewline
\toprule\noalign{}
& category & count \\
\midrule\noalign{}
\endfirsthead
\toprule\noalign{}
& category & count \\
\midrule\noalign{}
\endhead
\bottomrule\noalign{}
\endlastfoot
6 & major\_aspect & 1320 \\
7 & minor\_aspect & 792 \\
4 & ingress & 144 \\
10 & vedic\_malefic\_aspect & 120 \\
9 & retrograde & 12 \\
2 & dasha & 9 \\
8 & nakshatra\_lord & 9 \\
0 & combustion & 8 \\
1 & compound & 6 \\
5 & lunar\_phase & 4 \\
3 & eclipse & 2 \\
\end{longtable}

\chapter{Vedic Principles Coverage (Master
Catalog)}\label{vedic-principles-coverage-master-catalog-1}

To ensure scientific completeness, this report references a shared
\textbf{Vedic Principles Catalog} derived from classical Jyotish texts.
Computable indicators are evaluated quantitatively; non-computable
principles are documented for transparency and future extensibility.
(\textbf{bphs?}; \textbf{varahamihira\_brihat\_samhita?};
\textbf{saravali?}; \textbf{jataka\_parijata?};
\textbf{surya\_siddhanta?})

\begin{figure}[H]

\caption{Vedic Principles Catalog: Counts by Family}

{\centering \includegraphics{COMPREHENSIVE_RESEARCH_THESIS_files/figure-pdf/gold-vedic-catalog-counts-output-1.png}

}

\end{figure}%

\begin{longtable}[]{@{}lllllll@{}}

\caption{\label{tbl-gold-vedic-coverage}Coverage Matrix for Gold Report}

\tabularnewline

\toprule\noalign{}
coverage\_status & family & scope & computed & documented &
not\_included & coverage\_ratio \\
\midrule\noalign{}
\endhead
\bottomrule\noalign{}
\endlastfoot
3 & bhava & core & 0 & 0 & 13 & 0.0 \\
4 & combustion & core & 1 & 0 & 0 & 1.0 \\
6 & dignity & core & 0 & 0 & 1 & 0.0 \\
7 & drishti & core & 1 & 0 & 0 & 1.0 \\
8 & eclipse & core & 1 & 0 & 0 & 1.0 \\
10 & graha & core & 10 & 0 & 0 & 1.0 \\
11 & graha\_yuddha & core & 0 & 0 & 1 & 0.0 \\
13 & karana & core & 0 & 0 & 12 & 0.0 \\
15 & nakshatra & core & 28 & 0 & 0 & 1.0 \\
16 & rashi & core & 13 & 0 & 0 & 1.0 \\
17 & retrograde & core & 1 & 0 & 0 & 1.0 \\
18 & shadbala & core & 0 & 0 & 1 & 0.0 \\
19 & syzygy & core & 1 & 0 & 0 & 1.0 \\
20 & tithi & core & 0 & 0 & 31 & 0.0 \\
23 & yoga & core & 0 & 0 & 28 & 0.0 \\
0 & arudha & extended & 0 & 3 & 0 & 0.0 \\
1 & ashtakavarga & extended & 0 & 3 & 0 & 0.0 \\
2 & avastha & extended & 0 & 7 & 0 & 0.0 \\
5 & dasha & extended & 6 & 0 & 0 & 1.0 \\
9 & gocara & extended & 0 & 1 & 0 & 0.0 \\
12 & kala & extended & 0 & 1 & 0 & 0.0 \\
14 & muhurta & extended & 0 & 1 & 0 & 0.0 \\
21 & upagraha & extended & 0 & 6 & 0 & 0.0 \\
22 & varga & extended & 0 & 17 & 0 & 0.0 \\
24 & yoga\_classical & extended & 0 & 1 & 0 & 0.0 \\

\end{longtable}

\begin{longtable}[]{@{}lllllll@{}}

\caption{\label{tbl-gold-vedic-master}Master Vedic Principles Catalog
(Core + Extended)}

\tabularnewline

\toprule\noalign{}
& family & item\_type & name & scope & computable & description \\
\midrule\noalign{}
\endhead
\bottomrule\noalign{}
\endlastfoot
0 & arudha & family & Arudha (Pada) & extended & False &
Reflection-based indicators of manifestation a... \\
1 & arudha & member & Arudha Lagna & extended & False & Arudha concept:
Arudha Lagna. \\
2 & arudha & member & Bhava Arudha & extended & False & Arudha concept:
Bhava Arudha. \\
3 & ashtakavarga & family & Ashtakavarga & extended & False &
Point-based evaluation of planetary contributi... \\
4 & ashtakavarga & member & Bhinnashtakavarga & extended & False &
Ashtakavarga method: Bhinnashtakavarga. \\
... & ... & ... & ... & ... & ... & ... \\
184 & yoga & member & Vishkumbha & core & True & Yoga: Vishkumbha. \\
185 & yoga & member & Vriddhi & core & True & Yoga: Vriddhi. \\
186 & yoga & member & Vyaghata & core & True & Yoga: Vyaghata. \\
187 & yoga & member & Vyatipata & core & True & Yoga: Vyatipata. \\
188 & yoga\_classical & family & Classical Yogas & extended & False &
Combinational yogas such as Raja, Dhana, and A... \\

\end{longtable}

\chapter{Vimshottari Dasha Overlay (Global
Baseline)}\label{vimshottari-dasha-overlay-global-baseline}

For the phase‑1 global analysis, we compute the \textbf{Vimshottari
Mahadasha} using a global baseline chart (2000‑01‑01 00:00 UTC at 0°N,
0°E). This provides a long‑cycle macro timing feature that can be
compared against gold return spikes without country‑specific
assumptions.

\begin{figure}[H]

\caption{\label{fig-gold-dasha-days}Global Vimshottari Mahadasha: Day
Counts in Dataset}

\centering{

\includegraphics{COMPREHENSIVE_RESEARCH_THESIS_files/figure-pdf/fig-gold-dasha-days-output-1.png}

}

\end{figure}%

\chapter{Frequency Graphs}\label{frequency-graphs}

Activation frequency for each astrology principle category across gold
trading days.

\begin{figure}[H]

\caption{\label{fig-gold-astro-category}Total Activation Days by
Astrology Category}

\centering{

\includegraphics{COMPREHENSIVE_RESEARCH_THESIS_files/figure-pdf/fig-gold-astro-category-output-1.png}

}

\end{figure}%

\chapter{Alignment Graphs}\label{alignment-graphs}

Overlap between indicator activation and gold return spikes (top 5\%
absolute daily returns).

\begin{figure}[H]

\caption{\label{fig-gold-overlap}Mean Overlap Rate by Category (±3
days)}

\centering{

\includegraphics{COMPREHENSIVE_RESEARCH_THESIS_files/figure-pdf/fig-gold-overlap-output-1.png}

}

\end{figure}%

\begin{figure}[H]

\caption{\label{fig-gold-zscore-hist}Distribution of Z-Scores for
Overlap vs Random Baseline (±3 days)}

\centering{

\includegraphics{COMPREHENSIVE_RESEARCH_THESIS_files/figure-pdf/fig-gold-zscore-hist-output-1.png}

}

\end{figure}%

\newpage

\chapter{Consolidated Discussion and
Synthesis}\label{consolidated-discussion-and-synthesis}

The combined weight of these investigations establishes a clear boundary
between the symbolic utility of traditional systems and their predictive
accuracy in physical domains.

\newpage

\chapter{Appendix: Master Algorithm
Catalog}\label{appendix-master-algorithm-catalog}

\chapter{Algorithms \& Worked
Examples}\label{algorithms-worked-examples}

This section formalizes each numerology principle as a reproducible
algorithm.

\section{Digital Root (Core
Reduction)}\label{digital-root-core-reduction}

\[
DR(n) = (n - 1) \bmod 9 + 1
\]

\section{Universal Day Number (UDN)}\label{universal-day-number-udn}

\[
UDN = DR(day + month + sum(year\_digits))
\]

\section{Universal Month Number (UMN)}\label{universal-month-number-umn}

\[
UMN = DR(month + sum(year\_digits))
\]

\section{Universal Year Number (UYN)}\label{universal-year-number-uyn}

\[
UYN = DR(sum(year\_digits))
\]

\section{Missing Number Analysis (Lo
Shu)}\label{missing-number-analysis-lo-shu}

Count digits 1--9 in the date string (YYYYMMDD). A number is
\textbf{missing} if count = 0.

\section{Pythagorean Name Numerology (Expression, Soul Urge,
Personality)}\label{pythagorean-name-numerology-expression-soul-urge-personality}

Map letters to 1--9 using the Pythagorean table and reduce:

\begin{verbatim}
1: A J S
2: B K T
3: C L U
4: D M V
5: E N W
6: F O X
7: G P Y
8: H Q Z
9: I R
\end{verbatim}

\begin{itemize}
\tightlist
\item
  \textbf{Expression}: sum of all letter values
\item
  \textbf{Soul Urge}: sum of vowels only
\item
  \textbf{Personality}: sum of consonants only
\end{itemize}

\section{Chaldean Name Numerology}\label{chaldean-name-numerology}

Map letters to 1--8 using the Chaldean table and reduce:

\begin{verbatim}
1: A I J Q Y
2: B K R
3: C G L S
4: D M T
5: E H N X
6: U V W
7: O Z
8: F P
\end{verbatim}

\section{Master Numbers}\label{master-numbers}

Master numbers (11, 22, 33) are optionally preserved instead of
reduction.

\section{Karmic Debt Numbers}\label{karmic-debt-numbers}

If compound numbers 13, 14, 16, 19 appear, they are flagged as special
karmic patterns.

\section{Life Path Number (Birth
Date)}\label{life-path-number-birth-date}

Sum all digits of the birth date (YYYYMMDD) and reduce.

\section{Pinnacles \& Challenges}\label{pinnacles-challenges}

Let: - \(M = DR(month)\), \(D = DR(day)\), \(Y = DR(year)\)

Pinnacles: - \(P1 = DR(M + D)\) - \(P2 = DR(D + Y)\) -
\(P3 = DR(P1 + P2)\) - \(P4 = DR(M + Y)\)

Challenges: - \(C1 = |M - D|\) - \(C2 = |D - Y|\) - \(C3 = |C1 - C2|\) -
\(C4 = |M - Y|\)

\chapter{Algorithms and Worked
Examples}\label{algorithms-and-worked-examples}

This section formalizes each rule class into a concrete algorithmic
procedure.

\section{Algorithm A: Daily Astro Feature
Generation}\label{algorithm-a-daily-astro-feature-generation}

\begin{enumerate}
\def\labelenumi{\arabic{enumi}.}
\tightlist
\item
  Compute planetary longitudes at local noon for each day.
\item
  Derive retrograde and combustion status.
\item
  Compute Panchanga elements (tithi, karana, yoga).
\item
  Compute houses using ascendant at the same timestamp.
\item
  Store a daily feature table for downstream combination evaluation.
\end{enumerate}

Implementation lives in
\texttt{docs/research/track\_2\_earthquake\_prediction/scripts/generate\_daily\_astro\_features.py}.

\section{Algorithm B: Combination
Evaluation}\label{algorithm-b-combination-evaluation}

\begin{enumerate}
\def\labelenumi{\arabic{enumi}.}
\tightlist
\item
  Load the daily feature table.
\item
  For each combination in the catalog, evaluate its rule across all
  days.
\item
  Compute activation frequency and overlap with earthquake event
  windows.
\item
  Store results for graphing and significance testing.
\end{enumerate}

\section{Example: Aspect Rule (Mars--Saturn
square)}\label{example-aspect-rule-marssaturn-square}

Let \(\Delta = |\lambda_{Mars} - \lambda_{Saturn}| \bmod 360\).

Rule triggers if:

\[
|\Delta - 90^\circ| \leq 5^\circ
\]

\section{Example: Graha Yuddha}\label{example-graha-yuddha}

For any planet pair \((i,j)\):

\[
|\lambda_i - \lambda_j| < 1^\circ
\]

\section{Example: Combustion}\label{example-combustion}

For planet \(p\):

\[
|\lambda_p - \lambda_{Sun}| < 8^\circ
\]

\section{Example: Tithi}\label{example-tithi}

\[
Tithi = \left\lfloor \frac{(\lambda_{Moon}-\lambda_{Sun}) \bmod 360}{12} \right\rfloor + 1
\]

\chapter{Algorithms \& Worked
Examples}\label{algorithms-worked-examples-1}

This section formalizes the Vedic + Western astrology indicators used in
the gold report.

\section{Algorithm A: Trading-Day Ephemeris
Alignment}\label{algorithm-a-trading-day-ephemeris-alignment}

\begin{enumerate}
\def\labelenumi{\arabic{enumi}.}
\tightlist
\item
  Load gold trading dates from the local cache.
\item
  Compute planetary longitudes at \textbf{12:00 UTC} for each trading
  day.
\item
  Derive retrograde, combustion, sign ingress, and lunar phase
  indicators.
\end{enumerate}

Implementation:
\texttt{docs/research/track\_3\_gold\_market/scripts/generate\_gold\_astro\_features.py}

\section{Algorithm B: Aspect Trigger}\label{algorithm-b-aspect-trigger}

For planets \(i,j\) and target angle \(\theta\):

\[
|(\lambda_i - \lambda_j) \bmod 360 - \theta| \leq \epsilon
\]

We evaluate \textbf{all major + minor aspects} across all planet pairs.

\section{Algorithm C: Event Overlap}\label{algorithm-c-event-overlap}

Define \textbf{event days} as top 5\% absolute log-returns. Compute
overlap between indicator activation and event windows (±1, ±3, ±7, ±14,
±30 days).

Implementation:
\texttt{docs/research/track\_3\_gold\_market/scripts/compute\_gold\_astro\_metrics.py}

\phantomsection\label{refs}
\begin{CSLReferences}{1}{0}
\bibitem[\citeproctext]{ref-benigeri1992}
Benigeri, M., \& Pluye, P. (1992). Numerology: A study of numerical
associations in everyday life. \emph{Social Science Information},
\emph{31}(4), 583--602. \url{https://doi.org/10.1177/053901892031004004}

\bibitem[\citeproctext]{ref-berger2006}
Berger, A. A. (2006). Numerology and the art of predicting the future.
\emph{Semiotica}, \emph{2006}(162), 1--16.

\bibitem[\citeproctext]{ref-bishop2006}
Bishop, C. M. (2006). \emph{Pattern recognition and machine learning}.
Springer.

\bibitem[\citeproctext]{ref-damisch2010}
Damisch, L., Stoberock, B., \& Mussweiler, T. (2010). Keep your fingers
crossed! How superstition improves performance. \emph{Psychological
Science}, \emph{21}(7), 1014--1020.
\url{https://doi.org/10.1177/0956797610372631}

\bibitem[\citeproctext]{ref-hacking1990}
Hacking, I. (1990). \emph{The taming of chance}.

\bibitem[\citeproctext]{ref-hunter2007}
Hunter, J. D. (2007). Matplotlib: A 2D graphics environment.
\emph{Computing in Science \& Engineering}, \emph{9}(3), 90--95.

\bibitem[\citeproctext]{ref-knuth84}
Knuth, D. E. (1984). Literate programming. \emph{Comput. J.},
\emph{27}(2), 97--111. \url{https://doi.org/10.1093/comjnl/27.2.97}

\bibitem[\citeproctext]{ref-mcclelland2009}
McClelland, B. A. (2009). The place of the occult in popular culture.
\emph{Journal of Popular Culture}, \emph{42}(6), 1059--1074.

\bibitem[\citeproctext]{ref-mckinney2010}
McKinney, W. et al. (2010). Data structures for statistical computing in
python. \emph{Proceedings of the 9th Python in Science Conference},
\emph{445}(1), 51--56.

\bibitem[\citeproctext]{ref-pedregosa2011}
Pedregosa, F., Varoquaux, G., Gramfort, A., Michel, V., Thirion, B.,
Grisel, O., Blondel, M., Prettenhofer, P., Weiss, R., Dubourg, V., et
al. (2011). Scikit-learn: Machine learning in python. In \emph{Journal
of Machine Learning Research} (Vol. 12, pp. 2825--2830).

\bibitem[\citeproctext]{ref-schimmel1975}
Schimmel, A. (1975). \emph{The mystery of numbers}. Oxford University
Press.

\bibitem[\citeproctext]{ref-stigler1986}
Stigler, S. M. (1986). \emph{The history of statistics: The measurement
of uncertainty before 1900}. Harvard University Press.

\end{CSLReferences}



\end{document}
