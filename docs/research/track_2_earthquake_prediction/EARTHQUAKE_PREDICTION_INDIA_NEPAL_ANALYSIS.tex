% Options for packages loaded elsewhere
% Options for packages loaded elsewhere
\PassOptionsToPackage{unicode}{hyperref}
\PassOptionsToPackage{hyphens}{url}
\PassOptionsToPackage{dvipsnames,svgnames,x11names}{xcolor}
%
\documentclass[
  american,
  11pt,
  a4paper,
  oneside,
  openany]{article}
\usepackage{xcolor}
\usepackage[top=30mm,left=25mm,right=25mm,bottom=30mm,heightrounded,top=25mm,bottom=25mm]{geometry}
\usepackage{amsmath,amssymb}
\setcounter{secnumdepth}{5}
\usepackage{iftex}
\ifPDFTeX
  \usepackage[T1]{fontenc}
  \usepackage[utf8]{inputenc}
  \usepackage{textcomp} % provide euro and other symbols
\else % if luatex or xetex
  \usepackage{unicode-math} % this also loads fontspec
  \defaultfontfeatures{Scale=MatchLowercase}
  \defaultfontfeatures[\rmfamily]{Ligatures=TeX,Scale=1}
\fi
\usepackage[]{libertine}
\ifPDFTeX\else
  % xetex/luatex font selection
\fi
% Use upquote if available, for straight quotes in verbatim environments
\IfFileExists{upquote.sty}{\usepackage{upquote}}{}
\IfFileExists{microtype.sty}{% use microtype if available
  \usepackage[]{microtype}
  \UseMicrotypeSet[protrusion]{basicmath} % disable protrusion for tt fonts
}{}
\usepackage{setspace}
\makeatletter
\@ifundefined{KOMAClassName}{% if non-KOMA class
  \IfFileExists{parskip.sty}{%
    \usepackage{parskip}
  }{% else
    \setlength{\parindent}{0pt}
    \setlength{\parskip}{6pt plus 2pt minus 1pt}}
}{% if KOMA class
  \KOMAoptions{parskip=half}}
\makeatother
% Make \paragraph and \subparagraph free-standing
\makeatletter
\ifx\paragraph\undefined\else
  \let\oldparagraph\paragraph
  \renewcommand{\paragraph}{
    \@ifstar
      \xxxParagraphStar
      \xxxParagraphNoStar
  }
  \newcommand{\xxxParagraphStar}[1]{\oldparagraph*{#1}\mbox{}}
  \newcommand{\xxxParagraphNoStar}[1]{\oldparagraph{#1}\mbox{}}
\fi
\ifx\subparagraph\undefined\else
  \let\oldsubparagraph\subparagraph
  \renewcommand{\subparagraph}{
    \@ifstar
      \xxxSubParagraphStar
      \xxxSubParagraphNoStar
  }
  \newcommand{\xxxSubParagraphStar}[1]{\oldsubparagraph*{#1}\mbox{}}
  \newcommand{\xxxSubParagraphNoStar}[1]{\oldsubparagraph{#1}\mbox{}}
\fi
\makeatother

\usepackage{color}
\usepackage{fancyvrb}
\newcommand{\VerbBar}{|}
\newcommand{\VERB}{\Verb[commandchars=\\\{\}]}
\DefineVerbatimEnvironment{Highlighting}{Verbatim}{commandchars=\\\{\}}
% Add ',fontsize=\small' for more characters per line
\usepackage{framed}
\definecolor{shadecolor}{RGB}{241,243,245}
\newenvironment{Shaded}{\begin{snugshade}}{\end{snugshade}}
\newcommand{\AlertTok}[1]{\textcolor[rgb]{0.68,0.00,0.00}{#1}}
\newcommand{\AnnotationTok}[1]{\textcolor[rgb]{0.37,0.37,0.37}{#1}}
\newcommand{\AttributeTok}[1]{\textcolor[rgb]{0.40,0.45,0.13}{#1}}
\newcommand{\BaseNTok}[1]{\textcolor[rgb]{0.68,0.00,0.00}{#1}}
\newcommand{\BuiltInTok}[1]{\textcolor[rgb]{0.00,0.23,0.31}{#1}}
\newcommand{\CharTok}[1]{\textcolor[rgb]{0.13,0.47,0.30}{#1}}
\newcommand{\CommentTok}[1]{\textcolor[rgb]{0.37,0.37,0.37}{#1}}
\newcommand{\CommentVarTok}[1]{\textcolor[rgb]{0.37,0.37,0.37}{\textit{#1}}}
\newcommand{\ConstantTok}[1]{\textcolor[rgb]{0.56,0.35,0.01}{#1}}
\newcommand{\ControlFlowTok}[1]{\textcolor[rgb]{0.00,0.23,0.31}{\textbf{#1}}}
\newcommand{\DataTypeTok}[1]{\textcolor[rgb]{0.68,0.00,0.00}{#1}}
\newcommand{\DecValTok}[1]{\textcolor[rgb]{0.68,0.00,0.00}{#1}}
\newcommand{\DocumentationTok}[1]{\textcolor[rgb]{0.37,0.37,0.37}{\textit{#1}}}
\newcommand{\ErrorTok}[1]{\textcolor[rgb]{0.68,0.00,0.00}{#1}}
\newcommand{\ExtensionTok}[1]{\textcolor[rgb]{0.00,0.23,0.31}{#1}}
\newcommand{\FloatTok}[1]{\textcolor[rgb]{0.68,0.00,0.00}{#1}}
\newcommand{\FunctionTok}[1]{\textcolor[rgb]{0.28,0.35,0.67}{#1}}
\newcommand{\ImportTok}[1]{\textcolor[rgb]{0.00,0.46,0.62}{#1}}
\newcommand{\InformationTok}[1]{\textcolor[rgb]{0.37,0.37,0.37}{#1}}
\newcommand{\KeywordTok}[1]{\textcolor[rgb]{0.00,0.23,0.31}{\textbf{#1}}}
\newcommand{\NormalTok}[1]{\textcolor[rgb]{0.00,0.23,0.31}{#1}}
\newcommand{\OperatorTok}[1]{\textcolor[rgb]{0.37,0.37,0.37}{#1}}
\newcommand{\OtherTok}[1]{\textcolor[rgb]{0.00,0.23,0.31}{#1}}
\newcommand{\PreprocessorTok}[1]{\textcolor[rgb]{0.68,0.00,0.00}{#1}}
\newcommand{\RegionMarkerTok}[1]{\textcolor[rgb]{0.00,0.23,0.31}{#1}}
\newcommand{\SpecialCharTok}[1]{\textcolor[rgb]{0.37,0.37,0.37}{#1}}
\newcommand{\SpecialStringTok}[1]{\textcolor[rgb]{0.13,0.47,0.30}{#1}}
\newcommand{\StringTok}[1]{\textcolor[rgb]{0.13,0.47,0.30}{#1}}
\newcommand{\VariableTok}[1]{\textcolor[rgb]{0.07,0.07,0.07}{#1}}
\newcommand{\VerbatimStringTok}[1]{\textcolor[rgb]{0.13,0.47,0.30}{#1}}
\newcommand{\WarningTok}[1]{\textcolor[rgb]{0.37,0.37,0.37}{\textit{#1}}}

\usepackage{longtable,booktabs,array}
\usepackage{calc} % for calculating minipage widths
% Correct order of tables after \paragraph or \subparagraph
\usepackage{etoolbox}
\makeatletter
\patchcmd\longtable{\par}{\if@noskipsec\mbox{}\fi\par}{}{}
\makeatother
% Allow footnotes in longtable head/foot
\IfFileExists{footnotehyper.sty}{\usepackage{footnotehyper}}{\usepackage{footnote}}
\makesavenoteenv{longtable}
\usepackage{graphicx}
\makeatletter
\newsavebox\pandoc@box
\newcommand*\pandocbounded[1]{% scales image to fit in text height/width
  \sbox\pandoc@box{#1}%
  \Gscale@div\@tempa{\textheight}{\dimexpr\ht\pandoc@box+\dp\pandoc@box\relax}%
  \Gscale@div\@tempb{\linewidth}{\wd\pandoc@box}%
  \ifdim\@tempb\p@<\@tempa\p@\let\@tempa\@tempb\fi% select the smaller of both
  \ifdim\@tempa\p@<\p@\scalebox{\@tempa}{\usebox\pandoc@box}%
  \else\usebox{\pandoc@box}%
  \fi%
}
% Set default figure placement to htbp
\def\fps@figure{htbp}
\makeatother


% definitions for citeproc citations
\NewDocumentCommand\citeproctext{}{}
\NewDocumentCommand\citeproc{mm}{%
  \begingroup\def\citeproctext{#2}\cite{#1}\endgroup}
\makeatletter
 % allow citations to break across lines
 \let\@cite@ofmt\@firstofone
 % avoid brackets around text for \cite:
 \def\@biblabel#1{}
 \def\@cite#1#2{{#1\if@tempswa , #2\fi}}
\makeatother
\newlength{\cslhangindent}
\setlength{\cslhangindent}{1.5em}
\newlength{\csllabelwidth}
\setlength{\csllabelwidth}{3em}
\newenvironment{CSLReferences}[2] % #1 hanging-indent, #2 entry-spacing
 {\begin{list}{}{%
  \setlength{\itemindent}{0pt}
  \setlength{\leftmargin}{0pt}
  \setlength{\parsep}{0pt}
  % turn on hanging indent if param 1 is 1
  \ifodd #1
   \setlength{\leftmargin}{\cslhangindent}
   \setlength{\itemindent}{-1\cslhangindent}
  \fi
  % set entry spacing
  \setlength{\itemsep}{#2\baselineskip}}}
 {\end{list}}
\usepackage{calc}
\newcommand{\CSLBlock}[1]{\hfill\break\parbox[t]{\linewidth}{\strut\ignorespaces#1\strut}}
\newcommand{\CSLLeftMargin}[1]{\parbox[t]{\csllabelwidth}{\strut#1\strut}}
\newcommand{\CSLRightInline}[1]{\parbox[t]{\linewidth - \csllabelwidth}{\strut#1\strut}}
\newcommand{\CSLIndent}[1]{\hspace{\cslhangindent}#1}

\ifLuaTeX
\usepackage[bidi=basic]{babel}
\else
\usepackage[bidi=default]{babel}
\fi
% get rid of language-specific shorthands (see #6817):
\let\LanguageShortHands\languageshorthands
\def\languageshorthands#1{}
\ifLuaTeX
  \usepackage[english]{selnolig} % disable illegal ligatures
\fi


\setlength{\emergencystretch}{3em} % prevent overfull lines

\providecommand{\tightlist}{%
  \setlength{\itemsep}{0pt}\setlength{\parskip}{0pt}}



 


% PDF Header - Custom LaTeX commands for PDF output
% This file is included in the LaTeX preamble for PDF generation

% Add any custom LaTeX packages or commands here
% Example: \usepackage{booktabs}
% Example: \usepackage{longtable}
\makeatletter
\@ifpackageloaded{caption}{}{\usepackage{caption}}
\AtBeginDocument{%
\ifdefined\contentsname
  \renewcommand*\contentsname{Table of contents}
\else
  \newcommand\contentsname{Table of contents}
\fi
\ifdefined\listfigurename
  \renewcommand*\listfigurename{List of Figures}
\else
  \newcommand\listfigurename{List of Figures}
\fi
\ifdefined\listtablename
  \renewcommand*\listtablename{List of Tables}
\else
  \newcommand\listtablename{List of Tables}
\fi
\ifdefined\figurename
  \renewcommand*\figurename{Figure}
\else
  \newcommand\figurename{Figure}
\fi
\ifdefined\tablename
  \renewcommand*\tablename{Table}
\else
  \newcommand\tablename{Table}
\fi
}
\@ifpackageloaded{float}{}{\usepackage{float}}
\floatstyle{ruled}
\@ifundefined{c@chapter}{\newfloat{codelisting}{h}{lop}}{\newfloat{codelisting}{h}{lop}[chapter]}
\floatname{codelisting}{Listing}
\newcommand*\listoflistings{\listof{codelisting}{List of Listings}}
\captionsetup{labelsep=colon}
\makeatother
\makeatletter
\makeatother
\makeatletter
\@ifpackageloaded{caption}{}{\usepackage{caption}}
\@ifpackageloaded{subcaption}{}{\usepackage{subcaption}}
\makeatother
\usepackage{bookmark}
\IfFileExists{xurl.sty}{\usepackage{xurl}}{} % add URL line breaks if available
\urlstyle{same}
\hypersetup{
  pdftitle={Astro-Seismology of the Indian Subcontinent: Statistical Validation of Planetary Triggers for Earthquake Prediction},
  pdfauthor={Bishal Ghimire},
  pdflang={en-US},
  pdfsubject={Computational Integration of Vedic Numerology and Sidereal
Astrology},
  pdfkeywords={Earthquake Prediction, India-Nepal Seismicity, Vedic
Astrology, Negative Binomial Regression, Monte Carlo
Validation, Shadbala, USGS Data},
  colorlinks=true,
  linkcolor={blue},
  filecolor={Maroon},
  citecolor={green},
  urlcolor={blue},
  pdfcreator={LaTeX via pandoc}}


\title{Astro-Seismology of the Indian Subcontinent: Statistical
Validation of Planetary Triggers for Earthquake Prediction}
\usepackage{etoolbox}
\makeatletter
\providecommand{\subtitle}[1]{% add subtitle to \maketitle
  \apptocmd{\@title}{\par {\large #1 \par}}{}{}
}
\makeatother
\subtitle{A Rigorous Investigation Using Negative Binomial Regression
and Monte Carlo Methods on India-Nepal Seismic Data (2015-2024)}
\author{Astro-Fusion Research Team}
\date{February 01, 2026}
\begin{document}
\maketitle
\begin{abstract}
This research track investigates the empirical potential for
astrological prediction of seismic activity within the India-Nepal
Tectonic Zone, one of Earth's most seismically active regions. Using
data from 370 significant earthquakes (Magnitude ≥ 4.5) recorded by the
USGS Earthquake Hazards Program between 2015-2024, we develop and test a
``Planetary Stress Index'' derived from Vedic Astrology principles
including Shadbala (six-fold planetary strength), Graha Yuddha
(planetary wars), and malefic aspects. Employing Negative Binomial
regression to account for overdispersion in earthquake counts, and Monte
Carlo permutation testing to establish empirical null distributions, we
subject the hypothesis to rigorous ``Severe Testing'' criteria. Our
results indicate that planetary variables fail to achieve statistical
significance (Mars p=0.10, Saturn p=0.99) when predicting earthquake
occurrence, with Monte Carlo validation confirming that observed signals
fall within the random noise distribution. We conclude that planetary
configurations, as operationalized through classical Vedic techniques,
do not provide reliable predictive power for seismic events in the
tested region.
\end{abstract}

% PDF Before Body - Content to include before document body
% This file is included before the main document content

% Add any content that should appear before the main document here
% This could include custom title pages, abstracts, etc.

\renewcommand*\contentsname{Table of contents}
{
\hypersetup{linkcolor=}
\setcounter{tocdepth}{3}
\tableofcontents
}
\listoffigures
\listoftables

\setstretch{1.2}
\section{Introduction}\label{introduction}

\subsection{The Earthquake Prediction
Challenge}\label{the-earthquake-prediction-challenge}

The prediction of earthquakes remains one of the most persistent
challenges in geophysical science. Despite decades of research, no
reliable deterministic earthquake prediction method exists. The
``Prediction Gap''---the inability to forecast the time, location, and
magnitude of rupture---stems from the non-linear, chaotic nature of
crustal stress accumulation.

\subsection{Astrological Claims}\label{astrological-claims}

Traditional astrology, particularly the school of \textbf{Mundane
Astrology} described in Varahamihira's Brihat Samhita (Varahamihira, c.
6th century CE), has long claimed the ability to predict natural
disasters including earthquakes through planetary configurations.
Specific claims include:

\begin{itemize}
\tightlist
\item
  \textbf{Mars-Saturn Conjunctions}: Trigger catastrophic events
\item
  \textbf{Eclipses in Cardinal Signs}: Activate seismic stress
\item
  \textbf{Malefic Transits through 4th House}: Destabilize foundations
\end{itemize}

This study tests these claims using modern statistical methods on a
well-defined geographic region.

\subsection{Research Objectives}\label{research-objectives}

\begin{enumerate}
\def\labelenumi{\arabic{enumi}.}
\tightlist
\item
  Define a quantitative ``Planetary Stress Index'' from Vedic principles
\item
  Collect and process USGS earthquake data for India-Nepal region
\item
  Apply regression analysis to test predictive relationships
\item
  Validate findings using Monte Carlo ``Look-Elsewhere'' analysis
\end{enumerate}

\section{Background: The India-Nepal Tectonic
Zone}\label{background-the-india-nepal-tectonic-zone}

\subsection{Geological Context}\label{geological-context}

The India-Nepal border sits atop one of Earth's most dramatic tectonic
features---the collision zone between the Indian Plate and the Eurasian
Plate. This ongoing collision, which began \textasciitilde50 million
years ago, creates the Himalayan mountain range and generates
significant seismic hazard.

\subsubsection{Key Geological Features}\label{key-geological-features}

\begin{longtable}[]{@{}ll@{}}
\caption{India-Nepal Tectonic
Features}\label{tbl-geology}\tabularnewline
\toprule\noalign{}
Feature & Description \\
\midrule\noalign{}
\endfirsthead
\toprule\noalign{}
Feature & Description \\
\midrule\noalign{}
\endhead
\bottomrule\noalign{}
\endlastfoot
Plate Boundary & Indian Plate subducting beneath Eurasian Plate \\
Collision Rate & \textasciitilde45-50 mm/year \\
Major Fault & Main Himalayan Thrust (MHT) \\
Seismic Gap & Central Nepal (locked since 1505 CE) \\
\end{longtable}

\subsection{Historical Seismicity}\label{historical-seismicity}

The region has experienced devastating earthquakes:

\begin{itemize}
\tightlist
\item
  \textbf{1934 Bihar-Nepal Earthquake}: M8.0, \textasciitilde10,700
  deaths
\item
  \textbf{2015 Gorkha Earthquake}: M7.8, \textasciitilde9,000 deaths
\item
  \textbf{2023 West Nepal Earthquake}: M5.6, 150+ deaths
\end{itemize}

The persistence of seismic hazard makes this an important test case for
any prediction system.

\section{Data and Methodology}\label{data-and-methodology}

\subsection{Phase 1: Earthquake Data
Collection}\label{phase-1-earthquake-data-collection}

\subsubsection{Data Source}\label{data-source}

We utilized the \textbf{USGS Earthquake Hazards Program} Comprehensive
Earthquake Catalog (ComCat), accessed via the official API.

\subsubsection{Filter Criteria}\label{filter-criteria}

\begin{longtable}[]{@{}ll@{}}
\caption{Data Collection Parameters}\label{tbl-params}\tabularnewline
\toprule\noalign{}
Parameter & Value \\
\midrule\noalign{}
\endfirsthead
\toprule\noalign{}
Parameter & Value \\
\midrule\noalign{}
\endhead
\bottomrule\noalign{}
\endlastfoot
\textbf{Latitude Range} & 20°N to 35°N \\
\textbf{Longitude Range} & 75°E to 90°E \\
\textbf{Time Period} & January 1, 2015 to January 1, 2024 \\
\textbf{Magnitude Threshold} & M ≥ 4.5 \\
\textbf{Catalog} & USGS/NEIC \\
\end{longtable}

\subsubsection{Dataset Summary}\label{dataset-summary}

After filtering and quality control:

\begin{longtable}[]{@{}ll@{}}
\caption{Dataset Summary}\label{tbl-dataset}\tabularnewline
\toprule\noalign{}
Statistic & Value \\
\midrule\noalign{}
\endfirsthead
\toprule\noalign{}
Statistic & Value \\
\midrule\noalign{}
\endhead
\bottomrule\noalign{}
\endlastfoot
Total Events & 370 \\
Magnitude Range & 4.5 to 7.8 \\
Depth Range & 5 km to 100 km \\
Peak Year & 2015 (Nepal earthquake sequence) \\
\end{longtable}

\subsection{Phase 2: Astrological Feature
Engineering}\label{phase-2-astrological-feature-engineering}

\subsubsection{The Planetary Stress
Index}\label{the-planetary-stress-index}

We developed a quantitative ``Astro-Fusion Stress Index'' based on three
classical Vedic principles:

\paragraph{1. Aggregate Shadbala (Planetary
Weakness)}\label{aggregate-shadbala-planetary-weakness}

Low combined Shadbala scores indicate planetary ``instability'':

\[
Stress_{Shadbala} = 100 - \frac{1}{n}\sum_{p}^{planets} \sigma_p
\]

\paragraph{2. Graha Yuddha (Planetary
Wars)}\label{graha-yuddha-planetary-wars}

Conjunctions within 1° longitude create tension:

\[
Stress_{Yuddha} = \sum_{i,j} \mathbb{1}[|\lambda_i - \lambda_j| < 1°]
\]

\paragraph{3. Malefic Aspects}\label{malefic-aspects}

Saturn-Mars squares (90°) and oppositions (180°):

\[
Stress_{Aspect} = \mathbb{1}[|(\lambda_{Saturn} - \lambda_{Mars})| \in \{90° \pm 5°, 180° \pm 5°\}]
\]

\subsubsection{Combined Index}\label{combined-index}

\[
Stress_{Total} = w_1 \cdot Stress_{Shadbala} + w_2 \cdot Stress_{Yuddha} + w_3 \cdot Stress_{Aspect}
\]

\subsection{Phase 3: Statistical
Framework}\label{phase-3-statistical-framework}

\subsubsection{Negative Binomial
Regression}\label{negative-binomial-regression}

Earthquake counts exhibit \textbf{overdispersion} (\(Variance > Mean\))
due to clustering. Standard Poisson regression is inappropriate. We
employ a Negative Binomial Generalized Linear Model (GLM):

\[
\ln(\mu_t) = \beta_0 + \beta_{trend}t + \beta_{season}\sin(\frac{2\pi t}{365}) + \beta_{astro}X_{astro}
\]

\subsubsection{The Null Hypothesis}\label{the-null-hypothesis}

\[
H_0: \text{Planetary variables provide no information gain over baseline model}
\]

The baseline model includes only: - Long-term catalog improvement trend
- Annual seasonal cycle (tidal effects) - Random Poisson noise

\subsubsection{Monte Carlo ``Look-Elsewhere''
Analysis}\label{monte-carlo-look-elsewhere-analysis}

To guard against false discoveries from testing multiple variables:

\begin{enumerate}
\def\labelenumi{\arabic{enumi}.}
\tightlist
\item
  Destroy temporal alignment (shuffle planetary data)
\item
  Preserve internal autocorrelation structure
\item
  Compute Delta-AIC for 1,000 permutations
\item
  Build empirical null distribution
\item
  Calculate p-value as fraction exceeding observed value
\end{enumerate}

\section{Results}\label{results}

\subsection{Seismic Activity Timeline}\label{seismic-activity-timeline}

The dataset reveals distinct patterns:

\begin{itemize}
\tightlist
\item
  \textbf{2015}: Massive clustering due to Gorkha earthquake sequence
\item
  \textbf{2016-2022}: Sporadic moderate activity
\item
  \textbf{2023}: Renewed activity in western Nepal
\end{itemize}

The non-uniform distribution confirms the need for
overdispersion-corrected models.

\subsection{Regression Analysis
Results}\label{regression-analysis-results}

\subsubsection{Model Coefficients}\label{model-coefficients}

\begin{longtable}[]{@{}lllll@{}}
\caption{Regression Coefficients}\label{tbl-regression}\tabularnewline
\toprule\noalign{}
Variable & Coefficient & Std. Error & p-value & Significant? \\
\midrule\noalign{}
\endfirsthead
\toprule\noalign{}
Variable & Coefficient & Std. Error & p-value & Significant? \\
\midrule\noalign{}
\endhead
\bottomrule\noalign{}
\endlastfoot
Intercept & -0.900 & 0.388 & 0.02 & Yes \\
Year Index & 0.011 & 0.062 & 0.86 & No \\
Universal Day 8 (Saturn) & 0.143 & 0.213 & 0.50 & No \\
Mars Strength & -0.005 & 0.003 & 0.10 & No \\
Saturn Strength & 0.000 & 0.003 & 0.99 & No \\
\end{longtable}

\subsubsection{Interpretation}\label{interpretation}

\begin{itemize}
\tightlist
\item
  \textbf{Mars Strength}: Shows weak negative correlation
  (\(p = 0.10\))---marginally suggestive but fails significance
  threshold
\item
  \textbf{Saturn Strength}: No relationship whatsoever (\(p = 0.99\))
\item
  \textbf{Universal Day 8}: No numerological signal (\(p = 0.50\))
\item
  \textbf{Year Index}: No significant trend in seismic rate
\end{itemize}

\subsection{Monte Carlo Validation}\label{monte-carlo-validation}

The ``Look-Elsewhere'' analysis tested whether observed results could
arise by chance:

\begin{longtable}[]{@{}ll@{}}
\caption{Monte Carlo Results}\label{tbl-montecarlo}\tabularnewline
\toprule\noalign{}
Metric & Value \\
\midrule\noalign{}
\endfirsthead
\toprule\noalign{}
Metric & Value \\
\midrule\noalign{}
\endhead
\bottomrule\noalign{}
\endlastfoot
Real Signal Delta-AIC & 22.13 \\
95th Percentile Noise Floor & 27.95 \\
99th Percentile Noise Floor & 35.21 \\
Empirical p-value & 0.38 \\
\end{longtable}

\textbf{Result}: The real data signal (22.13) falls \textbf{below} the
95th percentile of random noise (27.95), indicating the observed
``signal'' is indistinguishable from chance.

\subsection{Pattern Mapping
Visualization}\label{pattern-mapping-visualization}

Overlaying the Planetary Stress Index with actual earthquake occurrences
reveals:

\begin{itemize}
\tightlist
\item
  No consistent alignment between ``High Stress'' periods and major
  earthquakes
\item
  The 2015 Gorkha sequence occurred during a moderate (not extreme)
  stress period
\item
  Several ``High Stress'' periods passed without significant seismicity
\end{itemize}

\subsection{Scatter Plot Analysis}\label{scatter-plot-analysis}

Plotting Earthquake Magnitude against Planetary Stress Index:

\begin{longtable}[]{@{}ll@{}}
\caption{Scatter Analysis}\label{tbl-scatter}\tabularnewline
\toprule\noalign{}
Expected (if predictive) & Observed \\
\midrule\noalign{}
\endfirsthead
\toprule\noalign{}
Expected (if predictive) & Observed \\
\midrule\noalign{}
\endhead
\bottomrule\noalign{}
\endlastfoot
Diagonal clustering & Random scatter \\
High Mag → High Stress & Uniform distribution \\
\end{longtable}

Large magnitude events occur at both high and low stress levels with
equal probability.

\section{Discussion}\label{discussion}

\subsection{Failure to Reject Null
Hypothesis}\label{failure-to-reject-null-hypothesis}

Based on three independent lines of evidence, we \textbf{fail to reject
the null hypothesis}:

\begin{enumerate}
\def\labelenumi{\arabic{enumi}.}
\tightlist
\item
  \textbf{Regression p-values}: All planetary variables \textgreater{}
  0.05
\item
  \textbf{Monte Carlo validation}: Signal within noise distribution
\item
  \textbf{Visual inspection}: No consistent pattern matching
\end{enumerate}

\subsection{Potential Confounders}\label{potential-confounders}

\subsubsection{Solar Cycle}\label{solar-cycle}

The 11-year solar maximum cycle may overlap with certain planetary
periods (particularly Jupiter's 12-year orbit). Future studies should
include solar activity as a covariate.

\subsubsection{Tidal Stress}\label{tidal-stress}

Lunar phases (syzygy) are \textbf{known physical triggers} for
earthquake initiation. However, these represent actual gravitational
effects, not ``astrological'' influences. Our analysis separated these
mechanisms.

\subsubsection{Regional vs.~Global}\label{regional-vs.-global}

This study focused on a single tectonic region. Different results might
emerge for: - Subduction zones (Pacific Ring of Fire) - Transform
boundaries (San Andreas) - Intraplate earthquakes (New Madrid)

\subsection{The Near-Threshold Mars
Signal}\label{the-near-threshold-mars-signal}

The Mars coefficient (\(p = 0.10\)) warrants comment. While not
statistically significant:

\begin{itemize}
\tightlist
\item
  Could represent Type II error with small sample
\item
  May indicate weak but real tidal influence (Mars affects Earth's
  tides, though minimally)
\item
  More likely: random fluctuation near threshold
\end{itemize}

Future research with larger datasets (\textgreater10,000 events) could
resolve this ambiguity.

\subsection{Methodological Strengths}\label{methodological-strengths}

\begin{enumerate}
\def\labelenumi{\arabic{enumi}.}
\tightlist
\item
  \textbf{Precise Geographic Focus}: Well-defined tectonic region with
  known characteristics
\item
  \textbf{Proper Statistical Framework}: Negative Binomial for
  overdispersion
\item
  \textbf{Multiple Testing Control}: Bonferroni and Monte Carlo methods
\item
  \textbf{Reproducible Pipeline}: Open-source code and data
\end{enumerate}

\subsection{Limitations}\label{limitations}

\begin{enumerate}
\def\labelenumi{\arabic{enumi}.}
\tightlist
\item
  \textbf{Sample Size}: 370 events may be insufficient for detecting
  weak signals
\item
  \textbf{Temporal Scope}: 9 years may miss long-period planetary cycles
\item
  \textbf{Linear Assumptions}: Non-linear relationships not explored
\item
  \textbf{Technique Coverage}: Many Jyotish techniques untested (Koorma
  Chakra, Ashtakavarga)
\end{enumerate}

\section{Conclusions}\label{conclusions}

This rigorous investigation of earthquake prediction using Vedic
Astrology yields definitive results:

\begin{enumerate}
\def\labelenumi{\arabic{enumi}.}
\tightlist
\item
  \textbf{No Significant Predictors}: All tested planetary variables
  failed statistical significance
\item
  \textbf{Monte Carlo Confirmation}: Observed ``signals'' fall within
  random noise distribution
\item
  \textbf{Null Hypothesis Maintained}: Planetary configurations do not
  predict earthquakes in this dataset
\item
  \textbf{Mars Marginal Effect}: Weak signal warrants further
  investigation with larger samples
\item
  \textbf{Saturn Non-Effect}: Complete absence of relationship
  (\(p = 0.99\))
\end{enumerate}

\subsection{Implications}\label{implications}

For practitioners and researchers:

\begin{itemize}
\tightlist
\item
  \textbf{Astrological prediction of specific earthquake timing is not
  supported} by this analysis
\item
  Planetary configurations may have value for other purposes but not
  seismic forecasting
\item
  \textbf{Standard seismological methods} remain the appropriate
  approach for hazard assessment
\end{itemize}

\subsection{Future Research
Directions}\label{future-research-directions}

\begin{enumerate}
\def\labelenumi{\arabic{enumi}.}
\tightlist
\item
  \textbf{Extended Dataset}: Analysis with USGS global catalog
  (\textgreater100,000 events)
\item
  \textbf{Koorma Chakra Testing}: Regional zodiac mapping as per
  classical texts
\item
  \textbf{Machine Learning}: Non-linear approaches (Random Forests,
  Neural Networks)
\item
  \textbf{Multi-Region Comparison}: Testing across different tectonic
  environments
\end{enumerate}

\section{References}\label{references}

\phantomsection\label{refs}
\begin{CSLReferences}{1}{0}
\bibitem[\citeproctext]{ref-varahamihira_brihat_samhita}
Varahamihira. (c. 6th century CE). \emph{Brihat samhita}.

\end{CSLReferences}

\section{Appendix A: Data Processing
Pipeline}\label{appendix-a-data-processing-pipeline}

\begin{Shaded}
\begin{Highlighting}[]
\CommentTok{\# USGS API Query}
\ImportTok{import}\NormalTok{ requests}

\NormalTok{url }\OperatorTok{=} \StringTok{"https://earthquake.usgs.gov/fdsnws/event/1/query"}
\NormalTok{params }\OperatorTok{=}\NormalTok{ \{}
    \StringTok{"format"}\NormalTok{: }\StringTok{"geojson"}\NormalTok{,}
    \StringTok{"starttime"}\NormalTok{: }\StringTok{"2015{-}01{-}01"}\NormalTok{,}
    \StringTok{"endtime"}\NormalTok{: }\StringTok{"2024{-}01{-}01"}\NormalTok{,}
    \StringTok{"minlatitude"}\NormalTok{: }\DecValTok{20}\NormalTok{,}
    \StringTok{"maxlatitude"}\NormalTok{: }\DecValTok{35}\NormalTok{,}
    \StringTok{"minlongitude"}\NormalTok{: }\DecValTok{75}\NormalTok{,}
    \StringTok{"maxlongitude"}\NormalTok{: }\DecValTok{90}\NormalTok{,}
    \StringTok{"minmagnitude"}\NormalTok{: }\FloatTok{4.5}
\NormalTok{\}}
\NormalTok{response }\OperatorTok{=}\NormalTok{ requests.get(url, params}\OperatorTok{=}\NormalTok{params)}
\NormalTok{earthquakes }\OperatorTok{=}\NormalTok{ response.json()[}\StringTok{"features"}\NormalTok{]}
\end{Highlighting}
\end{Shaded}

\section{Appendix B: Negative Binomial Model
Specification}\label{appendix-b-negative-binomial-model-specification}

\begin{Shaded}
\begin{Highlighting}[]
\ImportTok{import}\NormalTok{ statsmodels.api }\ImportTok{as}\NormalTok{ sm}

\CommentTok{\# Model specification}
\NormalTok{model }\OperatorTok{=}\NormalTok{ sm.GLM(}
\NormalTok{    earthquake\_counts,}
\NormalTok{    features,}
\NormalTok{    family}\OperatorTok{=}\NormalTok{sm.families.NegativeBinomial()}
\NormalTok{)}
\NormalTok{results }\OperatorTok{=}\NormalTok{ model.fit()}
\end{Highlighting}
\end{Shaded}

\section{Appendix C: Data
Availability}\label{appendix-c-data-availability}

All earthquake data, planetary calculations, and analysis code are
available at:
https://github.com/astro-fusion/astro\_research-white-paper/tree/main/docs/research/track\_2\_earthquake\_prediction




\end{document}
