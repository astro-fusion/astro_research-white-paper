% Options for packages loaded elsewhere
% Options for packages loaded elsewhere
\PassOptionsToPackage{unicode}{hyperref}
\PassOptionsToPackage{hyphens}{url}
\PassOptionsToPackage{dvipsnames,svgnames,x11names}{xcolor}
%
\documentclass[
  american,
  11pt,
  oneside,
  openany]{scrbook}
\usepackage{xcolor}
\usepackage[top=30mm,left=25mm,right=25mm,bottom=30mm,heightrounded]{geometry}
\usepackage{amsmath,amssymb}
\setcounter{secnumdepth}{5}
\usepackage{iftex}
\ifPDFTeX
  \usepackage[T1]{fontenc}
  \usepackage[utf8]{inputenc}
  \usepackage{textcomp} % provide euro and other symbols
\else % if luatex or xetex
  \usepackage{unicode-math} % this also loads fontspec
  \defaultfontfeatures{Scale=MatchLowercase}
  \defaultfontfeatures[\rmfamily]{Ligatures=TeX,Scale=1}
\fi
\usepackage[]{libertine}
\ifPDFTeX\else
  % xetex/luatex font selection
\fi
% Use upquote if available, for straight quotes in verbatim environments
\IfFileExists{upquote.sty}{\usepackage{upquote}}{}
\IfFileExists{microtype.sty}{% use microtype if available
  \usepackage[]{microtype}
  \UseMicrotypeSet[protrusion]{basicmath} % disable protrusion for tt fonts
}{}
\usepackage{setspace}
\makeatletter
\@ifundefined{KOMAClassName}{% if non-KOMA class
  \IfFileExists{parskip.sty}{%
    \usepackage{parskip}
  }{% else
    \setlength{\parindent}{0pt}
    \setlength{\parskip}{6pt plus 2pt minus 1pt}}
}{% if KOMA class
  \KOMAoptions{parskip=half}}
\makeatother
% Make \paragraph and \subparagraph free-standing
\makeatletter
\ifx\paragraph\undefined\else
  \let\oldparagraph\paragraph
  \renewcommand{\paragraph}{
    \@ifstar
      \xxxParagraphStar
      \xxxParagraphNoStar
  }
  \newcommand{\xxxParagraphStar}[1]{\oldparagraph*{#1}\mbox{}}
  \newcommand{\xxxParagraphNoStar}[1]{\oldparagraph{#1}\mbox{}}
\fi
\ifx\subparagraph\undefined\else
  \let\oldsubparagraph\subparagraph
  \renewcommand{\subparagraph}{
    \@ifstar
      \xxxSubParagraphStar
      \xxxSubParagraphNoStar
  }
  \newcommand{\xxxSubParagraphStar}[1]{\oldsubparagraph*{#1}\mbox{}}
  \newcommand{\xxxSubParagraphNoStar}[1]{\oldsubparagraph{#1}\mbox{}}
\fi
\makeatother


\usepackage{longtable,booktabs,array}
\usepackage{calc} % for calculating minipage widths
% Correct order of tables after \paragraph or \subparagraph
\usepackage{etoolbox}
\makeatletter
\patchcmd\longtable{\par}{\if@noskipsec\mbox{}\fi\par}{}{}
\makeatother
% Allow footnotes in longtable head/foot
\IfFileExists{footnotehyper.sty}{\usepackage{footnotehyper}}{\usepackage{footnote}}
\makesavenoteenv{longtable}
\usepackage{graphicx}
\makeatletter
\newsavebox\pandoc@box
\newcommand*\pandocbounded[1]{% scales image to fit in text height/width
  \sbox\pandoc@box{#1}%
  \Gscale@div\@tempa{\textheight}{\dimexpr\ht\pandoc@box+\dp\pandoc@box\relax}%
  \Gscale@div\@tempb{\linewidth}{\wd\pandoc@box}%
  \ifdim\@tempb\p@<\@tempa\p@\let\@tempa\@tempb\fi% select the smaller of both
  \ifdim\@tempa\p@<\p@\scalebox{\@tempa}{\usebox\pandoc@box}%
  \else\usebox{\pandoc@box}%
  \fi%
}
% Set default figure placement to htbp
\def\fps@figure{htbp}
\makeatother



\ifLuaTeX
\usepackage[bidi=basic]{babel}
\else
\usepackage[bidi=default]{babel}
\fi
% get rid of language-specific shorthands (see #6817):
\let\LanguageShortHands\languageshorthands
\def\languageshorthands#1{}
\ifLuaTeX
  \usepackage[english]{selnolig} % disable illegal ligatures
\fi


\setlength{\emergencystretch}{3em} % prevent overfull lines

\providecommand{\tightlist}{%
  \setlength{\itemsep}{0pt}\setlength{\parskip}{0pt}}



 


% PDF Header - Custom LaTeX commands for PDF output
% This file is included in the LaTeX preamble for PDF generation

% Add any custom LaTeX packages or commands here
% Example: \usepackage{booktabs}
% Example: \usepackage{longtable}
\makeatletter
\@ifpackageloaded{caption}{}{\usepackage{caption}}
\AtBeginDocument{%
\ifdefined\contentsname
  \renewcommand*\contentsname{Table of contents}
\else
  \newcommand\contentsname{Table of contents}
\fi
\ifdefined\listfigurename
  \renewcommand*\listfigurename{List of Figures}
\else
  \newcommand\listfigurename{List of Figures}
\fi
\ifdefined\listtablename
  \renewcommand*\listtablename{List of Tables}
\else
  \newcommand\listtablename{List of Tables}
\fi
\ifdefined\figurename
  \renewcommand*\figurename{Figure}
\else
  \newcommand\figurename{Figure}
\fi
\ifdefined\tablename
  \renewcommand*\tablename{Table}
\else
  \newcommand\tablename{Table}
\fi
}
\@ifpackageloaded{float}{}{\usepackage{float}}
\floatstyle{ruled}
\@ifundefined{c@chapter}{\newfloat{codelisting}{h}{lop}}{\newfloat{codelisting}{h}{lop}[chapter]}
\floatname{codelisting}{Listing}
\newcommand*\listoflistings{\listof{codelisting}{List of Listings}}
\captionsetup{labelsep=colon}
\makeatother
\makeatletter
\makeatother
\makeatletter
\@ifpackageloaded{caption}{}{\usepackage{caption}}
\@ifpackageloaded{subcaption}{}{\usepackage{subcaption}}
\makeatother
\usepackage{bookmark}
\IfFileExists{xurl.sty}{\usepackage{xurl}}{} % add URL line breaks if available
\urlstyle{same}
\hypersetup{
  pdftitle={Track 2: Astro-Seismology of the Indian Subcontinent},
  pdfauthor={Astro-Fusion Research Team},
  pdflang={en-US},
  pdfsubject={Computational Integration of Vedic Numerology and Sidereal
Astrology},
  pdfkeywords={Earthquakes, India-Nepal, Indian Plate, USGS
Data, Prediction, Shadbala, Pattern Matching},
  colorlinks=true,
  linkcolor={blue},
  filecolor={Maroon},
  citecolor={green},
  urlcolor={blue},
  pdfcreator={LaTeX via pandoc}}


\title{Track 2: Astro-Seismology of the Indian Subcontinent}
\usepackage{etoolbox}
\makeatletter
\providecommand{\subtitle}[1]{% add subtitle to \maketitle
  \apptocmd{\@title}{\par {\large #1 \par}}{}{}
}
\makeatother
\subtitle{A Scientific Investigation of Planetary Triggers for
India-Nepal Earthquakes (2015-2024)}
\author{Astro-Fusion Research Team}
\date{2026-01-26}
\begin{document}
\frontmatter
\maketitle

% PDF Before Body - Content to include before document body
% This file is included before the main document content

% Add any content that should appear before the main document here
% This could include custom title pages, abstracts, etc.


\setstretch{1.2}
\mainmatter
\chapter{Abstract}\label{abstract}

This research track investigates the potential for Astrological
Prediction of seismic activity specifically within the
\textbf{India-Nepal Tectonic Zone}. We define a strict data collection
protocol using USGS records for the region (Lat 20-35N, Lon 75-90E).
Analyzing \textbf{370 significant events} (Magnitude \textgreater{} 4.5)
from 2015 to 2024, we map these occurrences against a ``Planetary Stress
Index'' derived from Vedic Astrology principles (Shadbala and Geometric
Aspects). The study aims to compare and contrast the recurrence patterns
of Tectonic stress vs.~Astrological stress to validate if a predictive
signal exists.

\chapter{1. Phase 1: Earthquake Data Definition \&
Introspection}\label{phase-1-earthquake-data-definition-introspection}

\section{1.1. Data Collection
Methodology}\label{data-collection-methodology}

To ensure scientific rigor, we define our ``Target Region'' as the
active collision zone of the Indian Plate.

\begin{itemize}
\tightlist
\item
  \textbf{Source:} United States Geological Survey (USGS) Earthquake
  Hazards Program API.
\item
  \textbf{Region:} India-Nepal Border Area.

  \begin{itemize}
  \tightlist
  \item
    \emph{Latitude:} \(20^\circ N\) to \(35^\circ N\)
  \item
    \emph{Longitude:} \(75^\circ E\) to \(90^\circ E\)
  \end{itemize}
\item
  \textbf{Period:} January 1, 2015 to January 1, 2024 (9 Years).
\item
  \textbf{Threshold:} Moment Magnitude (\(M_w\)) \(\ge 4.5\).
\end{itemize}

\section{1.2. Introspective Analysis (The
Dataset)}\label{introspective-analysis-the-dataset}

We successfully fetched and processed \textbf{370 distinct seismic
events}.

\textbf{Key Characteristics:} - \textbf{Intensity Distribution:} The
dataset includes major events (Mag \textgreater{} 7.0) and frequent
moderate tremors (Mag 4.5-5.5). - \textbf{Temporal Clustering:} We
observe distinct ``swarms'' of activity rather than a uniform
distribution.

\begin{figure}[H]

\caption{\label{fig-timeline}Phase 1: Seismic Activity Timeline
(India-Nepal). The bubble size represents Magnitude intensity
(\(Mag^4\)). Note the high density of events in 2015 (Nepal Earthquake
sequence) and sporadic activity thereafter.}

\centering{

\includegraphics[width=0.9\linewidth,height=\textheight,keepaspectratio]{figures/phase1_seismic_timeline.png}

}

\end{figure}%

This timeline establishes the \textbf{``Ground Truth''}---the physical
reality we are attempting to predict.

\chapter{2. Phase 2: Astrology-Based Prediction
Model}\label{phase-2-astrology-based-prediction-model}

\section{2.1. Defining the
Combinations}\label{defining-the-combinations}

In this phase, we apply Vedic Astrology principles to define ``Trigger''
conditions. We hypothesize that earthquakes are more likely during
periods of high \textbf{``Planetary Stress''}.

\textbf{The Formula:} We utilize the \textbf{Astro-Fusion Stress Index},
calculated by analyzing: 1. \textbf{Shadbala (6-Fold Strength):} Low
Aggregate Strength = Instability. 2. \textbf{Graha Yuddha (Planetary
War):} Conjunctions within \(1^\circ\). 3. \textbf{Malefic Aspects:}
Saturn-Mars mutual aspects (\(180^\circ\) or \(90^\circ\)).

\section{2.2. Mapping Patterns (The
Repeater)}\label{mapping-patterns-the-repeater}

We map this ``Stress Index'' (Purple Line) against the actual Earthquake
events (Red Markers) to visualize potential synchronization.

\begin{figure}[H]

\caption{\label{fig-pattern}Phase 2: Pattern Mapping. The Purple Line
represents the modeled Astrological Stress. The Red Dashed Lines are
major earthquakes (Mag \textgreater{} 5.5). We look for alignment
between the Purple Peaks and Red Lines.}

\centering{

\includegraphics[width=0.9\linewidth,height=\textheight,keepaspectratio]{figures/phase2_astro_pattern_map.png}

}

\end{figure}%

\textbf{Observation:} The goal is to see if the ``Pattern of the Sky''
(Purple) repeats in sync with the ``Pattern of the Earth'' (Red).

\chapter{3. Phase 3: Validation (Compare \&
Contrast)}\label{phase-3-validation-compare-contrast}

\section{3.1. Machine Learning
Validation}\label{machine-learning-validation}

To move beyond visual anecdotal evidence, we perform a statistical
\textbf{Correlation Check}. We scatter the \emph{Planetary Stress Index}
(X-axis) against the \emph{Earthquake Magnitude} (Y-axis) for all 370
events.

\section{3.2. The Logic of Deduction}\label{the-logic-of-deduction}

\begin{itemize}
\tightlist
\item
  \textbf{Hypothesis:} If Astrology predicts earthquakes, large
  magnitudes (Y-High) should cluster at High Stress (X-High). We expect
  a \textbf{Linear Trend} (diagonal).
\item
  \textbf{Results:}
\end{itemize}

\begin{figure}[H]

\caption{\label{fig-scatter}Phase 3: Validation Scatter Plot. The
distribution is uniform/random. There is no clear diagonal clustering.
High magnitude events occur at both High and Low Astrological Stress
levels.}

\centering{

\includegraphics[width=0.8\linewidth,height=\textheight,keepaspectratio]{figures/phase3_validation_scatter.png}

}

\end{figure}%

\section{3.3. Conclusion}\label{conclusion}

Comparing the ``Pattern of the Sky'' with the ``Pattern of the Earth''
reveals: 1. \textbf{No Direct Pattern:} The Astrological Stress model
acts as a ``Random Number Generator'' relative to the specific
India-Nepal seismic record. 2. \textbf{Independence:} The placements of
planets (Saturn/Mars) do not \emph{consistently} trigger earthquakes in
this specific geographic bounding box. 3. \textbf{Final Verdict:} While
individual coincidences exist, \textbf{Planetary Positions cannot
reliably predict earthquake patterns} in this dataset using the current
linear methodology.


\backmatter


\end{document}
