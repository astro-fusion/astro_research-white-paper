% Options for packages loaded elsewhere
\PassOptionsToPackage{unicode}{hyperref}
\PassOptionsToPackage{hyphens}{url}
\PassOptionsToPackage{dvipsnames,svgnames,x11names}{xcolor}
%
\documentclass[
  11pt,
  a4paper,
  oneside,
  openany]{article}

\usepackage{amsmath,amssymb}
\usepackage{setspace}
\usepackage{iftex}
\ifPDFTeX
  \usepackage[T1]{fontenc}
  \usepackage[utf8]{inputenc}
  \usepackage{textcomp} % provide euro and other symbols
\else % if luatex or xetex
  \usepackage{unicode-math}
  \defaultfontfeatures{Scale=MatchLowercase}
  \defaultfontfeatures[\rmfamily]{Ligatures=TeX,Scale=1}
\fi
\usepackage[]{libertine}
\ifPDFTeX\else
    % xetex/luatex font selection
\fi
% Use upquote if available, for straight quotes in verbatim environments
\IfFileExists{upquote.sty}{\usepackage{upquote}}{}
\IfFileExists{microtype.sty}{% use microtype if available
  \usepackage[]{microtype}
  \UseMicrotypeSet[protrusion]{basicmath} % disable protrusion for tt fonts
}{}
\makeatletter
\@ifundefined{KOMAClassName}{% if non-KOMA class
  \IfFileExists{parskip.sty}{%
    \usepackage{parskip}
  }{% else
    \setlength{\parindent}{0pt}
    \setlength{\parskip}{6pt plus 2pt minus 1pt}}
}{% if KOMA class
  \KOMAoptions{parskip=half}}
\makeatother
\usepackage{xcolor}
\usepackage[top=30mm,left=25mm,right=25mm,bottom=30mm,heightrounded,top=25mm,bottom=25mm]{geometry}
\setlength{\emergencystretch}{3em} % prevent overfull lines
\setcounter{secnumdepth}{5}
% Make \paragraph and \subparagraph free-standing
\ifx\paragraph\undefined\else
  \let\oldparagraph\paragraph
  \renewcommand{\paragraph}[1]{\oldparagraph{#1}\mbox{}}
\fi
\ifx\subparagraph\undefined\else
  \let\oldsubparagraph\subparagraph
  \renewcommand{\subparagraph}[1]{\oldsubparagraph{#1}\mbox{}}
\fi

\usepackage{color}
\usepackage{fancyvrb}
\newcommand{\VerbBar}{|}
\newcommand{\VERB}{\Verb[commandchars=\\\{\}]}
\DefineVerbatimEnvironment{Highlighting}{Verbatim}{commandchars=\\\{\}}
% Add ',fontsize=\small' for more characters per line
\usepackage{framed}
\definecolor{shadecolor}{RGB}{241,243,245}
\newenvironment{Shaded}{\begin{snugshade}}{\end{snugshade}}
\newcommand{\AlertTok}[1]{\textcolor[rgb]{0.68,0.00,0.00}{#1}}
\newcommand{\AnnotationTok}[1]{\textcolor[rgb]{0.37,0.37,0.37}{#1}}
\newcommand{\AttributeTok}[1]{\textcolor[rgb]{0.40,0.45,0.13}{#1}}
\newcommand{\BaseNTok}[1]{\textcolor[rgb]{0.68,0.00,0.00}{#1}}
\newcommand{\BuiltInTok}[1]{\textcolor[rgb]{0.00,0.23,0.31}{#1}}
\newcommand{\CharTok}[1]{\textcolor[rgb]{0.13,0.47,0.30}{#1}}
\newcommand{\CommentTok}[1]{\textcolor[rgb]{0.37,0.37,0.37}{#1}}
\newcommand{\CommentVarTok}[1]{\textcolor[rgb]{0.37,0.37,0.37}{\textit{#1}}}
\newcommand{\ConstantTok}[1]{\textcolor[rgb]{0.56,0.35,0.01}{#1}}
\newcommand{\ControlFlowTok}[1]{\textcolor[rgb]{0.00,0.23,0.31}{#1}}
\newcommand{\DataTypeTok}[1]{\textcolor[rgb]{0.68,0.00,0.00}{#1}}
\newcommand{\DecValTok}[1]{\textcolor[rgb]{0.68,0.00,0.00}{#1}}
\newcommand{\DocumentationTok}[1]{\textcolor[rgb]{0.37,0.37,0.37}{\textit{#1}}}
\newcommand{\ErrorTok}[1]{\textcolor[rgb]{0.68,0.00,0.00}{#1}}
\newcommand{\ExtensionTok}[1]{\textcolor[rgb]{0.00,0.23,0.31}{#1}}
\newcommand{\FloatTok}[1]{\textcolor[rgb]{0.68,0.00,0.00}{#1}}
\newcommand{\FunctionTok}[1]{\textcolor[rgb]{0.28,0.35,0.67}{#1}}
\newcommand{\ImportTok}[1]{\textcolor[rgb]{0.00,0.46,0.62}{#1}}
\newcommand{\InformationTok}[1]{\textcolor[rgb]{0.37,0.37,0.37}{#1}}
\newcommand{\KeywordTok}[1]{\textcolor[rgb]{0.00,0.23,0.31}{#1}}
\newcommand{\NormalTok}[1]{\textcolor[rgb]{0.00,0.23,0.31}{#1}}
\newcommand{\OperatorTok}[1]{\textcolor[rgb]{0.37,0.37,0.37}{#1}}
\newcommand{\OtherTok}[1]{\textcolor[rgb]{0.00,0.23,0.31}{#1}}
\newcommand{\PreprocessorTok}[1]{\textcolor[rgb]{0.68,0.00,0.00}{#1}}
\newcommand{\RegionMarkerTok}[1]{\textcolor[rgb]{0.00,0.23,0.31}{#1}}
\newcommand{\SpecialCharTok}[1]{\textcolor[rgb]{0.37,0.37,0.37}{#1}}
\newcommand{\SpecialStringTok}[1]{\textcolor[rgb]{0.13,0.47,0.30}{#1}}
\newcommand{\StringTok}[1]{\textcolor[rgb]{0.13,0.47,0.30}{#1}}
\newcommand{\VariableTok}[1]{\textcolor[rgb]{0.07,0.07,0.07}{#1}}
\newcommand{\VerbatimStringTok}[1]{\textcolor[rgb]{0.13,0.47,0.30}{#1}}
\newcommand{\WarningTok}[1]{\textcolor[rgb]{0.37,0.37,0.37}{\textit{#1}}}

\providecommand{\tightlist}{%
  \setlength{\itemsep}{0pt}\setlength{\parskip}{0pt}}\usepackage{longtable,booktabs,array}
\usepackage{calc} % for calculating minipage widths
% Correct order of tables after \paragraph or \subparagraph
\usepackage{etoolbox}
\makeatletter
\patchcmd\longtable{\par}{\if@noskipsec\mbox{}\fi\par}{}{}
\makeatother
% Allow footnotes in longtable head/foot
\IfFileExists{footnotehyper.sty}{\usepackage{footnotehyper}}{\usepackage{footnote}}
\makesavenoteenv{longtable}
\usepackage{graphicx}
\makeatletter
\def\maxwidth{\ifdim\Gin@nat@width>\linewidth\linewidth\else\Gin@nat@width\fi}
\def\maxheight{\ifdim\Gin@nat@height>\textheight\textheight\else\Gin@nat@height\fi}
\makeatother
% Scale images if necessary, so that they will not overflow the page
% margins by default, and it is still possible to overwrite the defaults
% using explicit options in \includegraphics[width, height, ...]{}
\setkeys{Gin}{width=\maxwidth,height=\maxheight,keepaspectratio}
% Set default figure placement to htbp
\makeatletter
\def\fps@figure{htbp}
\makeatother
% definitions for citeproc citations
\NewDocumentCommand\citeproctext{}{}
\NewDocumentCommand\citeproc{mm}{%
  \begingroup\def\citeproctext{#2}\cite{#1}\endgroup}
\makeatletter
 % allow citations to break across lines
 \let\@cite@ofmt\@firstofone
 % avoid brackets around text for \cite:
 \def\@biblabel#1{}
 \def\@cite#1#2{{#1\if@tempswa , #2\fi}}
\makeatother
\newlength{\cslhangindent}
\setlength{\cslhangindent}{1.5em}
\newlength{\csllabelwidth}
\setlength{\csllabelwidth}{3em}
\newenvironment{CSLReferences}[2] % #1 hanging-indent, #2 entry-spacing
 {\begin{list}{}{%
  \setlength{\itemindent}{0pt}
  \setlength{\leftmargin}{0pt}
  \setlength{\parsep}{0pt}
  % turn on hanging indent if param 1 is 1
  \ifodd #1
   \setlength{\leftmargin}{\cslhangindent}
   \setlength{\itemindent}{-1\cslhangindent}
  \fi
  % set entry spacing
  \setlength{\itemsep}{#2\baselineskip}}}
 {\end{list}}
\usepackage{calc}
\newcommand{\CSLBlock}[1]{\hfill\break\parbox[t]{\linewidth}{\strut\ignorespaces#1\strut}}
\newcommand{\CSLLeftMargin}[1]{\parbox[t]{\csllabelwidth}{\strut#1\strut}}
\newcommand{\CSLRightInline}[1]{\parbox[t]{\linewidth - \csllabelwidth}{\strut#1\strut}}
\newcommand{\CSLIndent}[1]{\hspace{\cslhangindent}#1}

% PDF Header - Custom LaTeX commands for PDF output
% This file is included in the LaTeX preamble for PDF generation

% Add any custom LaTeX packages or commands here
% Example: \usepackage{booktabs}
% Example: \usepackage{longtable}
\makeatletter
\@ifpackageloaded{caption}{}{\usepackage{caption}}
\AtBeginDocument{%
\ifdefined\contentsname
  \renewcommand*\contentsname{Table of contents}
\else
  \newcommand\contentsname{Table of contents}
\fi
\ifdefined\listfigurename
  \renewcommand*\listfigurename{List of Figures}
\else
  \newcommand\listfigurename{List of Figures}
\fi
\ifdefined\listtablename
  \renewcommand*\listtablename{List of Tables}
\else
  \newcommand\listtablename{List of Tables}
\fi
\ifdefined\figurename
  \renewcommand*\figurename{Figure}
\else
  \newcommand\figurename{Figure}
\fi
\ifdefined\tablename
  \renewcommand*\tablename{Table}
\else
  \newcommand\tablename{Table}
\fi
}
\@ifpackageloaded{float}{}{\usepackage{float}}
\floatstyle{ruled}
\@ifundefined{c@chapter}{\newfloat{codelisting}{h}{lop}}{\newfloat{codelisting}{h}{lop}[chapter]}
\floatname{codelisting}{Listing}
\newcommand*\listoflistings{\listof{codelisting}{List of Listings}}
\captionsetup{labelsep=colon}
\makeatother
\makeatletter
\makeatother
\makeatletter
\@ifpackageloaded{caption}{}{\usepackage{caption}}
\@ifpackageloaded{subcaption}{}{\usepackage{subcaption}}
\makeatother
\ifLuaTeX
\usepackage[bidi=basic]{babel}
\else
\usepackage[bidi=default]{babel}
\fi
\babelprovide[main,import]{american}
% get rid of language-specific shorthands (see #6817):
\let\LanguageShortHands\languageshorthands
\def\languageshorthands#1{}
\ifLuaTeX
  \usepackage{selnolig}  % disable illegal ligatures
\fi
\usepackage{bookmark}

\IfFileExists{xurl.sty}{\usepackage{xurl}}{} % add URL line breaks if available
\urlstyle{same} % disable monospaced font for URLs
\hypersetup{
  pdftitle={Empirical Analysis of Vedic Predictive Systems: A Computational Investigation of Astrology, Numerology, and Terrestrial Phenomena},
  pdfauthor={Bishal Ghimire},
  pdflang={en-US},
  pdfsubject={Computational Integration of Vedic Numerology and Sidereal
Astrology},
  pdfkeywords={Vedic Astrology, Vedic Numerology, Earthquake
Prediction, Financial Astrology, Statistical Falsification, Granger
Causality, Swiss Ephemeris, Shadbala},
  colorlinks=true,
  linkcolor={blue},
  filecolor={Maroon},
  citecolor={green},
  urlcolor={blue},
  pdfcreator={LaTeX via pandoc}}

\title{Empirical Analysis of Vedic Predictive Systems: A Computational
Investigation of Astrology, Numerology, and Terrestrial Phenomena}
\usepackage{etoolbox}
\makeatletter
\providecommand{\subtitle}[1]{% add subtitle to \maketitle
  \apptocmd{\@title}{\par {\large #1 \par}}{}{}
}
\makeatother
\subtitle{A Consolidated Research Report Applying Rigorous Statistical
Methods to Traditional Knowledge Systems}
\author{Astro-Fusion Research Team}
\date{February 05, 2026}

\begin{document}
\maketitle
\begin{abstract}
This comprehensive research report presents a rigorous, multi-track
investigation into the empirical validity of Vedic predictive
systems---specifically Vedic Astrology (Jyotish) and Vedic Numerology
(Anka Jyotish). Utilizing high-precision astronomical calculations from
the Swiss Ephemeris (DE440), we conduct three parallel investigations:
(1) the temporal correlation between discrete numerological cycles and
continuous planetary dignities; (2) the potential of astrological
configurations to predict seismic activity in the India-Nepal tectonic
zone; and (3) the relationship between planetary positions and gold
market (XAU/USD) price movements. Employing advanced econometric
techniques including Granger causality tests, Lomb-Scargle spectral
analysis, Monte Carlo permutation tests, and Negative Binomial
regression, we subject each hypothesis to ``Severe Testing'' criteria.
Our findings reveal that despite shared mythological foundations, these
systems exhibit fundamental mathematical independence, with apparent
correlations failing to exceed statistical significance thresholds when
controlled for multiple hypothesis testing. This work provides a
reproducible computational framework for the scientific evaluation of
traditional knowledge systems.
\end{abstract}

% PDF Before Body - Content to include before document body
% This file is included before the main document content

% Add any content that should appear before the main document here
% This could include custom title pages, abstracts, etc.

\renewcommand*\contentsname{Table of contents}
{
\hypersetup{linkcolor=}
\setcounter{tocdepth}{3}
\tableofcontents
}
\listoffigures
\listoftables
\setstretch{1.2}
\section{Introduction}\label{introduction}

\subsection{Background and Motivation}\label{background-and-motivation}

The intersection of celestial observation and human affairs has been a
cornerstone of civilizations across the globe. In the Indian
subcontinent, this relationship crystallized into the sophisticated
systems of \textbf{Vedic Astrology (Jyotish)}---literally ``the science
of light''---and \textbf{Vedic Numerology (Anka Jyotish)}---the science
of numbers. Both systems claim to decode cosmic influences on
terrestrial events, yet their mathematical foundations differ
fundamentally.

This research initiative was born from a simple question: \textbf{Do
these ancient systems possess empirical validity when subjected to
modern statistical scrutiny?} Rather than approaching this question with
either blind acceptance or dismissive skepticism, we apply the
principles of ``Severe Testing'' (Mayo \& Spanos, 2006)---establishing
clear null hypotheses and only rejecting them in the face of
overwhelming statistical evidence.

\subsection{Scope of Investigation}\label{scope-of-investigation}

This report consolidates findings from three independent research
tracks:

\begin{enumerate}
\def\labelenumi{\arabic{enumi}.}
\item
  \textbf{Track 1: Numerology-Astrology Temporal Discontinuity} ---
  Investigating whether discrete numerological cycles correlate with
  continuous planetary strength scores.
\item
  \textbf{Track 2: Earthquake Prediction Analysis} --- Testing whether
  planetary configurations predict seismic activity in the India-Nepal
  tectonic zone.
\item
  \textbf{Track 3: Gold Market Correlation} --- Evaluating whether
  planetary positions influence XAU/USD spot prices, thus challenging
  the Efficient Market Hypothesis (EMH).
\end{enumerate}

\subsection{Research Philosophy}\label{research-philosophy}

We adopt the stance of Karl Popper's falsificationism (Popper, 1959): a
hypothesis gains scientific credibility not through confirmation but
through surviving rigorous attempts at falsification. Each research
track is designed as an attempt to \textbf{disprove} rather than prove
astrological claims.

\section{Literature Review}\label{literature-review}

\subsection{Classical Vedic Sources}\label{classical-vedic-sources}

The foundational text of Vedic Astrology is the \textbf{Brihat Parashara
Hora Shastra} (BPHS) (Parashara, c. 7th century CE), attributed to Sage
Parashara (c.~7th century CE). This text codifies:

\begin{itemize}
\tightlist
\item
  \textbf{Rashi (Signs)}: 12 divisions of the ecliptic
\item
  \textbf{Nakshatras}: 27 lunar mansions
\item
  \textbf{Graha (Planets)}: 9 celestial bodies including lunar nodes
\item
  \textbf{Shadbala}: Six-fold planetary strength calculation
\end{itemize}

The astronomical calculations are grounded in the \textbf{Surya
Siddhanta} (Anonymous, c. 4th century CE), which provides remarkably
accurate planetary ephemerides for its era. \textbf{Varahamihira's
Brihat Samhita} (Varahamihira, c. 6th century CE) extends these
principles to mundane astrology, including predictions of earthquakes,
weather, and political events.

Vedic Numerology draws from similar planetary associations, mapping
integers 1-9 to the Navagraha (nine planets) through modulo-9 arithmetic
on calendar dates.

\subsection{Modern Statistical
Framework}\label{modern-statistical-framework}

Our methodology draws from established econometric and signal processing
literature:

\begin{itemize}
\tightlist
\item
  \textbf{Granger Causality} (Granger, 1969): Testing whether past
  values of one variable improve predictions of another
\item
  \textbf{Lomb-Scargle Spectral Analysis} (Lomb, 1976): Detecting
  periodic signals in unevenly sampled data
\item
  \textbf{Unit Root Tests} (Dickey \& Fuller, 1979): Ensuring
  stationarity for valid time-series inference
\item
  \textbf{Monte Carlo Methods} (Fishman, 1996): Establishing empirical
  null distributions
\end{itemize}

\subsection{Previous Studies}\label{previous-studies}

Prior academic investigations of astrology have generally concluded null
results. However, most studies suffer from:

\begin{enumerate}
\def\labelenumi{\arabic{enumi}.}
\tightlist
\item
  Imprecise astronomical calculations (using Western tropical zodiac
  instead of Vedic sidereal)
\item
  Lack of Ayanamsa correction for precession
\item
  Failure to test specific Vedic techniques (Shadbala, Dasha systems)
\end{enumerate}

This study addresses these gaps by utilizing high-precision Swiss
Ephemeris (AG, 2023) calculations with proper Lahiri Ayanamsa (Lahiri,
1960) correction.

\section{Methodology}\label{methodology}

\subsection{Astronomical Calculations}\label{astronomical-calculations}

\subsubsection{Swiss Ephemeris
Implementation}\label{swiss-ephemeris-implementation}

All planetary positions are calculated using the \textbf{Swiss Ephemeris
(version 2.10)} with the DE440 JPL ephemeris, providing:

\begin{itemize}
\tightlist
\item
  Positional accuracy: \textless{} 0.001 arcseconds (1 milliarcsecond)
\item
  Temporal range: 13201 BCE to 17191 CE
\item
  Proper sidereal conversion via Lahiri Ayanamsa
\end{itemize}

The sidereal longitude \(\lambda_{sid}\) is calculated as:

\[
\lambda_{sid}(t) = \lambda_{trop}(t) - \alpha_{Lahiri}(t)
\]

where \(\alpha_{Lahiri}(t)\) is the precession-corrected Ayanamsa value
(approximately 24° in 2024).

\subsubsection{Shadbala (Six-Fold Strength)
Calculation}\label{shadbala-six-fold-strength-calculation}

Following BPHS specifications, planetary strength \(\sigma_p\) is
computed as:

\[
\sigma_p = \sum_{i=1}^{6} w_i \cdot B_i
\]

where the six Balas are:

\begin{enumerate}
\def\labelenumi{\arabic{enumi}.}
\tightlist
\item
  \textbf{Sthana Bala} (Positional): Sign placement
  (exaltation/debilitation)
\item
  \textbf{Dig Bala} (Directional): House position from angles
\item
  \textbf{Kala Bala} (Temporal): Day/night, hora, seasonal strength
\item
  \textbf{Chesta Bala} (Motional): Speed and retrograde status
\item
  \textbf{Naisargika Bala} (Natural): Intrinsic luminosity
\item
  \textbf{Drik Bala} (Aspectual): Benefic/malefic aspects received
\end{enumerate}

Each component is normalized to a 0-100 scale for cross-planet
comparison.

\subsection{Numerological
Calculations}\label{numerological-calculations}

The \textbf{Mulanka} (Root Number) for any date is calculated via
digital root reduction:

\[
Mulanka = ((Day - 1) \bmod 9) + 1
\]

The planetary mapping follows the standard Vedic schema:

\begin{longtable}[]{@{}lll@{}}
\caption{Vedic Numerology Planetary
Mapping}\label{tbl-numerology}\tabularnewline
\toprule\noalign{}
Number & Planet & Sanskrit \\
\midrule\noalign{}
\endfirsthead
\toprule\noalign{}
Number & Planet & Sanskrit \\
\midrule\noalign{}
\endhead
\bottomrule\noalign{}
\endlastfoot
1 & Sun & Surya \\
2 & Moon & Chandra \\
3 & Jupiter & Brihaspati \\
4 & Rahu & - \\
5 & Mercury & Budha \\
6 & Venus & Shukra \\
7 & Ketu & - \\
8 & Saturn & Shani \\
9 & Mars & Mangal \\
\end{longtable}

\subsection{Statistical Framework}\label{statistical-framework}

\subsubsection{Stationarity Validation}\label{stationarity-validation}

All time series are tested for stationarity using the Augmented
Dickey-Fuller (ADF) test (Dickey \& Fuller, 1979):

\[
H_0: \text{Series has a unit root (non-stationary)}
\]

Non-stationary series are differenced or log-transformed before
analysis.

\subsubsection{Granger Causality
Testing}\label{granger-causality-testing}

For each planetary variable \(X\) and target variable \(Y\), we test:

\[
H_0: X \text{ does not Granger-cause } Y
\]

using a Vector Autoregression (VAR) framework with optimal lag selection
via AIC criterion.

\subsubsection{Multiple Testing
Correction}\label{multiple-testing-correction}

To control the Family-Wise Error Rate (FWER) across multiple planet
tests, we apply the Bonferroni correction:

\[
\alpha_{adjusted} = \frac{\alpha}{n}
\]

where \(n\) is the number of independent hypotheses tested.

\subsubsection{Monte Carlo Permutation
Testing}\label{monte-carlo-permutation-testing}

To establish empirical null distributions, we perform 1,000 Monte Carlo
shuffles:

\begin{enumerate}
\def\labelenumi{\arabic{enumi}.}
\tightlist
\item
  Destroy temporal alignment between predictor and target
\item
  Preserve internal autocorrelation structure
\item
  Compute test statistic for each permutation
\item
  Calculate p-value as proportion exceeding observed value
\end{enumerate}

\section{Track 1: Numerology-Astrology Temporal
Discontinuity}\label{track-1-numerology-astrology-temporal-discontinuity}

\subsection{Research Question}\label{research-question}

\textbf{Do the discrete numerological cycles (changing daily) correlate
with continuous planetary dignity scores (changing hourly)?}

\subsection{Data Collection}\label{data-collection}

\begin{itemize}
\tightlist
\item
  \textbf{Period}: January 1, 2024 to December 31, 2024
\item
  \textbf{Astrological Resolution}: 2-hour intervals (4,380 data points
  per planet)
\item
  \textbf{Numerological Resolution}: Daily values (365 data points)
\item
  \textbf{Location}: New Delhi, India (28.6°N, 77.1°E)
\end{itemize}

\subsubsection{Visualizing Planetary
Cycles}\label{visualizing-planetary-cycles}

The following figure illustrates the normalized cyclic variations of key
planets over the study period. Note the distinct frequencies of Mars
vs.~Saturn, highlighting the temporal complexity.

\begin{figure}[H]

\caption{\label{fig-cycles}Planetary Cycles}

\centering{

\includegraphics[width=1\textwidth,height=\textheight]{../../reports/artifacts/fig_multi_planet_cycles.pdf}

}

\end{figure}%

\subsection{Analysis: Temporal Mechanics
Comparison}\label{analysis-temporal-mechanics-comparison}

\subsubsection{Frequency Mismatch}\label{frequency-mismatch}

The fundamental observation is a \textbf{120:1 frequency mismatch}:

\begin{itemize}
\tightlist
\item
  Astrology: 4,380 changes/year (sub-hourly variations)
\item
  Numerology: 73 changes/year (only when date digit changes)
\end{itemize}

This creates an inherent barrier to synchronization.

\subsubsection{Cosine Similarity
Analysis}\label{cosine-similarity-analysis}

To quantify independence, we computed the cosine similarity between
normalized time series:

\[
\cos(\theta) = \frac{\vec{A} \cdot \vec{N}}{\|\vec{A}\| \|\vec{N}\|}
\]

\textbf{Result}: \(\cos(\theta) = 0.0052\), indicating \textbf{99.48\%
mathematical orthogonality}.

\subsubsection{Nakshatra-Day Mapping}\label{nakshatra-day-mapping}

We tested whether the ``Mulanka Planet'' of the day matches the
``Nakshatra Lord'' (ruler of the Moon's constellation):

\begin{longtable}[]{@{}ll@{}}
\caption{Nakshatra-Day Synchronization
Test}\label{tbl-nakshatra}\tabularnewline
\toprule\noalign{}
Metric & Value \\
\midrule\noalign{}
\endfirsthead
\toprule\noalign{}
Metric & Value \\
\midrule\noalign{}
\endhead
\bottomrule\noalign{}
\endlastfoot
Total Days Tested & 365 \\
Exact Matches & 41 \\
Match Rate & 11.23\% \\
Expected by Chance & 11.11\% (1/9) \\
\end{longtable}

\textbf{Conclusion}: The match rate is statistically indistinguishable
from random chance.

\subsection{Track 1 Findings}\label{track-1-findings}

\begin{enumerate}
\def\labelenumi{\arabic{enumi}.}
\tightlist
\item
  \textbf{Temporal Mismatch}: The systems operate on fundamentally
  different temporal grids
\item
  \textbf{Mathematical Orthogonality}: 99.48\% independence between
  signals
\item
  \textbf{No Synchronization}: Nakshatra-Day matching equals random
  chance
\end{enumerate}

\section{Track 2: Earthquake Prediction
Analysis}\label{track-2-earthquake-prediction-analysis}

\subsection{Research Question}\label{research-question-1}

\textbf{Do planetary configurations (Shadbala, aspects, yogas) predict
seismic activity in the India-Nepal tectonic zone?}

\subsection{Data Collection}\label{data-collection-1}

\begin{itemize}
\tightlist
\item
  \textbf{Source}: USGS Earthquake Hazards Program API (United States
  Geological Survey, 2024)
\item
  \textbf{Region}: India-Nepal Border (Lat 20°N-35°N, Lon 75°E-90°E)
\item
  \textbf{Period}: January 1, 2015 to January 1, 2024
\item
  \textbf{Threshold}: Magnitude ≥ 4.5
\item
  \textbf{Total Events}: 370 earthquakes
\end{itemize}

\subsection{Methodology}\label{methodology-1}

\subsubsection{Planetary Stress Index}\label{planetary-stress-index}

We defined an ``Astro-Fusion Stress Index'' based on:

\begin{enumerate}
\def\labelenumi{\arabic{enumi}.}
\tightlist
\item
  \textbf{Low Aggregate Shadbala}: Planetary instability
\item
  \textbf{Graha Yuddha}: Planetary wars (conjunctions within 1°)
\item
  \textbf{Malefic Aspects}: Saturn-Mars squares and oppositions
\end{enumerate}

\begin{figure}[H]

\caption{\label{fig-mars}Mars Retrograde Cycles and Daily Motion}

\centering{

\includegraphics[width=1\textwidth,height=\textheight]{../../reports/artifacts/fig_mars_variation.pdf}

}

\end{figure}%

\subsubsection{Statistical Framework}\label{statistical-framework-1}

Using Negative Binomial regression (Molchan, 1997) to account for
overdispersion:

\[
\ln(\mu_t) = \beta_0 + \beta_{trend}t + \beta_{season}\sin(\frac{2\pi t}{365}) + \beta_{astro}X_{astro}
\]

\subsection{Results}\label{results}

\subsubsection{Regression Coefficients}\label{regression-coefficients}

\begin{longtable}[]{@{}llll@{}}
\caption{Earthquake Prediction Regression
Results}\label{tbl-earthquake}\tabularnewline
\toprule\noalign{}
Variable & Coefficient & Std. Error & p-value \\
\midrule\noalign{}
\endfirsthead
\toprule\noalign{}
Variable & Coefficient & Std. Error & p-value \\
\midrule\noalign{}
\endhead
\bottomrule\noalign{}
\endlastfoot
Intercept & -0.900 & 0.388 & 0.02 \\
Year Index & 0.011 & 0.062 & 0.86 \\
Universal Day 8 & 0.143 & 0.213 & 0.50 \\
Mars Strength & -0.005 & 0.003 & 0.10 \\
Saturn Strength & 0.000 & 0.003 & 0.99 \\
\end{longtable}

\subsubsection{Monte Carlo Validation}\label{monte-carlo-validation}

The Monte Carlo ``Look-Elsewhere'' analysis (N=1,000) revealed:

\begin{itemize}
\tightlist
\item
  \textbf{95th Percentile Noise Floor}: 27.95
\item
  \textbf{Real Signal Delta-AIC}: 22.13
\item
  \textbf{Conclusion}: Signal falls within random noise distribution
\end{itemize}

\subsection{Track 2 Findings}\label{track-2-findings}

\begin{enumerate}
\def\labelenumi{\arabic{enumi}.}
\tightlist
\item
  \textbf{No Significant Predictors}: All planetary variables fail
  significance threshold
\item
  \textbf{Mars Near-Threshold}: \(p = 0.10\) suggests weak but
  inconclusive signal
\item
  \textbf{Monte Carlo Confirmation}: Results indistinguishable from
  chance
\end{enumerate}

\begin{figure}[H]

\caption{\label{fig-monte-carlo}Monte Carlo Permutation Distribution}

\centering{

\includegraphics[width=0.8\textwidth,height=\textheight]{../../reports/artifacts/fig_permutation_dist.pdf}

}

\end{figure}%

\section{Track 3: Gold Market
Correlation}\label{track-3-gold-market-correlation}

\subsection{Research Question}\label{research-question-2}

\textbf{Do planetary positions provide predictive information for
XAU/USD (Gold) spot prices, thus challenging the Efficient Market
Hypothesis?}

\subsection{Data Collection}\label{data-collection-2}

\begin{itemize}
\tightlist
\item
  \textbf{Source}: Yahoo Finance (Yahoo Finance, 2024)
\item
  \textbf{Asset}: XAU/USD (Gold Spot Price)
\item
  \textbf{Period}: January 1, 2000 to December 31, 2024
\item
  \textbf{Frequency}: Daily closing prices
\item
  \textbf{Planetary Data}: Swiss Ephemeris geocentric positions
\end{itemize}

\subsection{Methodology}\label{methodology-2}

\subsubsection{Data Preprocessing}\label{data-preprocessing}

Raw prices were transformed to log-returns for stationarity:

\[
R_t = \ln(P_t) - \ln(P_{t-1})
\]

Planetary positions were encoded as sine/cosine components:

\[
X_{planet} = \sin(\lambda_{planet}) + \cos(\lambda_{planet})
\]

\subsubsection{Spectral Analysis}\label{spectral-analysis}

Lomb-Scargle periodograms (Lomb, 1976) were computed to detect cyclic
signals at planetary synodic periods:

\begin{itemize}
\tightlist
\item
  Moon: \textasciitilde29.5 days
\item
  Mercury: \textasciitilde116 days
\item
  Venus: \textasciitilde584 days
\item
  Mars: \textasciitilde780 days
\end{itemize}

\begin{figure}[H]

\caption{\label{fig-periodogram}Lomb-Scargle Periodogram of Gold
log-returns. No significant peaks observed at planetary frequencies.}

\centering{

\includegraphics[width=1\textwidth,height=\textheight]{../../reports/artifacts/fig_periodogram.pdf}

}

\end{figure}%

\subsection{Results}\label{results-1}

\subsubsection{Stationarity Tests}\label{stationarity-tests}

\begin{longtable}[]{@{}llll@{}}
\caption{Stationarity Test
Results}\label{tbl-stationarity}\tabularnewline
\toprule\noalign{}
Variable & ADF Statistic & p-value & Stationary? \\
\midrule\noalign{}
\endfirsthead
\toprule\noalign{}
Variable & ADF Statistic & p-value & Stationary? \\
\midrule\noalign{}
\endhead
\bottomrule\noalign{}
\endlastfoot
Gold Log-Returns & -52.34 & \textless{} 0.001 & Yes \\
Sun Longitude & -3.21 & 0.018 & Yes \\
Moon Longitude & -4.89 & \textless{} 0.001 & Yes \\
Mars Speed & -8.72 & \textless{} 0.001 & Yes \\
\end{longtable}

\subsubsection{Granger Causality
Results}\label{granger-causality-results}

After Bonferroni correction (\(\alpha_{adj} = 0.05/9 = 0.0056\)):

\begin{longtable}[]{@{}lllll@{}}
\caption{Granger Causality Test
Results}\label{tbl-granger}\tabularnewline
\toprule\noalign{}
Planet & Best Lag & F-Statistic & p-value & Significant? \\
\midrule\noalign{}
\endfirsthead
\toprule\noalign{}
Planet & Best Lag & F-Statistic & p-value & Significant? \\
\midrule\noalign{}
\endhead
\bottomrule\noalign{}
\endlastfoot
Sun & 3 & 1.24 & 0.312 & No \\
Moon & 5 & 0.89 & 0.478 & No \\
Mars & 7 & 1.67 & 0.089 & No \\
Jupiter & 12 & 0.54 & 0.721 & No \\
Saturn & 14 & 1.12 & 0.334 & No \\
\end{longtable}

\subsubsection{Molchan Diagram Analysis}\label{molchan-diagram-analysis}

Testing Mars speed variations as a binary predictor for extreme
volatility events:

\begin{itemize}
\tightlist
\item
  \textbf{Result}: Trajectory hugs the diagonal (random guessing line)
\item
  \textbf{Conclusion}: No information gain over chance
\end{itemize}

\begin{figure}[H]

\caption{\label{fig-molchan}Molchan Diagram for Mars Speed vs Extreme
Volatility}

\centering{

\includegraphics[width=0.8\textwidth,height=\textheight]{../../reports/artifacts/fig_molchan.pdf}

}

\end{figure}%

\subsection{Track 3 Findings}\label{track-3-findings}

\begin{enumerate}
\def\labelenumi{\arabic{enumi}.}
\tightlist
\item
  \textbf{EMH Supported}: Planetary positions contain no unique
  predictive information
\item
  \textbf{No Spectral Peaks}: No significant cycles at known synodic
  periods
\item
  \textbf{Granger Null Maintained}: All planets fail to Granger-cause
  returns
\end{enumerate}

\section{Discussion}\label{discussion}

\subsection{Synthesis of Findings}\label{synthesis-of-findings}

Across all three research tracks, we observe a consistent pattern:
\textbf{Vedic predictive systems, when subjected to rigorous statistical
testing, fail to demonstrate empirical validity that exceeds random
chance.}

\begin{longtable}[]{@{}lll@{}}
\caption{Summary of Research Track
Findings}\label{tbl-summary}\tabularnewline
\toprule\noalign{}
Track & Domain & Primary Finding \\
\midrule\noalign{}
\endfirsthead
\toprule\noalign{}
Track & Domain & Primary Finding \\
\midrule\noalign{}
\endhead
\bottomrule\noalign{}
\endlastfoot
1 & Numerology-Astrology & 99.48\% mathematical independence \\
2 & Earthquake Prediction & No significant planetary predictors \\
3 & Gold Market & EMH supported; no Granger causality \\
\end{longtable}

\subsection{Theoretical Implications}\label{theoretical-implications}

\subsubsection{System Independence}\label{system-independence}

The near-zero correlation between numerological and astrological systems
suggests they should be treated as \textbf{orthogonal frameworks} rather
than redundant alternatives. They appear to measure different conceptual
dimensions:

\begin{itemize}
\tightlist
\item
  \textbf{Numerology}: Symbolic/archetypal rhythm tied to human calendar
  constructs
\item
  \textbf{Astrology}: Physical/astronomical rhythm tied to observable
  celestial mechanics
\end{itemize}

\subsubsection{The Role of Pattern
Recognition}\label{the-role-of-pattern-recognition}

Human cognition is predisposed to pattern recognition, often perceiving
meaningful signals in random noise---a phenomenon known as
\textbf{apophenia}. Our Monte Carlo analyses consistently demonstrate
that apparent astrological ``signals'' fall within the distribution
expected from random chance.

\subsection{Limitations}\label{limitations}

\begin{enumerate}
\def\labelenumi{\arabic{enumi}.}
\tightlist
\item
  \textbf{Geographic Specificity}: Earthquake analysis limited to
  India-Nepal region
\item
  \textbf{Temporal Scope}: 25 years for gold, 10 years for earthquakes
\item
  \textbf{Linear Methods}: Non-linear relationships not fully explored
\item
  \textbf{Technique Coverage}: Many advanced Jyotish techniques (e.g.,
  Ashtakavarga, Koorma Chakra) not tested
\end{enumerate}

\subsection{Future Research
Directions}\label{future-research-directions}

\begin{enumerate}
\def\labelenumi{\arabic{enumi}.}
\tightlist
\item
  \textbf{Extended Datasets}: Analysis with \textgreater100,000
  earthquake events globally
\item
  \textbf{Non-Linear Models}: Machine learning approaches (Random
  Forests, Neural Networks)
\item
  \textbf{Regional Hypotheses}: Testing Koorma Chakra (regional zodiac
  mapping) for earthquakes
\item
  \textbf{Alternative Ayanamsas}: Comparison of Lahiri, Raman, and
  Krishnamurti systems
\end{enumerate}

\section{Conclusions}\label{conclusions}

This consolidated research report represents the most comprehensive
computational investigation of Vedic predictive systems to date. Our
findings support the following conclusions:

\begin{enumerate}
\def\labelenumi{\arabic{enumi}.}
\item
  \textbf{Temporal Discontinuity Confirmed}: Vedic Astrology and
  Numerology operate on fundamentally incompatible temporal grids,
  rendering assumptions of synchronization unfounded.
\item
  \textbf{Earthquake Prediction Null}: Planetary configurations, as
  operationalized through Shadbala and aspects, do not provide
  statistically significant predictive power for seismic events in the
  tested region.
\item
  \textbf{Financial Astrology Falsified}: The Efficient Market
  Hypothesis is supported; planetary positions contain no unique
  information for gold price prediction.
\item
  \textbf{Methodological Contribution}: We provide a reproducible,
  open-source framework for testing similar claims, embodying the
  principles of Literate Programming (Knuth, 1984).
\item
  \textbf{Cultural Value Preserved}: The failure to demonstrate
  predictive validity does not diminish the cultural, philosophical, and
  historical significance of these systems. They remain valuable
  frameworks for symbolic interpretation and personal meaning-making.
\end{enumerate}

\subsection{Final Statement}\label{final-statement}

Science advances through the courageous application of skepticism to
cherished beliefs. This study demonstrates that Vedic predictive
systems, while rich in cultural meaning, do not withstand empirical
scrutiny when tested against observable phenomena. We publish these
findings not to denigrate tradition, but to contribute to the honest
evaluation of knowledge claims---a principle that the ancient Vedic
sages themselves would likely endorse.

\begin{center}\rule{0.5\linewidth}{0.5pt}\end{center}

\emph{This research was conducted using open-source tools and
reproducible methods. All code, data, and analysis pipelines are
available at:
https://github.com/astro-fusion/astro\_research-white-paper}

\section{References}\label{references}

\phantomsection\label{refs}
\begin{CSLReferences}{1}{0}
\bibitem[\citeproctext]{ref-swiss_ephemeris}
AG, A. (2023). \emph{Swiss ephemeris: High precision planetary ephemeris
DE440}. \url{https://www.astro.com/swisseph/}

\bibitem[\citeproctext]{ref-surya_siddhanta}
Anonymous. (c. 4th century CE). \emph{Surya siddhanta: A textbook of
hindu astronomy}. Various translations.

\bibitem[\citeproctext]{ref-dickey_fuller}
Dickey, D. A., \& Fuller, W. A. (1979). Distribution of the estimators
for autoregressive time series with a unit root. \emph{Journal of the
American Statistical Association}, \emph{74}(366), 427--431.

\bibitem[\citeproctext]{ref-monte_carlo_methods}
Fishman, G. S. (1996). \emph{Monte carlo: Concepts, algorithms, and
applications}. Springer.

\bibitem[\citeproctext]{ref-granger_causality}
Granger, C. W. J. (1969). Investigating causal relations by econometric
models and cross-spectral methods. \emph{Econometrica}, \emph{37}(3),
424--438. \url{https://doi.org/10.2307/1912791}

\bibitem[\citeproctext]{ref-knuth_literate}
Knuth, D. E. (1984). Literate programming. In \emph{The Computer
Journal} (Vol. 27, pp. 97--111).
\url{https://doi.org/10.1093/comjnl/27.2.97}

\bibitem[\citeproctext]{ref-lahiri_ayanamsa}
Lahiri, N. C. (1960). Ayanamsa for sidereal ephemeris. \emph{Indian
Astronomical Ephemeris}.

\bibitem[\citeproctext]{ref-lomb_scargle}
Lomb, N. R. (1976). Least-squares frequency analysis of unequally spaced
data. \emph{Astrophysics and Space Science}, \emph{39}, 447--462.
\url{https://doi.org/10.1007/BF00648343}

\bibitem[\citeproctext]{ref-mayo_severe_testing}
Mayo, D. G., \& Spanos, A. (2006). Severe testing as a basic concept in
a neyman--pearson philosophy of induction. \emph{The British Journal for
the Philosophy of Science}, \emph{57}(2), 323--357.
\url{https://doi.org/10.1093/bjps/axl003}

\bibitem[\citeproctext]{ref-molchan_diagram}
Molchan, G. M. (1997). Earthquake prediction as a decision-making
problem. \emph{Pure and Applied Geophysics}, \emph{149}, 233--247.
\url{https://doi.org/10.1007/BF00945169}

\bibitem[\citeproctext]{ref-bphs}
Parashara, S. (c. 7th century CE). \emph{Brihat parashara hora shastra}.
Various translations.

\bibitem[\citeproctext]{ref-popper_falsification}
Popper, K. (1959). \emph{The logic of scientific discovery}. Routledge.

\bibitem[\citeproctext]{ref-usgs_earthquake}
United States Geological Survey. (2024). \emph{USGS earthquake hazards
program: Comprehensive earthquake catalog}.
\url{https://earthquake.usgs.gov/earthquakes/search/}

\bibitem[\citeproctext]{ref-varahamihira_brihat_samhita}
Varahamihira. (c. 6th century CE). \emph{Brihat samhita}.

\bibitem[\citeproctext]{ref-yahoo_finance}
Yahoo Finance. (2024). \emph{XAU/USD historical data}.
\url{https://finance.yahoo.com/}

\end{CSLReferences}

\section{Appendix A: Software
Dependencies}\label{appendix-a-software-dependencies}

\begin{longtable}[]{@{}lll@{}}
\caption{Software Dependencies}\label{tbl-software}\tabularnewline
\toprule\noalign{}
Package & Version & Purpose \\
\midrule\noalign{}
\endfirsthead
\toprule\noalign{}
Package & Version & Purpose \\
\midrule\noalign{}
\endhead
\bottomrule\noalign{}
\endlastfoot
pyswisseph & 2.10+ & Swiss Ephemeris bindings \\
pandas & 2.0+ & Data manipulation \\
numpy & 1.24+ & Numerical computing \\
scipy & 1.11+ & Statistical tests \\
matplotlib & 3.7+ & Visualization \\
statsmodels & 0.14+ & Econometric models \\
quarto & 1.4+ & Document rendering \\
\end{longtable}

\section{Appendix B: Reproducibility
Statement}\label{appendix-b-reproducibility-statement}

All analyses can be reproduced by executing:

\begin{Shaded}
\begin{Highlighting}[]
\FunctionTok{git}\NormalTok{ clone https://github.com/astro{-}fusion/astro\_research{-}white{-}paper.git}
\BuiltInTok{cd}\NormalTok{ astro\_research{-}white{-}paper}
\ExtensionTok{pip}\NormalTok{ install }\AttributeTok{{-}e} \StringTok{".[all]"}
\ExtensionTok{python}\NormalTok{ src/generate\_artifacts.py}
\ExtensionTok{quarto}\NormalTok{ render docs/research/VEDIC\_SYSTEMS\_EMPIRICAL\_ANALYSIS.qmd }\AttributeTok{{-}{-}to}\NormalTok{ pdf}
\end{Highlighting}
\end{Shaded}




\end{document}
