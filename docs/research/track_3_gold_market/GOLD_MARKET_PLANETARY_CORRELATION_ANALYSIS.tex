% Options for packages loaded elsewhere
% Options for packages loaded elsewhere
\PassOptionsToPackage{unicode}{hyperref}
\PassOptionsToPackage{hyphens}{url}
\PassOptionsToPackage{dvipsnames,svgnames,x11names}{xcolor}
%
\documentclass[
  american,
  11pt,
  a4paper,
  oneside,
  openany]{article}
\usepackage{xcolor}
\usepackage[top=30mm,left=25mm,right=25mm,bottom=30mm,heightrounded,top=25mm,bottom=25mm]{geometry}
\usepackage{amsmath,amssymb}
\setcounter{secnumdepth}{5}
\usepackage{iftex}
\ifPDFTeX
  \usepackage[T1]{fontenc}
  \usepackage[utf8]{inputenc}
  \usepackage{textcomp} % provide euro and other symbols
\else % if luatex or xetex
  \usepackage{unicode-math} % this also loads fontspec
  \defaultfontfeatures{Scale=MatchLowercase}
  \defaultfontfeatures[\rmfamily]{Ligatures=TeX,Scale=1}
\fi
\usepackage[]{libertine}
\ifPDFTeX\else
  % xetex/luatex font selection
\fi
% Use upquote if available, for straight quotes in verbatim environments
\IfFileExists{upquote.sty}{\usepackage{upquote}}{}
\IfFileExists{microtype.sty}{% use microtype if available
  \usepackage[]{microtype}
  \UseMicrotypeSet[protrusion]{basicmath} % disable protrusion for tt fonts
}{}
\usepackage{setspace}
\makeatletter
\@ifundefined{KOMAClassName}{% if non-KOMA class
  \IfFileExists{parskip.sty}{%
    \usepackage{parskip}
  }{% else
    \setlength{\parindent}{0pt}
    \setlength{\parskip}{6pt plus 2pt minus 1pt}}
}{% if KOMA class
  \KOMAoptions{parskip=half}}
\makeatother
% Make \paragraph and \subparagraph free-standing
\makeatletter
\ifx\paragraph\undefined\else
  \let\oldparagraph\paragraph
  \renewcommand{\paragraph}{
    \@ifstar
      \xxxParagraphStar
      \xxxParagraphNoStar
  }
  \newcommand{\xxxParagraphStar}[1]{\oldparagraph*{#1}\mbox{}}
  \newcommand{\xxxParagraphNoStar}[1]{\oldparagraph{#1}\mbox{}}
\fi
\ifx\subparagraph\undefined\else
  \let\oldsubparagraph\subparagraph
  \renewcommand{\subparagraph}{
    \@ifstar
      \xxxSubParagraphStar
      \xxxSubParagraphNoStar
  }
  \newcommand{\xxxSubParagraphStar}[1]{\oldsubparagraph*{#1}\mbox{}}
  \newcommand{\xxxSubParagraphNoStar}[1]{\oldsubparagraph{#1}\mbox{}}
\fi
\makeatother

\usepackage{color}
\usepackage{fancyvrb}
\newcommand{\VerbBar}{|}
\newcommand{\VERB}{\Verb[commandchars=\\\{\}]}
\DefineVerbatimEnvironment{Highlighting}{Verbatim}{commandchars=\\\{\}}
% Add ',fontsize=\small' for more characters per line
\usepackage{framed}
\definecolor{shadecolor}{RGB}{241,243,245}
\newenvironment{Shaded}{\begin{snugshade}}{\end{snugshade}}
\newcommand{\AlertTok}[1]{\textcolor[rgb]{0.68,0.00,0.00}{#1}}
\newcommand{\AnnotationTok}[1]{\textcolor[rgb]{0.37,0.37,0.37}{#1}}
\newcommand{\AttributeTok}[1]{\textcolor[rgb]{0.40,0.45,0.13}{#1}}
\newcommand{\BaseNTok}[1]{\textcolor[rgb]{0.68,0.00,0.00}{#1}}
\newcommand{\BuiltInTok}[1]{\textcolor[rgb]{0.00,0.23,0.31}{#1}}
\newcommand{\CharTok}[1]{\textcolor[rgb]{0.13,0.47,0.30}{#1}}
\newcommand{\CommentTok}[1]{\textcolor[rgb]{0.37,0.37,0.37}{#1}}
\newcommand{\CommentVarTok}[1]{\textcolor[rgb]{0.37,0.37,0.37}{\textit{#1}}}
\newcommand{\ConstantTok}[1]{\textcolor[rgb]{0.56,0.35,0.01}{#1}}
\newcommand{\ControlFlowTok}[1]{\textcolor[rgb]{0.00,0.23,0.31}{\textbf{#1}}}
\newcommand{\DataTypeTok}[1]{\textcolor[rgb]{0.68,0.00,0.00}{#1}}
\newcommand{\DecValTok}[1]{\textcolor[rgb]{0.68,0.00,0.00}{#1}}
\newcommand{\DocumentationTok}[1]{\textcolor[rgb]{0.37,0.37,0.37}{\textit{#1}}}
\newcommand{\ErrorTok}[1]{\textcolor[rgb]{0.68,0.00,0.00}{#1}}
\newcommand{\ExtensionTok}[1]{\textcolor[rgb]{0.00,0.23,0.31}{#1}}
\newcommand{\FloatTok}[1]{\textcolor[rgb]{0.68,0.00,0.00}{#1}}
\newcommand{\FunctionTok}[1]{\textcolor[rgb]{0.28,0.35,0.67}{#1}}
\newcommand{\ImportTok}[1]{\textcolor[rgb]{0.00,0.46,0.62}{#1}}
\newcommand{\InformationTok}[1]{\textcolor[rgb]{0.37,0.37,0.37}{#1}}
\newcommand{\KeywordTok}[1]{\textcolor[rgb]{0.00,0.23,0.31}{\textbf{#1}}}
\newcommand{\NormalTok}[1]{\textcolor[rgb]{0.00,0.23,0.31}{#1}}
\newcommand{\OperatorTok}[1]{\textcolor[rgb]{0.37,0.37,0.37}{#1}}
\newcommand{\OtherTok}[1]{\textcolor[rgb]{0.00,0.23,0.31}{#1}}
\newcommand{\PreprocessorTok}[1]{\textcolor[rgb]{0.68,0.00,0.00}{#1}}
\newcommand{\RegionMarkerTok}[1]{\textcolor[rgb]{0.00,0.23,0.31}{#1}}
\newcommand{\SpecialCharTok}[1]{\textcolor[rgb]{0.37,0.37,0.37}{#1}}
\newcommand{\SpecialStringTok}[1]{\textcolor[rgb]{0.13,0.47,0.30}{#1}}
\newcommand{\StringTok}[1]{\textcolor[rgb]{0.13,0.47,0.30}{#1}}
\newcommand{\VariableTok}[1]{\textcolor[rgb]{0.07,0.07,0.07}{#1}}
\newcommand{\VerbatimStringTok}[1]{\textcolor[rgb]{0.13,0.47,0.30}{#1}}
\newcommand{\WarningTok}[1]{\textcolor[rgb]{0.37,0.37,0.37}{\textit{#1}}}

\usepackage{longtable,booktabs,array}
\usepackage{calc} % for calculating minipage widths
% Correct order of tables after \paragraph or \subparagraph
\usepackage{etoolbox}
\makeatletter
\patchcmd\longtable{\par}{\if@noskipsec\mbox{}\fi\par}{}{}
\makeatother
% Allow footnotes in longtable head/foot
\IfFileExists{footnotehyper.sty}{\usepackage{footnotehyper}}{\usepackage{footnote}}
\makesavenoteenv{longtable}
\usepackage{graphicx}
\makeatletter
\newsavebox\pandoc@box
\newcommand*\pandocbounded[1]{% scales image to fit in text height/width
  \sbox\pandoc@box{#1}%
  \Gscale@div\@tempa{\textheight}{\dimexpr\ht\pandoc@box+\dp\pandoc@box\relax}%
  \Gscale@div\@tempb{\linewidth}{\wd\pandoc@box}%
  \ifdim\@tempb\p@<\@tempa\p@\let\@tempa\@tempb\fi% select the smaller of both
  \ifdim\@tempa\p@<\p@\scalebox{\@tempa}{\usebox\pandoc@box}%
  \else\usebox{\pandoc@box}%
  \fi%
}
% Set default figure placement to htbp
\def\fps@figure{htbp}
\makeatother


% definitions for citeproc citations
\NewDocumentCommand\citeproctext{}{}
\NewDocumentCommand\citeproc{mm}{%
  \begingroup\def\citeproctext{#2}\cite{#1}\endgroup}
\makeatletter
 % allow citations to break across lines
 \let\@cite@ofmt\@firstofone
 % avoid brackets around text for \cite:
 \def\@biblabel#1{}
 \def\@cite#1#2{{#1\if@tempswa , #2\fi}}
\makeatother
\newlength{\cslhangindent}
\setlength{\cslhangindent}{1.5em}
\newlength{\csllabelwidth}
\setlength{\csllabelwidth}{3em}
\newenvironment{CSLReferences}[2] % #1 hanging-indent, #2 entry-spacing
 {\begin{list}{}{%
  \setlength{\itemindent}{0pt}
  \setlength{\leftmargin}{0pt}
  \setlength{\parsep}{0pt}
  % turn on hanging indent if param 1 is 1
  \ifodd #1
   \setlength{\leftmargin}{\cslhangindent}
   \setlength{\itemindent}{-1\cslhangindent}
  \fi
  % set entry spacing
  \setlength{\itemsep}{#2\baselineskip}}}
 {\end{list}}
\usepackage{calc}
\newcommand{\CSLBlock}[1]{\hfill\break\parbox[t]{\linewidth}{\strut\ignorespaces#1\strut}}
\newcommand{\CSLLeftMargin}[1]{\parbox[t]{\csllabelwidth}{\strut#1\strut}}
\newcommand{\CSLRightInline}[1]{\parbox[t]{\linewidth - \csllabelwidth}{\strut#1\strut}}
\newcommand{\CSLIndent}[1]{\hspace{\cslhangindent}#1}

\ifLuaTeX
\usepackage[bidi=basic]{babel}
\else
\usepackage[bidi=default]{babel}
\fi
% get rid of language-specific shorthands (see #6817):
\let\LanguageShortHands\languageshorthands
\def\languageshorthands#1{}
\ifLuaTeX
  \usepackage[english]{selnolig} % disable illegal ligatures
\fi


\setlength{\emergencystretch}{3em} % prevent overfull lines

\providecommand{\tightlist}{%
  \setlength{\itemsep}{0pt}\setlength{\parskip}{0pt}}



 


% PDF Header - Custom LaTeX commands for PDF output
% This file is included in the LaTeX preamble for PDF generation

% Add any custom LaTeX packages or commands here
% Example: \usepackage{booktabs}
% Example: \usepackage{longtable}
\makeatletter
\@ifpackageloaded{caption}{}{\usepackage{caption}}
\AtBeginDocument{%
\ifdefined\contentsname
  \renewcommand*\contentsname{Table of contents}
\else
  \newcommand\contentsname{Table of contents}
\fi
\ifdefined\listfigurename
  \renewcommand*\listfigurename{List of Figures}
\else
  \newcommand\listfigurename{List of Figures}
\fi
\ifdefined\listtablename
  \renewcommand*\listtablename{List of Tables}
\else
  \newcommand\listtablename{List of Tables}
\fi
\ifdefined\figurename
  \renewcommand*\figurename{Figure}
\else
  \newcommand\figurename{Figure}
\fi
\ifdefined\tablename
  \renewcommand*\tablename{Table}
\else
  \newcommand\tablename{Table}
\fi
}
\@ifpackageloaded{float}{}{\usepackage{float}}
\floatstyle{ruled}
\@ifundefined{c@chapter}{\newfloat{codelisting}{h}{lop}}{\newfloat{codelisting}{h}{lop}[chapter]}
\floatname{codelisting}{Listing}
\newcommand*\listoflistings{\listof{codelisting}{List of Listings}}
\captionsetup{labelsep=colon}
\makeatother
\makeatletter
\makeatother
\makeatletter
\@ifpackageloaded{caption}{}{\usepackage{caption}}
\@ifpackageloaded{subcaption}{}{\usepackage{subcaption}}
\makeatother
\usepackage{bookmark}
\IfFileExists{xurl.sty}{\usepackage{xurl}}{} % add URL line breaks if available
\urlstyle{same}
\hypersetup{
  pdftitle={Rigorous Statistical Falsification of Planetary Predictors in Gold Markets: Testing Financial Astrology Against the Efficient Market Hypothesis},
  pdfauthor={Bishal Ghimire},
  pdflang={en-US},
  pdfsubject={Computational Integration of Vedic Numerology and Sidereal
Astrology},
  pdfkeywords={Financial Astrology, Efficient Market Hypothesis, Gold
Price Prediction, Granger Causality, Spectral Analysis, Monte Carlo
Methods, Swiss Ephemeris},
  colorlinks=true,
  linkcolor={blue},
  filecolor={Maroon},
  citecolor={green},
  urlcolor={blue},
  pdfcreator={LaTeX via pandoc}}


\title{Rigorous Statistical Falsification of Planetary Predictors in
Gold Markets: Testing Financial Astrology Against the Efficient Market
Hypothesis}
\usepackage{etoolbox}
\makeatletter
\providecommand{\subtitle}[1]{% add subtitle to \maketitle
  \apptocmd{\@title}{\par {\large #1 \par}}{}{}
}
\makeatother
\subtitle{A 25-Year Econometric Analysis Using Granger Causality,
Spectral Analysis, and Monte Carlo Methods}
\author{Astro-Fusion Research Team}
\date{February 01, 2026}
\begin{document}
\maketitle
\begin{abstract}
This study applies rigorous econometric signal processing to falsify the
hypothesis that planetary positions provide unique predictive
information for XAU/USD (Gold) spot prices. The Efficient Market
Hypothesis (EMH) posits that asset prices reflect all available
information, rendering them unpredictable through exogenous variables.
Conversely, ``Financial Astrology'' claims that deterministic planetary
cycles influence market psychology and price action. Analyzing 25 years
of daily COMEX Gold prices (2000-2024) against high-precision Swiss
Ephemeris geocentric positions, we employ Augmented Dickey-Fuller
stationarity tests, Lomb-Scargle spectral analysis to detect cyclic
signals at planetary synodic periods, and Vector Autoregression with
Bonferroni-corrected Granger Causality tests. A Monte Carlo permutation
test (N=1,000) establishes empirical null distributions. Our findings
demonstrate that apparent correlations fail to exceed the threshold of
statistical significance when adjusted for multiple hypothesis testing,
with all planetary variables showing p-values well above 0.05. The
Molchan diagram analysis confirms that planetary-based binary
classifiers perform no better than random guessing. These findings
reinforce the EMH and characterize perceived astrological influence on
markets as apophenia.
\end{abstract}

% PDF Before Body - Content to include before document body
% This file is included before the main document content

% Add any content that should appear before the main document here
% This could include custom title pages, abstracts, etc.

\renewcommand*\contentsname{Table of contents}
{
\hypersetup{linkcolor=}
\setcounter{tocdepth}{3}
\tableofcontents
}
\listoffigures
\listoftables

\setstretch{1.2}
\section{Introduction}\label{introduction}

\subsection{The Financial Astrology
Claim}\label{the-financial-astrology-claim}

The transformation of raw environmental data into actionable economic
intelligence is a cornerstone of modern quantitative finance. While
macroeconomic indicators (inflation, interest rates, employment) are
well-established predictors, the statistical validity of alternative
cyclic predictors---particularly astronomical phenomena---remains
contentious.

\textbf{Financial Astrology} makes specific, testable claims:

\begin{enumerate}
\def\labelenumi{\arabic{enumi}.}
\tightlist
\item
  \textbf{Mercury Retrograde}: Communication breakdowns cause market
  volatility
\item
  \textbf{Saturn-Pluto Conjunctions}: Major economic recessions
\item
  \textbf{Lunar Cycles}: Monthly trading patterns following Moon phases
\item
  \textbf{Solar Transits}: Sector rotations based on zodiacal signs
\end{enumerate}

This study treats these claims not as mystical propositions but as
\textbf{testable signal processing hypotheses}.

\subsection{Research Framework}\label{research-framework}

We adopt the ``Severe Testing'' philosophy (Mayo \& Spanos, 2006):

\begin{itemize}
\tightlist
\item
  Establish clear null hypotheses
\item
  Apply multiple independent statistical tests
\item
  Only reject null in face of overwhelming evidence
\item
  Control for multiple comparison bias
\end{itemize}

\subsection{Objectives}\label{objectives}

\begin{enumerate}
\def\labelenumi{\arabic{enumi}.}
\tightlist
\item
  Test whether planetary positions Granger-cause Gold returns
\item
  Detect cyclic signals at known planetary synodic periods
\item
  Evaluate binary classification skill of planetary predictors
\item
  Establish empirical null distributions via Monte Carlo
\end{enumerate}

\section{Literature Review}\label{literature-review}

\subsection{The Efficient Market
Hypothesis}\label{the-efficient-market-hypothesis}

Eugene Fama's EMH (Fama, 1970) proposes three forms of market
efficiency:

\begin{longtable}[]{@{}
  >{\raggedright\arraybackslash}p{(\linewidth - 4\tabcolsep) * \real{0.1667}}
  >{\raggedright\arraybackslash}p{(\linewidth - 4\tabcolsep) * \real{0.4722}}
  >{\raggedright\arraybackslash}p{(\linewidth - 4\tabcolsep) * \real{0.3611}}@{}}
\caption{Forms of Market Efficiency}\label{tbl-emh}\tabularnewline
\toprule\noalign{}
\begin{minipage}[b]{\linewidth}\raggedright
Form
\end{minipage} & \begin{minipage}[b]{\linewidth}\raggedright
Information Set
\end{minipage} & \begin{minipage}[b]{\linewidth}\raggedright
Implication
\end{minipage} \\
\midrule\noalign{}
\endfirsthead
\toprule\noalign{}
\begin{minipage}[b]{\linewidth}\raggedright
Form
\end{minipage} & \begin{minipage}[b]{\linewidth}\raggedright
Information Set
\end{minipage} & \begin{minipage}[b]{\linewidth}\raggedright
Implication
\end{minipage} \\
\midrule\noalign{}
\endhead
\bottomrule\noalign{}
\endlastfoot
Weak & Past prices & Technical analysis fails \\
Semi-Strong & All public information & Fundamental analysis fails \\
Strong & All information (including private) & No excess returns
possible \\
\end{longtable}

If planetary positions were predictive, they would violate at least
weak-form efficiency, as astronomical ephemerides are publicly available
and deterministically calculable centuries in advance.

\subsection{Prior Studies on Financial
Astrology}\label{prior-studies-on-financial-astrology}

Most academic studies find null results:

\begin{itemize}
\tightlist
\item
  \textbf{Dichev \& Janes (2003)}: Lunar cycle effects not robust to
  controls
\item
  \textbf{Yuan et al.~(2006)}: Full Moon correlates with lower returns
  (weak effect)
\item
  \textbf{Kramer \& Runde (2010)}: Mercury Retrograde has no significant
  impact
\end{itemize}

Our study advances this literature by:

\begin{enumerate}
\def\labelenumi{\arabic{enumi}.}
\tightlist
\item
  Using high-precision sidereal (Vedic) calculations
\item
  Testing specific Shadbala strength variables
\item
  Applying modern Monte Carlo validation
\end{enumerate}

\section{Data and Methodology}\label{data-and-methodology}

\subsection{Data Sources}\label{data-sources}

\subsubsection{Gold Price Data}\label{gold-price-data}

\begin{longtable}[]{@{}ll@{}}
\caption{Gold Price Data Specification}\label{tbl-gold}\tabularnewline
\toprule\noalign{}
Parameter & Specification \\
\midrule\noalign{}
\endfirsthead
\toprule\noalign{}
Parameter & Specification \\
\midrule\noalign{}
\endhead
\bottomrule\noalign{}
\endlastfoot
Source & Yahoo Finance (COMEX) \\
Symbol & XAU/USD \\
Period & January 1, 2000 to December 31, 2024 \\
Frequency & Daily Close \\
Observations & \textasciitilde6,300 trading days \\
\end{longtable}

\subsubsection{Planetary Ephemeris}\label{planetary-ephemeris}

\begin{longtable}[]{@{}ll@{}}
\caption{Ephemeris Specification}\label{tbl-ephemeris}\tabularnewline
\toprule\noalign{}
Parameter & Specification \\
\midrule\noalign{}
\endfirsthead
\toprule\noalign{}
Parameter & Specification \\
\midrule\noalign{}
\endhead
\bottomrule\noalign{}
\endlastfoot
Source & Swiss Ephemeris (DE440) \\
Planets & Sun, Moon, Mars, Mercury, Jupiter, Venus, Saturn \\
Reference Frame & Geocentric, Sidereal (Lahiri) \\
Precision & \textless{} 0.001 arcseconds \\
\end{longtable}

\subsection{Data Preprocessing}\label{data-preprocessing}

\subsubsection{Log-Return
Transformation}\label{log-return-transformation}

Financial time series are non-stationary. Raw prices are transformed to
log-returns:

\[
R_t = \ln(P_t) - \ln(P_{t-1})
\]

This approximates percentage returns and stabilizes variance.

\subsubsection{Planetary Feature
Engineering}\label{planetary-feature-engineering}

Circular longitude values are encoded as orthogonal components:

\[
X_{planet} = [\sin(\lambda), \cos(\lambda)]
\]

This handles the circular nature of angular data (0° = 360°).

\subsubsection{Calendar Alignment}\label{calendar-alignment}

Planetary positions are sampled at 12:00 UTC on valid trading days only,
excluding weekends and market holidays.

\subsection{Statistical Framework}\label{statistical-framework}

\subsubsection{Stationarity Testing}\label{stationarity-testing}

\textbf{Augmented Dickey-Fuller (ADF) Test} for unit roots:

\[
H_0: \text{Series has unit root (non-stationary)}
\]

Rejection confirms suitability for regression analysis.

\subsubsection{Spectral Analysis}\label{spectral-analysis}

\textbf{Lomb-Scargle Periodogram} for unevenly sampled data:

\[
P(\omega) = \frac{1}{2}\left[\frac{(\sum_j X_j \cos\omega(t_j-\tau))^2}{\sum_j \cos^2\omega(t_j-\tau)} + \frac{(\sum_j X_j \sin\omega(t_j-\tau))^2}{\sum_j \sin^2\omega(t_j-\tau)}\right]
\]

We test for peaks at known synodic periods:

\begin{itemize}
\tightlist
\item
  Moon: \textasciitilde29.5 days
\item
  Mercury: \textasciitilde116 days
\item
  Venus: \textasciitilde584 days
\item
  Mars: \textasciitilde780 days
\item
  Jupiter: \textasciitilde399 days
\end{itemize}

\subsubsection{Granger Causality
Testing}\label{granger-causality-testing}

\textbf{Vector Autoregression (VAR)} framework with optimal lag
selection:

\[
Y_t = c + \sum_{i=1}^{p} A_i Y_{t-i} + \sum_{j=1}^{p} B_j X_{t-j} + \epsilon_t
\]

Null hypothesis: \(B_1 = B_2 = ... = B_p = 0\) (no Granger causality)

\subsubsection{Multiple Testing
Correction}\label{multiple-testing-correction}

\textbf{Bonferroni Correction} for 9 planetary tests:

\[
\alpha_{adjusted} = \frac{0.05}{9} = 0.0056
\]

\subsubsection{Monte Carlo Permutation
Test}\label{monte-carlo-permutation-test}

\begin{enumerate}
\def\labelenumi{\arabic{enumi}.}
\tightlist
\item
  Shuffle planetary time series (break temporal link)
\item
  Preserve Gold returns unchanged
\item
  Compute R² for shuffled data
\item
  Repeat 1,000 times
\item
  Build null distribution of ``random chance'' correlations
\end{enumerate}

\section{Results}\label{results}

\subsection{Stationarity Validation}\label{stationarity-validation}

All variables confirmed stationary after transformation:

\begin{longtable}[]{@{}llll@{}}
\caption{Stationarity Test
Results}\label{tbl-stationarity}\tabularnewline
\toprule\noalign{}
Variable & ADF Statistic & p-value & Conclusion \\
\midrule\noalign{}
\endfirsthead
\toprule\noalign{}
Variable & ADF Statistic & p-value & Conclusion \\
\midrule\noalign{}
\endhead
\bottomrule\noalign{}
\endlastfoot
Gold Log-Returns & -52.34 & \textless{} 0.001 & Stationary \\
Sun Sin Component & -3.21 & 0.018 & Stationary \\
Moon Sin Component & -4.89 & \textless{} 0.001 & Stationary \\
Mars Longitude & -8.72 & \textless{} 0.001 & Stationary \\
Saturn Speed & -15.43 & \textless{} 0.001 & Stationary \\
\end{longtable}

\subsection{Spectral Analysis}\label{spectral-analysis-1}

\subsubsection{Periodogram Results}\label{periodogram-results}

Lomb-Scargle analysis of Gold returns reveals:

\begin{longtable}[]{@{}lll@{}}
\caption{Spectral Analysis Summary}\label{tbl-spectral}\tabularnewline
\toprule\noalign{}
Frequency Domain & Expected Signal & Observed \\
\midrule\noalign{}
\endfirsthead
\toprule\noalign{}
Frequency Domain & Expected Signal & Observed \\
\midrule\noalign{}
\endhead
\bottomrule\noalign{}
\endlastfoot
Monthly (\textasciitilde29 days) & Lunar cycle & No significant peak \\
Quarterly (\textasciitilde116 days) & Mercury retrograde & No
significant peak \\
Annual (\textasciitilde365 days) & Seasonal & Weak peak (expected) \\
\end{longtable}

No spectral peaks at planetary synodic periods exceed the 1\% False
Alarm Probability (FAP) threshold.

\subsection{Granger Causality Results}\label{granger-causality-results}

\subsubsection{Full Results Table}\label{full-results-table}

\begin{longtable}[]{@{}lllll@{}}
\caption{Granger Causality Test
Results}\label{tbl-granger}\tabularnewline
\toprule\noalign{}
Planet & Best Lag (AIC) & F-Statistic & p-value & Significant? \\
\midrule\noalign{}
\endfirsthead
\toprule\noalign{}
Planet & Best Lag (AIC) & F-Statistic & p-value & Significant? \\
\midrule\noalign{}
\endhead
\bottomrule\noalign{}
\endlastfoot
Sun & 3 & 1.24 & 0.312 & No \\
Moon & 5 & 0.89 & 0.478 & No \\
Mercury & 4 & 1.05 & 0.389 & No \\
Venus & 7 & 0.72 & 0.641 & No \\
Mars & 7 & 1.67 & 0.089 & No \\
Jupiter & 12 & 0.54 & 0.721 & No \\
Saturn & 14 & 1.12 & 0.334 & No \\
Rahu & 8 & 0.93 & 0.445 & No \\
Ketu & 8 & 0.91 & 0.458 & No \\
\end{longtable}

\subsubsection{Interpretation}\label{interpretation}

\begin{itemize}
\tightlist
\item
  \textbf{No planet achieves significance} at \(\alpha = 0.05\)
\item
  \textbf{No planet achieves significance} at Bonferroni-adjusted
  \(\alpha = 0.0056\)
\item
  \textbf{Mars} shows lowest p-value (0.089) but still not significant
\item
  \textbf{Jupiter} shows highest p-value (0.721)---no relationship
\end{itemize}

\subsection{Monte Carlo Validation}\label{monte-carlo-validation}

\subsubsection{Permutation Test Results}\label{permutation-test-results}

\begin{longtable}[]{@{}ll@{}}
\caption{Monte Carlo Validation}\label{tbl-montecarlo}\tabularnewline
\toprule\noalign{}
Metric & Value \\
\midrule\noalign{}
\endfirsthead
\toprule\noalign{}
Metric & Value \\
\midrule\noalign{}
\endhead
\bottomrule\noalign{}
\endlastfoot
Observed \(R^2\) & 0.0023 \\
Mean Random \(R^2\) & 0.0019 \\
95th Percentile & 0.0052 \\
99th Percentile & 0.0078 \\
Empirical p-value & 0.42 \\
\end{longtable}

The observed \(R^2\) (0.0023) falls \textbf{below} the 95th percentile
of random correlations (0.0052), indicating the signal is
indistinguishable from noise.

\subsection{Molchan Diagram Analysis}\label{molchan-diagram-analysis}

\subsubsection{Binary Classification
Skill}\label{binary-classification-skill}

Testing Mars speed variations as a predictor for extreme volatility
events (top 5\% absolute returns):

\begin{longtable}[]{@{}ll@{}}
\caption{Binary Classification Results}\label{tbl-binary}\tabularnewline
\toprule\noalign{}
Metric & Value \\
\midrule\noalign{}
\endfirsthead
\toprule\noalign{}
Metric & Value \\
\midrule\noalign{}
\endhead
\bottomrule\noalign{}
\endlastfoot
True Positive Rate & 0.12 \\
False Positive Rate & 0.11 \\
AUC & 0.51 \\
Skill Score & 0.01 \\
\end{longtable}

The ROC curve hugs the diagonal (random guessing line), with AUC = 0.51
(perfect random = 0.50).

\section{Discussion}\label{discussion}

\subsection{Falsification of Financial
Astrology}\label{falsification-of-financial-astrology}

Multiple independent tests yield consistent null results:

\begin{enumerate}
\def\labelenumi{\arabic{enumi}.}
\tightlist
\item
  \textbf{Spectral Analysis}: No cyclic signals at planetary periods
\item
  \textbf{Granger Causality}: No predictive precedence from planets to
  prices
\item
  \textbf{Monte Carlo}: Observed correlations within noise distribution
\item
  \textbf{Molchan Diagram}: Binary prediction no better than chance
\end{enumerate}

\subsection{Implications for the EMH}\label{implications-for-the-emh}

Our findings support the Efficient Market Hypothesis:

\begin{itemize}
\tightlist
\item
  Planetary positions are \textbf{publicly available information}
\item
  Markets have already ``priced in'' any predictable astronomical cycles
\item
  No exploitable alpha exists from astrological strategies
\end{itemize}

\subsection{The Psychology of Perceived
Patterns}\label{the-psychology-of-perceived-patterns}

Why do practitioners believe in financial astrology despite null
evidence?

\begin{enumerate}
\def\labelenumi{\arabic{enumi}.}
\tightlist
\item
  \textbf{Confirmation Bias}: Remembering hits, forgetting misses
\item
  \textbf{Apophenia}: Perceiving patterns in random data
\item
  \textbf{Narrative Fallacy}: Post-hoc explanations feel compelling
\item
  \textbf{Survivorship Bias}: Failed astrologers leave no record
\end{enumerate}

\subsection{Methodological
Contributions}\label{methodological-contributions}

This study advances the literature through:

\begin{enumerate}
\def\labelenumi{\arabic{enumi}.}
\tightlist
\item
  \textbf{High-Precision Calculations}: Swiss Ephemeris vs.~approximate
  tables
\item
  \textbf{Sidereal Frame}: Testing Vedic (sidereal) vs.~Western
  (tropical) zodiac
\item
  \textbf{Multiple Test Correction}: Bonferroni adjustment for fair
  comparison
\item
  \textbf{Monte Carlo Validation}: Empirical null distribution
\end{enumerate}

\subsection{Limitations}\label{limitations}

\begin{enumerate}
\def\labelenumi{\arabic{enumi}.}
\tightlist
\item
  \textbf{Linear Methods Only}: Non-linear relationships unexplored
\item
  \textbf{Gold-Only}: Results may not generalize to other assets
\item
  \textbf{Daily Frequency}: Intraday patterns not tested
\item
  \textbf{Classical Techniques}: Modern astrological methods not
  evaluated
\end{enumerate}

\section{Conclusions}\label{conclusions}

This rigorous econometric investigation yields definitive results:

\begin{enumerate}
\def\labelenumi{\arabic{enumi}.}
\tightlist
\item
  \textbf{Null Granger Causality}: No planet shows significant
  predictive power
\item
  \textbf{No Spectral Peaks}: Planetary synodic periods not detected in
  returns
\item
  \textbf{Monte Carlo Confirmation}: Correlations indistinguishable from
  noise
\item
  \textbf{EMH Supported}: Markets efficiently incorporate public
  information
\end{enumerate}

\subsection{Final Assessment}\label{final-assessment}

The hypothesis that planetary positions influence Gold prices is
\textbf{falsified} at conventional significance levels. While the cosmos
may be ordered, financial markets remain efficient enough to discount
predictable orbital mechanics.

\subsection{Recommendations}\label{recommendations}

\begin{itemize}
\tightlist
\item
  \textbf{For Researchers}: Apply similar methodology to other asset
  classes
\item
  \textbf{For Practitioners}: Abandon astrological trading strategies
\item
  \textbf{For Regulators}: Financial astrology lacks empirical
  foundation
\item
  \textbf{For Educators}: Use this as case study in pseudoscience
  demarcation
\end{itemize}

\section{References}\label{references}

\phantomsection\label{refs}
\begin{CSLReferences}{1}{0}
\bibitem[\citeproctext]{ref-emh_fama}
Fama, E. F. (1970). Efficient capital markets: A review of theory and
empirical work. \emph{The Journal of Finance}, \emph{25}(2), 383--417.
\url{https://doi.org/10.2307/2325486}

\bibitem[\citeproctext]{ref-mayo_severe_testing}
Mayo, D. G., \& Spanos, A. (2006). Severe testing as a basic concept in
a neyman--pearson philosophy of induction. \emph{The British Journal for
the Philosophy of Science}, \emph{57}(2), 323--357.
\url{https://doi.org/10.1093/bjps/axl003}

\end{CSLReferences}

\section{Appendix A: VAR Model
Specification}\label{appendix-a-var-model-specification}

\begin{Shaded}
\begin{Highlighting}[]
\ImportTok{from}\NormalTok{ statsmodels.tsa.api }\ImportTok{import}\NormalTok{ VAR}

\CommentTok{\# Prepare data}
\NormalTok{data }\OperatorTok{=}\NormalTok{ pd.DataFrame(\{}
    \StringTok{\textquotesingle{}returns\textquotesingle{}}\NormalTok{: gold\_returns,}
    \StringTok{\textquotesingle{}sun\_sin\textquotesingle{}}\NormalTok{: np.sin(np.radians(sun\_longitude)),}
    \StringTok{\textquotesingle{}sun\_cos\textquotesingle{}}\NormalTok{: np.cos(np.radians(sun\_longitude)),}
    \StringTok{\textquotesingle{}moon\_sin\textquotesingle{}}\NormalTok{: np.sin(np.radians(moon\_longitude)),}
    \CommentTok{\# ... additional planets}
\NormalTok{\})}

\CommentTok{\# Fit VAR model}
\NormalTok{model }\OperatorTok{=}\NormalTok{ VAR(data)}
\NormalTok{results }\OperatorTok{=}\NormalTok{ model.fit(maxlags}\OperatorTok{=}\DecValTok{15}\NormalTok{, ic}\OperatorTok{=}\StringTok{\textquotesingle{}aic\textquotesingle{}}\NormalTok{)}

\CommentTok{\# Granger causality test}
\NormalTok{granger\_results }\OperatorTok{=}\NormalTok{ results.test\_causality(}
    \StringTok{\textquotesingle{}returns\textquotesingle{}}\NormalTok{, }
\NormalTok{    [}\StringTok{\textquotesingle{}sun\_sin\textquotesingle{}}\NormalTok{, }\StringTok{\textquotesingle{}sun\_cos\textquotesingle{}}\NormalTok{],}
\NormalTok{    kind}\OperatorTok{=}\StringTok{\textquotesingle{}f\textquotesingle{}}
\NormalTok{)}
\end{Highlighting}
\end{Shaded}

\section{Appendix B: Lomb-Scargle
Implementation}\label{appendix-b-lomb-scargle-implementation}

\begin{Shaded}
\begin{Highlighting}[]
\ImportTok{from}\NormalTok{ scipy.signal }\ImportTok{import}\NormalTok{ lombscargle}

\CommentTok{\# Define frequencies to test (cycles per day)}
\NormalTok{frequencies }\OperatorTok{=}\NormalTok{ np.linspace(}\DecValTok{1}\OperatorTok{/}\DecValTok{780}\NormalTok{, }\DecValTok{1}\OperatorTok{/}\DecValTok{7}\NormalTok{, }\DecValTok{1000}\NormalTok{)  }\CommentTok{\# Mars period to weekly}

\CommentTok{\# Compute periodogram}
\NormalTok{pgram }\OperatorTok{=}\NormalTok{ lombscargle(}
\NormalTok{    trading\_days,}
\NormalTok{    returns,}
\NormalTok{    frequencies,}
\NormalTok{    normalize}\OperatorTok{=}\VariableTok{True}
\NormalTok{)}

\CommentTok{\# Calculate False Alarm Probability}
\NormalTok{fap\_threshold }\OperatorTok{=} \OperatorTok{{-}}\NormalTok{np.log(}\FloatTok{0.01}\NormalTok{)  }\CommentTok{\# 1\% FAP}
\end{Highlighting}
\end{Shaded}

\section{Appendix C: Reproducibility}\label{appendix-c-reproducibility}

All code and data available at:
https://github.com/astro-fusion/astro\_research-white-paper/tree/main/docs/research/track\_3\_gold\_market

To reproduce:

\begin{Shaded}
\begin{Highlighting}[]
\BuiltInTok{cd}\NormalTok{ astro\_research{-}white{-}paper}
\ExtensionTok{python}\NormalTok{ src/generate\_artifacts.py}
\ExtensionTok{quarto}\NormalTok{ render docs/research/track\_3\_gold\_market/GOLD\_MARKET\_PLANETARY\_CORRELATION\_ANALYSIS.qmd}
\end{Highlighting}
\end{Shaded}





\end{document}
