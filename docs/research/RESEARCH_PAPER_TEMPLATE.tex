% Options for packages loaded elsewhere
\PassOptionsToPackage{unicode}{hyperref}
\PassOptionsToPackage{hyphens}{url}
\PassOptionsToPackage{dvipsnames,svgnames,x11names}{xcolor}
%
\documentclass[
  11pt,
  oneside,
  openany]{scrbook}

\usepackage{amsmath,amssymb}
\usepackage{setspace}
\usepackage{iftex}
\ifPDFTeX
  \usepackage[T1]{fontenc}
  \usepackage[utf8]{inputenc}
  \usepackage{textcomp} % provide euro and other symbols
\else % if luatex or xetex
  \usepackage{unicode-math}
  \defaultfontfeatures{Scale=MatchLowercase}
  \defaultfontfeatures[\rmfamily]{Ligatures=TeX,Scale=1}
\fi
\usepackage[]{libertine}
\ifPDFTeX\else  
    % xetex/luatex font selection
\fi
% Use upquote if available, for straight quotes in verbatim environments
\IfFileExists{upquote.sty}{\usepackage{upquote}}{}
\IfFileExists{microtype.sty}{% use microtype if available
  \usepackage[]{microtype}
  \UseMicrotypeSet[protrusion]{basicmath} % disable protrusion for tt fonts
}{}
\makeatletter
\@ifundefined{KOMAClassName}{% if non-KOMA class
  \IfFileExists{parskip.sty}{%
    \usepackage{parskip}
  }{% else
    \setlength{\parindent}{0pt}
    \setlength{\parskip}{6pt plus 2pt minus 1pt}}
}{% if KOMA class
  \KOMAoptions{parskip=half}}
\makeatother
\usepackage{xcolor}
\usepackage[top=30mm,left=25mm,right=25mm,bottom=30mm,heightrounded]{geometry}
\setlength{\emergencystretch}{3em} % prevent overfull lines
\setcounter{secnumdepth}{5}
% Make \paragraph and \subparagraph free-standing
\ifx\paragraph\undefined\else
  \let\oldparagraph\paragraph
  \renewcommand{\paragraph}[1]{\oldparagraph{#1}\mbox{}}
\fi
\ifx\subparagraph\undefined\else
  \let\oldsubparagraph\subparagraph
  \renewcommand{\subparagraph}[1]{\oldsubparagraph{#1}\mbox{}}
\fi

\usepackage{color}
\usepackage{fancyvrb}
\newcommand{\VerbBar}{|}
\newcommand{\VERB}{\Verb[commandchars=\\\{\}]}
\DefineVerbatimEnvironment{Highlighting}{Verbatim}{commandchars=\\\{\}}
% Add ',fontsize=\small' for more characters per line
\usepackage{framed}
\definecolor{shadecolor}{RGB}{241,243,245}
\newenvironment{Shaded}{\begin{snugshade}}{\end{snugshade}}
\newcommand{\AlertTok}[1]{\textcolor[rgb]{0.68,0.00,0.00}{#1}}
\newcommand{\AnnotationTok}[1]{\textcolor[rgb]{0.37,0.37,0.37}{#1}}
\newcommand{\AttributeTok}[1]{\textcolor[rgb]{0.40,0.45,0.13}{#1}}
\newcommand{\BaseNTok}[1]{\textcolor[rgb]{0.68,0.00,0.00}{#1}}
\newcommand{\BuiltInTok}[1]{\textcolor[rgb]{0.00,0.23,0.31}{#1}}
\newcommand{\CharTok}[1]{\textcolor[rgb]{0.13,0.47,0.30}{#1}}
\newcommand{\CommentTok}[1]{\textcolor[rgb]{0.37,0.37,0.37}{#1}}
\newcommand{\CommentVarTok}[1]{\textcolor[rgb]{0.37,0.37,0.37}{\textit{#1}}}
\newcommand{\ConstantTok}[1]{\textcolor[rgb]{0.56,0.35,0.01}{#1}}
\newcommand{\ControlFlowTok}[1]{\textcolor[rgb]{0.00,0.23,0.31}{#1}}
\newcommand{\DataTypeTok}[1]{\textcolor[rgb]{0.68,0.00,0.00}{#1}}
\newcommand{\DecValTok}[1]{\textcolor[rgb]{0.68,0.00,0.00}{#1}}
\newcommand{\DocumentationTok}[1]{\textcolor[rgb]{0.37,0.37,0.37}{\textit{#1}}}
\newcommand{\ErrorTok}[1]{\textcolor[rgb]{0.68,0.00,0.00}{#1}}
\newcommand{\ExtensionTok}[1]{\textcolor[rgb]{0.00,0.23,0.31}{#1}}
\newcommand{\FloatTok}[1]{\textcolor[rgb]{0.68,0.00,0.00}{#1}}
\newcommand{\FunctionTok}[1]{\textcolor[rgb]{0.28,0.35,0.67}{#1}}
\newcommand{\ImportTok}[1]{\textcolor[rgb]{0.00,0.46,0.62}{#1}}
\newcommand{\InformationTok}[1]{\textcolor[rgb]{0.37,0.37,0.37}{#1}}
\newcommand{\KeywordTok}[1]{\textcolor[rgb]{0.00,0.23,0.31}{#1}}
\newcommand{\NormalTok}[1]{\textcolor[rgb]{0.00,0.23,0.31}{#1}}
\newcommand{\OperatorTok}[1]{\textcolor[rgb]{0.37,0.37,0.37}{#1}}
\newcommand{\OtherTok}[1]{\textcolor[rgb]{0.00,0.23,0.31}{#1}}
\newcommand{\PreprocessorTok}[1]{\textcolor[rgb]{0.68,0.00,0.00}{#1}}
\newcommand{\RegionMarkerTok}[1]{\textcolor[rgb]{0.00,0.23,0.31}{#1}}
\newcommand{\SpecialCharTok}[1]{\textcolor[rgb]{0.37,0.37,0.37}{#1}}
\newcommand{\SpecialStringTok}[1]{\textcolor[rgb]{0.13,0.47,0.30}{#1}}
\newcommand{\StringTok}[1]{\textcolor[rgb]{0.13,0.47,0.30}{#1}}
\newcommand{\VariableTok}[1]{\textcolor[rgb]{0.07,0.07,0.07}{#1}}
\newcommand{\VerbatimStringTok}[1]{\textcolor[rgb]{0.13,0.47,0.30}{#1}}
\newcommand{\WarningTok}[1]{\textcolor[rgb]{0.37,0.37,0.37}{\textit{#1}}}

\providecommand{\tightlist}{%
  \setlength{\itemsep}{0pt}\setlength{\parskip}{0pt}}\usepackage{longtable,booktabs,array}
\usepackage{calc} % for calculating minipage widths
% Correct order of tables after \paragraph or \subparagraph
\usepackage{etoolbox}
\makeatletter
\patchcmd\longtable{\par}{\if@noskipsec\mbox{}\fi\par}{}{}
\makeatother
% Allow footnotes in longtable head/foot
\IfFileExists{footnotehyper.sty}{\usepackage{footnotehyper}}{\usepackage{footnote}}
\makesavenoteenv{longtable}
\usepackage{graphicx}
\makeatletter
\def\maxwidth{\ifdim\Gin@nat@width>\linewidth\linewidth\else\Gin@nat@width\fi}
\def\maxheight{\ifdim\Gin@nat@height>\textheight\textheight\else\Gin@nat@height\fi}
\makeatother
% Scale images if necessary, so that they will not overflow the page
% margins by default, and it is still possible to overwrite the defaults
% using explicit options in \includegraphics[width, height, ...]{}
\setkeys{Gin}{width=\maxwidth,height=\maxheight,keepaspectratio}
% Set default figure placement to htbp
\makeatletter
\def\fps@figure{htbp}
\makeatother

% PDF Header - Custom LaTeX commands for PDF output
% This file is included in the LaTeX preamble for PDF generation

% Add any custom LaTeX packages or commands here
% Example: \usepackage{booktabs}
% Example: \usepackage{longtable}
\makeatletter
\@ifpackageloaded{caption}{}{\usepackage{caption}}
\AtBeginDocument{%
\ifdefined\contentsname
  \renewcommand*\contentsname{Table of contents}
\else
  \newcommand\contentsname{Table of contents}
\fi
\ifdefined\listfigurename
  \renewcommand*\listfigurename{List of Figures}
\else
  \newcommand\listfigurename{List of Figures}
\fi
\ifdefined\listtablename
  \renewcommand*\listtablename{List of Tables}
\else
  \newcommand\listtablename{List of Tables}
\fi
\ifdefined\figurename
  \renewcommand*\figurename{Figure}
\else
  \newcommand\figurename{Figure}
\fi
\ifdefined\tablename
  \renewcommand*\tablename{Table}
\else
  \newcommand\tablename{Table}
\fi
}
\@ifpackageloaded{float}{}{\usepackage{float}}
\floatstyle{ruled}
\@ifundefined{c@chapter}{\newfloat{codelisting}{h}{lop}}{\newfloat{codelisting}{h}{lop}[chapter]}
\floatname{codelisting}{Listing}
\newcommand*\listoflistings{\listof{codelisting}{List of Listings}}
\captionsetup{labelsep=colon}
\makeatother
\makeatletter
\makeatother
\makeatletter
\@ifpackageloaded{caption}{}{\usepackage{caption}}
\@ifpackageloaded{subcaption}{}{\usepackage{subcaption}}
\makeatother
\ifLuaTeX
\usepackage[bidi=basic]{babel}
\else
\usepackage[bidi=default]{babel}
\fi
\babelprovide[main,import]{american}
% get rid of language-specific shorthands (see #6817):
\let\LanguageShortHands\languageshorthands
\def\languageshorthands#1{}
\ifLuaTeX
  \usepackage{selnolig}  % disable illegal ligatures
\fi
\usepackage{bookmark}

\IfFileExists{xurl.sty}{\usepackage{xurl}}{} % add URL line breaks if available
\urlstyle{same} % disable monospaced font for URLs
\hypersetup{
  pdftitle={Vedic Numerology-Astrology Integration System},
  pdfauthor={Bishal Ghimire},
  pdflang={en-US},
  pdfsubject={Computational Integration of Vedic Numerology and Sidereal
Astrology},
  pdfkeywords={vedic numerology, astrology, swiss ephemeris,
computational astrology},
  colorlinks=true,
  linkcolor={blue},
  filecolor={Maroon},
  citecolor={green},
  urlcolor={blue},
  pdfcreator={LaTeX via pandoc}}

\title{Vedic Numerology-Astrology Integration System}
\author{Bishal Ghimire}
\date{}

\begin{document}
\frontmatter
\maketitle

% PDF Before Body - Content to include before document body
% This file is included before the main document content

% Add any content that should appear before the main document here
% This could include custom title pages, abstracts, etc.

\setstretch{1.2}
\mainmatter
\chapter{Template: Research Paper Using Generated
Reports}\label{template-research-paper-using-generated-reports}

\section{How to Integrate the Generated PDFs into Your Research
Paper}\label{how-to-integrate-the-generated-pdfs-into-your-research-paper}

This template shows how to structure your research paper using the
generated correlation analysis reports and individual planet variation
graphs.

\begin{center}\rule{0.5\linewidth}{0.5pt}\end{center}

\section{📄 Paper Structure Template}\label{paper-structure-template}

\begin{verbatim}
TITLE: Correlation Analysis Between Vedic Astrology and Vedic Numerology: 
       An Empirical Investigation of System Independence

ABSTRACT
--------
[50-150 words summarizing research question, methodology, findings, conclusions]

KEYWORDS
--------
Vedic Astrology, Vedic Numerology, Correlation Analysis, Planetary Strength, 
Mulanka, Bhagyanka, Statistical Validation

1. INTRODUCTION
---------------
This section should explain:
- Historical context of Vedic systems
- Current gap in comparative analysis literature
- Why correlation analysis matters
- Your research question: "Is there correlation between astrology and numerology?"

2. LITERATURE REVIEW
--------------------
Discuss:
- Vedic Astrology principles (cite classical texts)
- Vedic Numerology principles (cite classical texts)
- Existing comparative studies (if any)
- Theoretical reasons why correlation might/might not exist

3. RESEARCH METHODOLOGY
-----------------------
[INSERT CONTENT FROM: vedic_correlation_research_report.pdf → "Methodology" section]

**Data Collection**
- Hourly planetary strength data for 365-day period
- Daily numerology values (Mulanka and Bhagyanka)
- Birth data: [Your birth date, time, location]

**Methodology**
[Copy exact methodology section from PDF]

**Statistical Approach**
- Pearson correlation coefficient
- Daily aggregation of hourly astrology data
- Comparison with discrete numerology values

4. INDIVIDUAL PLANET ANALYSIS
------------------------------
[INSERT GRAPHS FROM: planet_individual_variations.pdf]

For EACH of the 9 planets:

### 4.1 The Sun
[INSERT PAGE 1 FROM planet_individual_variations.pdf]

**Analysis of The Sun:**
The Sun exhibits [describe astrology pattern] with a complete cycle approximately 
every [period]. In contrast, Mulanka strength shows [describe numerology pattern] 
with only [number] discrete changes annually. The measured correlation between 
Sun strength and Mulanka is [value], indicating [interpretation].

### 4.2 The Moon
[INSERT PAGE 2 FROM planet_individual_variations.pdf]

**Analysis of The Moon:**
The Moon shows [describe pattern]...

### 4.3 Mars
[INSERT PAGE 3 FROM planet_individual_variations.pdf]

**Analysis of Mars:**
Mars exhibits [describe pattern]...

### 4.4 Mercury
[INSERT PAGE 4 FROM planet_individual_variations.pdf]

**Analysis of Mercury:**
Mercury demonstrates [describe pattern]...

### 4.5 Jupiter
[INSERT PAGE 5 FROM planet_individual_variations.pdf]

**Analysis of Jupiter:**
Jupiter shows [describe pattern]...

### 4.6 Venus
[INSERT PAGE 6 FROM planet_individual_variations.pdf]

**Analysis of Venus:**
Venus exhibits [describe pattern]...

### 4.7 Saturn
[INSERT PAGE 7 FROM planet_individual_variations.pdf]

**Analysis of Saturn:**
Saturn demonstrates [describe pattern]...

### 4.8 Rahu
[INSERT PAGE 8 FROM planet_individual_variations.pdf]

**Analysis of Rahu:**
Rahu shows [describe pattern]...

### 4.9 Ketu
[INSERT PAGE 9 FROM planet_individual_variations.pdf]

**Analysis of Ketu:**
Ketu exhibits [describe pattern]...

5. CORRELATION ANALYSIS RESULTS
--------------------------------
[INSERT CONTENT FROM: vedic_correlation_research_report.pdf → "Correlation Analysis Results" section]

**Table 5.1: Correlation Coefficients**
[Copy exact table from PDF showing correlations for all 9 planets]

**Interpretation:**
- [Describe what the correlation values mean]
- [Compare to expected ranges]
- [Note which planets show highest/lowest correlations]
- [Statistical significance if applicable]

6. STATISTICAL FINDINGS
-----------------------
[INSERT CONTENT FROM: vedic_correlation_research_report.pdf → "Key Findings" section]

Key findings include:

1. Temporal Discontinuity
   [Explain from PDF findings]

2. Correlation Analysis
   [Explain from PDF findings]

3. Independence of Systems
   [Explain from PDF findings]

4. System Characteristics
   [Explain from PDF findings]

7. DISCUSSION
-------------
Interpret your findings in context:

**What does low correlation mean?**
- The two systems are fundamentally independent
- They measure different aspects of cosmic influence
- They should be viewed as complementary, not redundant

**Practical Implications:**
- Vedic Astrology and Numerology can be used together as complementary tools
- Low correlation means they provide non-redundant information
- A person could have weak planets but strong numerology numbers (or vice versa)

**Theoretical Implications:**
- Validates traditional approach of using both systems separately
- Questions whether one system can replace the other
- Supports the idea of multiple valid approaches to Vedic analysis

**Limitations of This Study:**
- Uses simulated planetary strength data
- Limited to 365-day period
- Doesn't account for planetary aspects/transits
- Recommendations for future research using real ephemeris data

8. CONCLUSIONS
---------------
[INSERT CONTENT FROM: vedic_correlation_research_report.pdf → "Conclusions" section]

**Research Question Answer:** 
Based on analysis of [period], the evidence indicates NO significant correlation 
between Vedic Astrology planetary strength and Vedic Numerology values.

**Academic Significance:**
[Explain why this matters for understanding Vedic systems]

**Future Research:**
- Extend to multiple birth charts
- Use actual ephemeris data (Swiss Ephemeris)
- Validate against empirical life events
- Include planetary aspects and transits
- Compare with other numerology systems

9. REFERENCES
-------------
[Include all academic sources cited in your paper]

Classical Vedic Sources:
- Parasara Hora Shastra [details]
- Jaatakadeshamarga [details]
- Surya Siddhanta [details]

Modern Academic Sources:
- [Your citations here]

APPENDIX A: DATA TABLES
-----------------------
[You can extract raw data tables from the research report]

APPENDIX B: DETAILED STATISTICAL ANALYSIS
------------------------------------------
[Include additional correlation analysis details]

APPENDIX C: TECHNICAL SPECIFICATIONS
-------------------------------------
[Reference content from RESEARCH_DATA_REFERENCE.md]
\end{verbatim}

\begin{center}\rule{0.5\linewidth}{0.5pt}\end{center}

\section{🎨 How to Extract and Insert PDF
Content}\label{how-to-extract-and-insert-pdf-content}

\subsection{Step 1: Extract Text from Main
PDF}\label{step-1-extract-text-from-main-pdf}

\begin{Shaded}
\begin{Highlighting}[]
\CommentTok{\# On Mac, you can copy text directly from the PDF}
\CommentTok{\# Or use a PDF tool to extract text}
\ExtensionTok{pdftotext}\NormalTok{ vedic\_correlation\_research\_report.pdf research\_report.txt}
\end{Highlighting}
\end{Shaded}

\subsection{Step 2: Copy Graphs from Variations
PDF}\label{step-2-copy-graphs-from-variations-pdf}

\begin{Shaded}
\begin{Highlighting}[]
\CommentTok{\# Extract individual pages as images}
\CommentTok{\# Page 1 (SUN) = planet\_individual\_variations.pdf page 1}
\CommentTok{\# Page 2 (MOON) = planet\_individual\_variations.pdf page 2}
\CommentTok{\# etc.}

\CommentTok{\# Insert these as high{-}resolution figures in your paper}
\end{Highlighting}
\end{Shaded}

\subsection{Step 3: Create Figure
Captions}\label{step-3-create-figure-captions}

\begin{verbatim}
Figure 1: The Sun - Vedic Astrology (blue, continuous) vs Numerology (orange, discrete)
The line graph shows hourly-to-daily aggregated astrology strength variations 
compared to daily numerology Mulanka strength. The step pattern indicates discrete 
daily numerology changes.
\end{verbatim}

\begin{center}\rule{0.5\linewidth}{0.5pt}\end{center}

\section{📊 Example Section: How to Write About Your
Results}\label{example-section-how-to-write-about-your-results}

\subsection{Before (Generic)}\label{before-generic}

\begin{verbatim}
We analyzed planetary strengths and found low correlation.
\end{verbatim}

\subsection{After (Specific from
Report)}\label{after-specific-from-report}

\begin{verbatim}
We analyzed 9 Navagraha planets over 365 days, generating 8,760 hourly 
astrology readings and 365 daily numerology values. Daily aggregation 
yielded 365 comparable data points per planet. 

The correlation analysis revealed:
- Average absolute correlation across all planets: 0.12
- Range: -0.18 to 0.22
- All values within -0.3 to 0.3 range indicating negligible correlation

This stark contrast between the systems' temporal mechanics (8,760 changes 
vs 73 changes annually) and calculation methods suggests fundamental independence.
\end{verbatim}

\begin{center}\rule{0.5\linewidth}{0.5pt}\end{center}

\section{🖼️ Figure Placement Guide}\label{figure-placement-guide}

\subsection{In Your Manuscript}\label{in-your-manuscript}

\begin{Shaded}
\begin{Highlighting}[]
\FunctionTok{\#\#\# 4.1 The Sun}

\NormalTok{Figure 4.1 presents the strength variation patterns for The Sun over }
\NormalTok{the 365{-}day analysis period.}

\CommentTok{[}\OtherTok{HERE INSERT: Page 1 from planet\_individual\_variations.pdf}\CommentTok{]}

\NormalTok{**Figure 4.1 Caption:**}
\NormalTok{The Sun\textquotesingle{}s strength variation pattern comparing Vedic Astrology }
\NormalTok{(top panel, blue line with shading showing min{-}max daily range) }
\NormalTok{and Vedic Numerology Mulanka influence (bottom panel, orange }
\NormalTok{step function showing discrete daily changes). The astrology }
\NormalTok{system shows 24 complete cycles while numerology shows only }
\NormalTok{discrete jumps, reflecting their fundamentally different temporal }
\NormalTok{resolution and calculation methods.}

\NormalTok{The correlation coefficient between Sun strength and Mulanka }
\NormalTok{is r = }\CommentTok{[}\OtherTok{value}\CommentTok{]}\NormalTok{, indicating negligible linear relationship.}
\end{Highlighting}
\end{Shaded}

\begin{center}\rule{0.5\linewidth}{0.5pt}\end{center}

\section{📝 Writing Tips for Research
Paper}\label{writing-tips-for-research-paper}

\subsection{1. Explain Your Research Question
Clearly}\label{explain-your-research-question-clearly}

\begin{verbatim}
"Our central research question: Given that Vedic Astrology strength 
measurements change hourly while Vedic Numerology values change daily, 
is there a measurable correlation between these two independent systems?"
\end{verbatim}

\subsection{2. Justify Your Methodology}\label{justify-your-methodology}

\begin{verbatim}
"To enable direct comparison of these systems with different temporal 
granularities, we aggregated the 24 hourly astrology readings to produce 
daily average values, creating aligned datasets suitable for correlation analysis."
\end{verbatim}

\subsection{3. Present Your Findings
Objectively}\label{present-your-findings-objectively}

\begin{verbatim}
"Our analysis of [period] reveals low correlation coefficients 
averaging [value] across all planets, with 95% of values falling 
in the -0.3 to 0.3 range indicating negligible correlation."
\end{verbatim}

\subsection{4. Discuss Implications}\label{discuss-implications}

\begin{verbatim}
"The absence of significant correlation suggests these systems 
operate on fundamentally different principles and should be understood 
as providing complementary rather than redundant information."
\end{verbatim}

\begin{center}\rule{0.5\linewidth}{0.5pt}\end{center}

\section{🔍 Specific Content to Copy from
Reports}\label{specific-content-to-copy-from-reports}

\subsection{From
vedic\_correlation\_research\_report.pdf:}\label{from-vedic_correlation_research_report.pdf}

\textbf{Methodology Section} (for your Methods chapter) - Data
Collection paragraph - Temporal Comparison paragraph\\
- Correlation Analysis paragraph

\textbf{Correlation Results Table} (for your Results chapter) - All 9
planets with correlation coefficients - Can reformat as needed for your
paper

\textbf{Key Findings} (for your Results/Discussion) - Temporal
Discontinuity finding - Correlation Analysis finding - Independence of
Systems finding - System Characteristics finding

\textbf{Conclusions} (for your Conclusions section) - Main answer to
research question - Interpretation of findings - Academic significance -
Limitations - Future research directions

\subsection{From
planet\_individual\_variations.pdf:}\label{from-planet_individual_variations.pdf}

\textbf{All 9 Planet Graphs} (for your Analysis section) - Each planet
gets a dedicated figure with two panels - Top panel: Vedic Astrology
strength (continuous line) - Bottom panel: Vedic Numerology strength
(discrete steps)

\begin{center}\rule{0.5\linewidth}{0.5pt}\end{center}

\section{📋 Checklist for Paper
Completion}\label{checklist-for-paper-completion}

Using these reports, ensure your paper includes:

\begin{itemize}
\tightlist
\item[$\square$]
  Introduction explaining both Vedic systems
\item[$\square$]
  Literature review of existing comparative studies\\
\item[$\square$]
  Clear research question statement
\item[$\square$]
  Methodology section (copied/adapted from PDF)
\item[$\square$]
  All 9 planet analysis sections with graphs
\item[$\square$]
  Correlation analysis table with coefficients
\item[$\square$]
  Statistical interpretation of findings
\item[$\square$]
  Discussion of what low correlation means
\item[$\square$]
  Conclusions answering your research question
\item[$\square$]
  Acknowledgment of limitations
\item[$\square$]
  Suggestions for future research
\item[$\square$]
  Proper citations to Vedic texts and modern sources
\item[$\square$]
  High-quality figure captions
\item[$\square$]
  References in appropriate academic format
\end{itemize}

\begin{center}\rule{0.5\linewidth}{0.5pt}\end{center}

\section{🎓 Academic Tone Examples}\label{academic-tone-examples}

\subsection{How to Present Your Findings
Academically:}\label{how-to-present-your-findings-academically}

\textbf{Instead of:} ``The systems don't match up''

\textbf{Write:} ``The quantitative analysis reveals negligible linear
correlation between the continuous astrology system and the discrete
numerology system, with correlation coefficients concentrated in the
-0.3 to 0.3 range.''

\textbf{Instead of:} ``Numbers were different''

\textbf{Write:} ``The fundamental difference in temporal granularity
(8,760 hourly astrology measurements vs 73 annual numerology changes)
combined with distinct calculation methodologies yields statistically
independent variables.''

\textbf{Instead of:} ``They probably work separately''

\textbf{Write:} ``These findings corroborate the traditional Vedic
approach of treating astrology and numerology as complementary systems
providing non-redundant insights into cosmic influence.''

\begin{center}\rule{0.5\linewidth}{0.5pt}\end{center}

\section{🚀 Next Steps}\label{next-steps}

\begin{enumerate}
\def\labelenumi{\arabic{enumi}.}
\tightlist
\item
  \textbf{Review the main PDF:} vedic\_correlation\_research\_report.pdf
\item
  \textbf{Extract content:} Copy methodology, findings, conclusions
\item
  \textbf{Review individual graphs:} planet\_individual\_variations.pdf
\item
  \textbf{Study data reference:} RESEARCH\_DATA\_REFERENCE.md
\item
  \textbf{Write your paper:} Use template structure above
\item
  \textbf{Insert figures:} Place each planet graph with caption
\item
  \textbf{Format properly:} Follow your journal's style guide
\item
  \textbf{Submit:} Ready for publication
\end{enumerate}

\begin{center}\rule{0.5\linewidth}{0.5pt}\end{center}

\section{📞 Customization Notes}\label{customization-notes}

All generated content is yours to use: - ✅ Copy text from PDFs into
your paper - ✅ Insert graphs and tables as figures - ✅ Adapt writing
as needed for your voice - ✅ Reorganize sections for your paper
structure - ✅ Add additional analysis or discussion as desired - ✅
Cite appropriately in references

\begin{center}\rule{0.5\linewidth}{0.5pt}\end{center}

\textbf{Template Created:} January 25, 2026\\
\textbf{For Use With:} vedic\_correlation\_research\_report.pdf +
planet\_individual\_variations.pdf\\
\textbf{Documentation:} RESEARCH\_REPORT\_GUIDE.md +
RESEARCH\_DATA\_REFERENCE.md\\
\textbf{Status:} ✅ Ready for Publication


\backmatter

\end{document}
