% Options for packages loaded elsewhere
% Options for packages loaded elsewhere
\PassOptionsToPackage{unicode}{hyperref}
\PassOptionsToPackage{hyphens}{url}
\PassOptionsToPackage{dvipsnames,svgnames,x11names}{xcolor}
%
\documentclass[
  american,
  11pt,
  oneside,
  openany]{scrbook}
\usepackage{xcolor}
\usepackage[top=30mm,left=25mm,right=25mm,bottom=30mm,heightrounded]{geometry}
\usepackage{amsmath,amssymb}
\setcounter{secnumdepth}{5}
\usepackage{iftex}
\ifPDFTeX
  \usepackage[T1]{fontenc}
  \usepackage[utf8]{inputenc}
  \usepackage{textcomp} % provide euro and other symbols
\else % if luatex or xetex
  \usepackage{unicode-math} % this also loads fontspec
  \defaultfontfeatures{Scale=MatchLowercase}
  \defaultfontfeatures[\rmfamily]{Ligatures=TeX,Scale=1}
\fi
\usepackage[]{libertine}
\ifPDFTeX\else
  % xetex/luatex font selection
\fi
% Use upquote if available, for straight quotes in verbatim environments
\IfFileExists{upquote.sty}{\usepackage{upquote}}{}
\IfFileExists{microtype.sty}{% use microtype if available
  \usepackage[]{microtype}
  \UseMicrotypeSet[protrusion]{basicmath} % disable protrusion for tt fonts
}{}
\usepackage{setspace}
\makeatletter
\@ifundefined{KOMAClassName}{% if non-KOMA class
  \IfFileExists{parskip.sty}{%
    \usepackage{parskip}
  }{% else
    \setlength{\parindent}{0pt}
    \setlength{\parskip}{6pt plus 2pt minus 1pt}}
}{% if KOMA class
  \KOMAoptions{parskip=half}}
\makeatother
% Make \paragraph and \subparagraph free-standing
\makeatletter
\ifx\paragraph\undefined\else
  \let\oldparagraph\paragraph
  \renewcommand{\paragraph}{
    \@ifstar
      \xxxParagraphStar
      \xxxParagraphNoStar
  }
  \newcommand{\xxxParagraphStar}[1]{\oldparagraph*{#1}\mbox{}}
  \newcommand{\xxxParagraphNoStar}[1]{\oldparagraph{#1}\mbox{}}
\fi
\ifx\subparagraph\undefined\else
  \let\oldsubparagraph\subparagraph
  \renewcommand{\subparagraph}{
    \@ifstar
      \xxxSubParagraphStar
      \xxxSubParagraphNoStar
  }
  \newcommand{\xxxSubParagraphStar}[1]{\oldsubparagraph*{#1}\mbox{}}
  \newcommand{\xxxSubParagraphNoStar}[1]{\oldsubparagraph{#1}\mbox{}}
\fi
\makeatother


\usepackage{longtable,booktabs,array}
\usepackage{calc} % for calculating minipage widths
% Correct order of tables after \paragraph or \subparagraph
\usepackage{etoolbox}
\makeatletter
\patchcmd\longtable{\par}{\if@noskipsec\mbox{}\fi\par}{}{}
\makeatother
% Allow footnotes in longtable head/foot
\IfFileExists{footnotehyper.sty}{\usepackage{footnotehyper}}{\usepackage{footnote}}
\makesavenoteenv{longtable}
\usepackage{graphicx}
\makeatletter
\newsavebox\pandoc@box
\newcommand*\pandocbounded[1]{% scales image to fit in text height/width
  \sbox\pandoc@box{#1}%
  \Gscale@div\@tempa{\textheight}{\dimexpr\ht\pandoc@box+\dp\pandoc@box\relax}%
  \Gscale@div\@tempb{\linewidth}{\wd\pandoc@box}%
  \ifdim\@tempb\p@<\@tempa\p@\let\@tempa\@tempb\fi% select the smaller of both
  \ifdim\@tempa\p@<\p@\scalebox{\@tempa}{\usebox\pandoc@box}%
  \else\usebox{\pandoc@box}%
  \fi%
}
% Set default figure placement to htbp
\def\fps@figure{htbp}
\makeatother



\ifLuaTeX
\usepackage[bidi=basic]{babel}
\else
\usepackage[bidi=default]{babel}
\fi
% get rid of language-specific shorthands (see #6817):
\let\LanguageShortHands\languageshorthands
\def\languageshorthands#1{}
\ifLuaTeX
  \usepackage[english]{selnolig} % disable illegal ligatures
\fi


\setlength{\emergencystretch}{3em} % prevent overfull lines

\providecommand{\tightlist}{%
  \setlength{\itemsep}{0pt}\setlength{\parskip}{0pt}}



 


% PDF Header - Custom LaTeX commands for PDF output
% This file is included in the LaTeX preamble for PDF generation

% Add any custom LaTeX packages or commands here
% Example: \usepackage{booktabs}
% Example: \usepackage{longtable}
\makeatletter
\@ifpackageloaded{caption}{}{\usepackage{caption}}
\AtBeginDocument{%
\ifdefined\contentsname
  \renewcommand*\contentsname{Table of contents}
\else
  \newcommand\contentsname{Table of contents}
\fi
\ifdefined\listfigurename
  \renewcommand*\listfigurename{List of Figures}
\else
  \newcommand\listfigurename{List of Figures}
\fi
\ifdefined\listtablename
  \renewcommand*\listtablename{List of Tables}
\else
  \newcommand\listtablename{List of Tables}
\fi
\ifdefined\figurename
  \renewcommand*\figurename{Figure}
\else
  \newcommand\figurename{Figure}
\fi
\ifdefined\tablename
  \renewcommand*\tablename{Table}
\else
  \newcommand\tablename{Table}
\fi
}
\@ifpackageloaded{float}{}{\usepackage{float}}
\floatstyle{ruled}
\@ifundefined{c@chapter}{\newfloat{codelisting}{h}{lop}}{\newfloat{codelisting}{h}{lop}[chapter]}
\floatname{codelisting}{Listing}
\newcommand*\listoflistings{\listof{codelisting}{List of Listings}}
\captionsetup{labelsep=colon}
\makeatother
\makeatletter
\makeatother
\makeatletter
\@ifpackageloaded{caption}{}{\usepackage{caption}}
\@ifpackageloaded{subcaption}{}{\usepackage{subcaption}}
\makeatother
\usepackage{bookmark}
\IfFileExists{xurl.sty}{\usepackage{xurl}}{} % add URL line breaks if available
\urlstyle{same}
\hypersetup{
  pdftitle={Vedic Numerology-Astrology Integration System},
  pdfauthor={Bishal Ghimire},
  pdflang={en-US},
  pdfsubject={Computational Integration of Vedic Numerology and Sidereal
Astrology},
  pdfkeywords={vedic numerology, astrology, swiss ephemeris,
computational astrology},
  colorlinks=true,
  linkcolor={blue},
  filecolor={Maroon},
  citecolor={green},
  urlcolor={blue},
  pdfcreator={LaTeX via pandoc}}


\title{Vedic Numerology-Astrology Integration System}
\author{Bishal Ghimire}
\date{}
\begin{document}
\frontmatter
\maketitle

% PDF Before Body - Content to include before document body
% This file is included before the main document content

% Add any content that should appear before the main document here
% This could include custom title pages, abstracts, etc.


\setstretch{1.2}
\mainmatter
\chapter{Correlation Analysis Between Vedic Astrology and Vedic
Numerology: An Empirical Investigation of System
Independence}\label{correlation-analysis-between-vedic-astrology-and-vedic-numerology-an-empirical-investigation-of-system-independence}

\textbf{Authors:} Astro-Fusion Research Team\\
\textbf{Date:} January 26, 2026\\
\textbf{Status:} DRAFT (Phase 6 Findings)

\begin{center}\rule{0.5\linewidth}{0.5pt}\end{center}

\section{ABSTRACT}\label{abstract}

This study investigates the statistical correlation between traditional
Vedic systems---Vedic Astrology (Parashari Jyotish) and Vedic Numerology
(Sankhya Sastra)---specifically in the context of earthquake prediction.
Using a novel data pipeline, we analyzed a sample of seismic events to
determine if ``Universal Day Number'' (numerology) and ``Planetary
Dignity'' (astrology) show periodicity or predictive correlation.
\textbf{Results indicated no statistically significant correlation
(Schuster's p-value = 0.19)} in the observed sample. Furthermore, a
Negative Binomial regression model failed to converge (Delta AIC: NaN),
suggesting that earthquake occurrence in the sample dataset is
indistinguishable from random noise. This finding supports the null
hypothesis that these symbolic systems operate independently of physical
seismic triggers in the short term.

\textbf{Keywords:} Vedic Astrology, Numerology, Earthquake Prediction,
Schuster's Test, Monte Carlo Simulation, Negative Result.

\begin{center}\rule{0.5\linewidth}{0.5pt}\end{center}

\section{1. INTRODUCTION}\label{introduction}

The ``Astro-Fusion'' initiative aims to bridge the gap between
deterministic ancient systems and stochastic geophysical phenomena.
While anecdotal evidence suggests correlations between planetary
alignments and earthquakes, rigorous statistical validation has been
sparse.

Our central research question: \textbf{``Does the 9-day Numerological
Cycle or the continuous variation of Planetary Strength offer
Information Gain over a Poisson baseline for earthquake forecasting?''}

\begin{center}\rule{0.5\linewidth}{0.5pt}\end{center}

\section{2. METHODOLOGY}\label{methodology}

We constructed a computational pipeline consisting of: 1. \textbf{Data
Ingestion:} USGS Earthquake Catalog (or representative sample). 2.
\textbf{Feature Engineering:} * \emph{Numerology:} Universal Day Number
(1-9) calculated using the Pythagorean reduction. * \emph{Astrology:}
Planetary Strength (0-100) calculated using Swiss Ephemeris. 3.
\textbf{Statistical Tests:} * \textbf{Schuster's Test for Periodicity:}
To detect cyclic clustering on the 9-day numerology wheel. *
\textbf{Negative Binomial Regression:} To model earthquake counts as a
function of numerological and astrological predictors. * \textbf{Monte
Carlo Validation:} 1,000 random shuffles to establish a significance
threshold.

\begin{center}\rule{0.5\linewidth}{0.5pt}\end{center}

\section{3. RESULTS}\label{results}

\subsection{3.1. Periodicity Analysis (Schuster's
Test)}\label{periodicity-analysis-schusters-test}

We mapped earthquake occurrences to a phase angle \(\theta\) on the
9-day numerology cycle (\(1 \to 0, \dots, 9 \to 2\pi\)).

\begin{itemize}
\tightlist
\item
  \textbf{Resultant Vector (\(R\)):} The vector sum of events on the
  unit circle.
\item
  \textbf{Schuster's p-value:} \(P = e^{-R^2/N}\).
\end{itemize}

\textbf{Findings:} * \textbf{Calculated p-value:} \texttt{0.190} *
\textbf{Interpretation:} Since \(p > 0.05\), we \textbf{fail to reject
the null hypothesis}. The distribution of earthquakes across the 9
numerological days is statistically indistinguishable from a uniform
random distribution. There is no significant 9-day periodicity in the
sample.

\subsection{3.2. Regression Modeling
(GLM)}\label{regression-modeling-glm}

We attempted to fit a Generalized Linear Model (GLM) using a Negative
Binomial family to account for overdispersion.

\begin{itemize}
\tightlist
\item
  \textbf{Model:}
  \texttt{Eq\_Count\ \textasciitilde{}\ seasonality\ +\ C(Universal\_Day)\ +\ Planetary\_Strength}
\item
  \textbf{Result:} The model encountered \textbf{Perfect Separation}
  warnings and failed to converge (\texttt{Delta\ AIC:\ NaN}).
\item
  \textbf{Interpretation:} The sample size (N=5-15 events) was
  insufficient relative to the number of parameters, leading to
  overfitting. No robust predictive signal could be extracted.
\end{itemize}

\subsection{3.3. Significance Testing (Monte
Carlo)}\label{significance-testing-monte-carlo}

To verify if any signal was an artifact, we performed 1,000 Monte Carlo
shuffles of the earthquake time series.

\begin{itemize}
\tightlist
\item
  \textbf{95th Percentile Threshold:} -3.38
\item
  \textbf{Real Delta AIC:} NaN (Failed)
\item
  \textbf{Conclusion:} \textbf{VALIDATION FAIL}. The signal did not
  exceed the noise floor.
\end{itemize}

\begin{center}\rule{0.5\linewidth}{0.5pt}\end{center}

\section{4. DISCUSSION}\label{discussion}

The absence of correlation in this study highlights the challenges of
``Astro-Seismology.''

\begin{enumerate}
\def\labelenumi{\arabic{enumi}.}
\tightlist
\item
  \textbf{Temporal Granularity:} Astrological strength changes hourly,
  while Numerology changes daily. This temporal mismatch may obscure
  finer correlations.
\item
  \textbf{Data Limitations:} The use of a small sample dataset for
  validation prevented the regression model from converging. A larger
  longitudinal study (1900-2023) is required to definitively rule out
  subtle effect sizes.
\item
  \textbf{Independence of Systems:} The lack of correlation between
  Planet Strenth and Numerology suggests these systems might be
  describing different dimensions of experience (e.g., Physical
  vs.~Archetypal) rather than coupled physical forces.
\end{enumerate}

\begin{center}\rule{0.5\linewidth}{0.5pt}\end{center}

\section{5. CONCLUSION}\label{conclusion}

Based on the analysis of the current dataset, we find \textbf{no
evidence} to support the hypothesis that Vedic Numerology or Planetary
Strength (as defined) are predictive of earthquake occurrences. The
systems appear statistically independent. Future work should focus on:
1. Expanding the dataset to \textgreater10,000 events. 2. Implementing
non-linear ML models (Random Forest) to capture complex interactions. 3.
Refining the ``Universal Day'' definition to account for local
timezones.

\begin{center}\rule{0.5\linewidth}{0.5pt}\end{center}

\section{APPENDIX A: RAW DATA}\label{appendix-a-raw-data}

\begin{itemize}
\tightlist
\item
  \textbf{Validation Report:} \texttt{validation\_report.json}
\item
  \textbf{Regression Matrix:} \texttt{regression\_matrix.csv}
\end{itemize}


\backmatter


\end{document}
