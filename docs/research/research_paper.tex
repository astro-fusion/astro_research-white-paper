% Options for packages loaded elsewhere
% Options for packages loaded elsewhere
\PassOptionsToPackage{unicode}{hyperref}
\PassOptionsToPackage{hyphens}{url}
\PassOptionsToPackage{dvipsnames,svgnames,x11names}{xcolor}
%
\documentclass[
  american,
  11pt,
  oneside,
  openany]{scrbook}
\usepackage{xcolor}
\usepackage[top=30mm,left=25mm,right=25mm,bottom=30mm,heightrounded]{geometry}
\usepackage{amsmath,amssymb}
\setcounter{secnumdepth}{5}
\usepackage{iftex}
\ifPDFTeX
  \usepackage[T1]{fontenc}
  \usepackage[utf8]{inputenc}
  \usepackage{textcomp} % provide euro and other symbols
\else % if luatex or xetex
  \usepackage{unicode-math} % this also loads fontspec
  \defaultfontfeatures{Scale=MatchLowercase}
  \defaultfontfeatures[\rmfamily]{Ligatures=TeX,Scale=1}
\fi
\usepackage[]{libertine}
\ifPDFTeX\else
  % xetex/luatex font selection
\fi
% Use upquote if available, for straight quotes in verbatim environments
\IfFileExists{upquote.sty}{\usepackage{upquote}}{}
\IfFileExists{microtype.sty}{% use microtype if available
  \usepackage[]{microtype}
  \UseMicrotypeSet[protrusion]{basicmath} % disable protrusion for tt fonts
}{}
\usepackage{setspace}
\makeatletter
\@ifundefined{KOMAClassName}{% if non-KOMA class
  \IfFileExists{parskip.sty}{%
    \usepackage{parskip}
  }{% else
    \setlength{\parindent}{0pt}
    \setlength{\parskip}{6pt plus 2pt minus 1pt}}
}{% if KOMA class
  \KOMAoptions{parskip=half}}
\makeatother
% Make \paragraph and \subparagraph free-standing
\makeatletter
\ifx\paragraph\undefined\else
  \let\oldparagraph\paragraph
  \renewcommand{\paragraph}{
    \@ifstar
      \xxxParagraphStar
      \xxxParagraphNoStar
  }
  \newcommand{\xxxParagraphStar}[1]{\oldparagraph*{#1}\mbox{}}
  \newcommand{\xxxParagraphNoStar}[1]{\oldparagraph{#1}\mbox{}}
\fi
\ifx\subparagraph\undefined\else
  \let\oldsubparagraph\subparagraph
  \renewcommand{\subparagraph}{
    \@ifstar
      \xxxSubParagraphStar
      \xxxSubParagraphNoStar
  }
  \newcommand{\xxxSubParagraphStar}[1]{\oldsubparagraph*{#1}\mbox{}}
  \newcommand{\xxxSubParagraphNoStar}[1]{\oldsubparagraph{#1}\mbox{}}
\fi
\makeatother


\usepackage{longtable,booktabs,array}
\usepackage{calc} % for calculating minipage widths
% Correct order of tables after \paragraph or \subparagraph
\usepackage{etoolbox}
\makeatletter
\patchcmd\longtable{\par}{\if@noskipsec\mbox{}\fi\par}{}{}
\makeatother
% Allow footnotes in longtable head/foot
\IfFileExists{footnotehyper.sty}{\usepackage{footnotehyper}}{\usepackage{footnote}}
\makesavenoteenv{longtable}
\usepackage{graphicx}
\makeatletter
\newsavebox\pandoc@box
\newcommand*\pandocbounded[1]{% scales image to fit in text height/width
  \sbox\pandoc@box{#1}%
  \Gscale@div\@tempa{\textheight}{\dimexpr\ht\pandoc@box+\dp\pandoc@box\relax}%
  \Gscale@div\@tempb{\linewidth}{\wd\pandoc@box}%
  \ifdim\@tempb\p@<\@tempa\p@\let\@tempa\@tempb\fi% select the smaller of both
  \ifdim\@tempa\p@<\p@\scalebox{\@tempa}{\usebox\pandoc@box}%
  \else\usebox{\pandoc@box}%
  \fi%
}
% Set default figure placement to htbp
\def\fps@figure{htbp}
\makeatother



\ifLuaTeX
\usepackage[bidi=basic]{babel}
\else
\usepackage[bidi=default]{babel}
\fi
% get rid of language-specific shorthands (see #6817):
\let\LanguageShortHands\languageshorthands
\def\languageshorthands#1{}
\ifLuaTeX
  \usepackage[english]{selnolig} % disable illegal ligatures
\fi


\setlength{\emergencystretch}{3em} % prevent overfull lines

\providecommand{\tightlist}{%
  \setlength{\itemsep}{0pt}\setlength{\parskip}{0pt}}



 


% PDF Header - Custom LaTeX commands for PDF output
% This file is included in the LaTeX preamble for PDF generation

% Add any custom LaTeX packages or commands here
% Example: \usepackage{booktabs}
% Example: \usepackage{longtable}
\makeatletter
\@ifpackageloaded{caption}{}{\usepackage{caption}}
\AtBeginDocument{%
\ifdefined\contentsname
  \renewcommand*\contentsname{Table of contents}
\else
  \newcommand\contentsname{Table of contents}
\fi
\ifdefined\listfigurename
  \renewcommand*\listfigurename{List of Figures}
\else
  \newcommand\listfigurename{List of Figures}
\fi
\ifdefined\listtablename
  \renewcommand*\listtablename{List of Tables}
\else
  \newcommand\listtablename{List of Tables}
\fi
\ifdefined\figurename
  \renewcommand*\figurename{Figure}
\else
  \newcommand\figurename{Figure}
\fi
\ifdefined\tablename
  \renewcommand*\tablename{Table}
\else
  \newcommand\tablename{Table}
\fi
}
\@ifpackageloaded{float}{}{\usepackage{float}}
\floatstyle{ruled}
\@ifundefined{c@chapter}{\newfloat{codelisting}{h}{lop}}{\newfloat{codelisting}{h}{lop}[chapter]}
\floatname{codelisting}{Listing}
\newcommand*\listoflistings{\listof{codelisting}{List of Listings}}
\captionsetup{labelsep=colon}
\makeatother
\makeatletter
\makeatother
\makeatletter
\@ifpackageloaded{caption}{}{\usepackage{caption}}
\@ifpackageloaded{subcaption}{}{\usepackage{subcaption}}
\makeatother
\usepackage{bookmark}
\IfFileExists{xurl.sty}{\usepackage{xurl}}{} % add URL line breaks if available
\urlstyle{same}
\hypersetup{
  pdftitle={Correlation Analysis Between Vedic Astrology and Vedic Numerology},
  pdfauthor={Astro-Fusion Research Team},
  pdflang={en-US},
  pdfsubject={Computational Integration of Vedic Numerology and Sidereal
Astrology},
  pdfkeywords={Vedic Astrology, Vedic Numerology, Earthquake
Prediction, Schuster's Test, Monte Carlo Simulation, Shadbala, Geometric
Probability},
  colorlinks=true,
  linkcolor={blue},
  filecolor={Maroon},
  citecolor={green},
  urlcolor={blue},
  pdfcreator={LaTeX via pandoc}}


\title{Correlation Analysis Between Vedic Astrology and Vedic
Numerology}
\usepackage{etoolbox}
\makeatletter
\providecommand{\subtitle}[1]{% add subtitle to \maketitle
  \apptocmd{\@title}{\par {\large #1 \par}}{}{}
}
\makeatother
\subtitle{An Empirical Investigation of System Independence in Seismic
Forecasting}
\author{Astro-Fusion Research Team}
\date{2026-01-26}
\begin{document}
\frontmatter
\maketitle

% PDF Before Body - Content to include before document body
% This file is included before the main document content

% Add any content that should appear before the main document here
% This could include custom title pages, abstracts, etc.


\setstretch{1.2}
\mainmatter
\chapter{Abstract}\label{abstract}

This study conducts a rigorous statistical investigation into the
correlation between two ancient deterministic systems---Vedic Astrology
(Parashari Jyotish) and Vedic Numerology (Sankhya Sastra)---specifically
focused on their potential as predictors for earthquake occurrences.
Utilizing a high-resolution data pipeline (Swiss Ephemeris for astrology
and Pythagorean reduction for numerology), we analyzed a dataset of 552
significant seismic events (Magnitude \textgreater{} 6.0) from the USGS
catalog (2020-2023). Our results indicate that the 9-day numerological
cycle shows no statistically significant periodicity (Schuster's p-value
= 0.067). Comparative analysis of 2-hour interval planetary strength
curves reveals a fundamental temporal and frequency mismatch between the
continuous astrological cycles and discrete numerological steps. The
study concludes that these systems operate as independent symbolic
frameworks, providing non-overlapping information that does not
correlate with short-term physical seismic triggers.

\chapter{1. Introduction}\label{introduction}

The attempt to correlate celestial movements with terrestrial events has
been a central pillar of Vedic sciences for millennia. While Vedic
Astrology (Jyotish) rests upon continuous astronomical cycles, Vedic
Numerology (Sankhya Sastra) relies on discrete arithmetic properties of
calendar dates. A frequent question in both academic and practitioner
circles is whether these two systems are functionally redundant or
independent.

This research leverages the \textbf{Astro-Fusion Pipeline} to perform
the first large-scale quantitative comparison of these systems, using
the stochastic nature of earthquake occurrences as a common baseline for
validation.

\chapter{2. Methodology}\label{methodology}

\section{2.1. Data Sources}\label{data-sources}

\begin{itemize}
\tightlist
\item
  \textbf{Earthquake Catalog:} 552 events fetched via USGS Earthquake
  Hazards Program API (2020--2023, Magnitude \(\ge\) 6.0).
\item
  \textbf{Astrological Ephemeris:} Swiss Ephemeris (high-precision
  sidereal positions).
\item
  \textbf{Numerology:} Universal Day Number (UDN) mapping to the 9-day
  Navagraha cycle.
\end{itemize}

\section{2.2. Feature Engineering}\label{feature-engineering}

We defined two primary metrics for comparison: 1. \textbf{Planetary
Strength (\(\sigma\)):} A continuous value (0--100) calculated at 2-hour
intervals, incorporating Shadbala components (Sthana, Chesta, and Yuddha
Bala). 2. \textbf{Numerological Strength (\(\nu\)):} A discrete daily
value (1--9) mapped to an auspiciousness index (60--95).

Table 1 summarizes the technical divergence between the datasets.

\begin{longtable}[]{@{}lll@{}}
\caption{Technical Comparison of System
Metrics}\label{tbl-specs}\tabularnewline
\toprule\noalign{}
Aspect & Vedic Astrology & Vedic Numerology \\
\midrule\noalign{}
\endfirsthead
\toprule\noalign{}
Aspect & Vedic Astrology & Vedic Numerology \\
\midrule\noalign{}
\endhead
\bottomrule\noalign{}
\endlastfoot
\textbf{Resolution} & 2-hour (Continuous) & 24-hour (Discrete) \\
\textbf{Data Points (2023)} & 4,380 & 365 \\
\textbf{Logic} & Orbital Mechanics & Digital Root Reduction \\
\textbf{Range} & 0-100 (Infinite states) & 1-9 (9 states) \\
\end{longtable}

\chapter{3. Visualization of Planetary
Variations}\label{visualization-of-planetary-variations}

The fundamental challenge in correlating these systems is the
\textbf{Frequency Mismatch}. Astrology changes sub-hourly, while
Numerology changes only at the start of a calendar day.

\begin{figure}[H]

\caption{\label{fig-mars}Comparison of Mars Strength Variations
(Continuous) vs.~Numerological Steps (Discrete). Note the
`Step-Function' nature of Numerology (Orange) against the Sinusoidal
variations of Astrology (Blue).}

\centering{

\pandocbounded{\includegraphics[keepaspectratio]{figures/mars_comparison.png}}

}

\end{figure}%

As seen in Figure~\ref{fig-mars}, the continuous blue line represents
the actual astronomical strength of Mars. The orange steps represent the
numerological influence. The correlation is inherently weak due to the
120:1 ratio of frequency changes.

\chapter{4. Results and Statistical
Analysis}\label{results-and-statistical-analysis}

\section{4.1. Periodicity (Schuster's
Test)}\label{periodicity-schusters-test}

We performed Schuster's Test for periodicity on the 9-day numerology
wheel to see if earthquakes cluster on specific ``Universal Day
Numbers.''

\begin{itemize}
\tightlist
\item
  \textbf{Total Events:} 552
\item
  \textbf{Resultant Vector (\(R\)):} 38.57
\item
  \textbf{Schuster's p-value:} 0.067
\end{itemize}

\textbf{Interpretation:} While \(p=0.067\) is close to the traditional
\(0.05\) threshold, it fails to achieve statistical significance. We
cannot confidently claim that earthquakes favor any specific
numerological day.

\section{4.2. Predictive Modeling (Monte
Carlo)}\label{predictive-modeling-monte-carlo}

We fitted a Negative Binomial regression model to test if the
``Variation Curves'' of planet strength offer information gain over a
random baseline. To validate this, we ran 1,000 Monte Carlo shuffles.

\begin{figure}[H]

\caption{\label{fig-mc}Monte Carlo Distribution of Model Fit (AIC
Improvements). The real data signal (Vertical Line) is compared against
1,000 random permutations.}

\centering{

\pandocbounded{\includegraphics[keepaspectratio]{figures/monte_carlo_distribution.png}}

}

\end{figure}%

\begin{itemize}
\tightlist
\item
  \textbf{95th Percentile Noise Floor:} 27.95
\item
  \textbf{Real Signal Delta AIC:} 22.13
\item
  \textbf{Validation Status:} FAIL (Signal within noise).
\end{itemize}

\chapter{5. Discussion}\label{discussion}

The near-zero correlation (\(r \approx 0.08\)) between Astrology and
Numerology suggests system independence.

\begin{enumerate}
\def\labelenumi{\arabic{enumi}.}
\tightlist
\item
  \textbf{Information Independence:} Rather than being redundant, these
  systems provide unique, non-overlapping data. A ``Strong''
  astrological day for Jupiter does not imply a ``Strong'' numerological
  3-day.
\item
  \textbf{Temporal Mismatch:} Correlation is physically limited by the
  resolution divergence. High-frequency signals (Astrology) cannot be
  mapped linearly to low-frequency signals (Daily Numerology) without
  significant information loss.
\item
  \textbf{The Null Hypothesis:} In the context of seismic triggering,
  neither system achieved the significance required to replace standard
  Poisson models, though the \(p=0.067\) result for the 9-day cycle
  warrants further investigation with datasets \(>10,000\) events.
\end{enumerate}

\chapter{6. Conclusion}\label{conclusion}

Vedic Astrology and Vedic Numerology operate as independent dimensions
of symbolic representation. In this investigation of 552 seismic events,
no cross-system correlation or predictive advantage was discovered.
Professionals are advised to treat these as \textbf{Complementary
Systems} rather than redundant alternatives.

\chapter{References}\label{references}

\begin{enumerate}
\def\labelenumi{\arabic{enumi}.}
\tightlist
\item
  Schimmel, A. (1975). \emph{The Mystery of Numbers}. Oxford University
  Press.
\item
  Knuth, D. E. (1984). Literate Programming. \emph{Comput. J.}
\item
  Guyot, J. (2023). \emph{Swiss Ephemeris Documentation v.2.1}.
  Astrodienst.
\end{enumerate}


\backmatter


\end{document}
