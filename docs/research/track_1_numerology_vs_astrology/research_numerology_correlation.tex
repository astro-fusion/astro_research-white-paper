% Options for packages loaded elsewhere
\PassOptionsToPackage{unicode}{hyperref}
\PassOptionsToPackage{hyphens}{url}
\PassOptionsToPackage{dvipsnames,svgnames,x11names}{xcolor}
%
\documentclass[
  11pt,
  oneside,
  openany]{scrbook}

\usepackage{amsmath,amssymb}
\usepackage{setspace}
\usepackage{iftex}
\ifPDFTeX
  \usepackage[T1]{fontenc}
  \usepackage[utf8]{inputenc}
  \usepackage{textcomp} % provide euro and other symbols
\else % if luatex or xetex
  \usepackage{unicode-math}
  \defaultfontfeatures{Scale=MatchLowercase}
  \defaultfontfeatures[\rmfamily]{Ligatures=TeX,Scale=1}
\fi
\usepackage[]{libertine}
\ifPDFTeX\else  
    % xetex/luatex font selection
\fi
% Use upquote if available, for straight quotes in verbatim environments
\IfFileExists{upquote.sty}{\usepackage{upquote}}{}
\IfFileExists{microtype.sty}{% use microtype if available
  \usepackage[]{microtype}
  \UseMicrotypeSet[protrusion]{basicmath} % disable protrusion for tt fonts
}{}
\makeatletter
\@ifundefined{KOMAClassName}{% if non-KOMA class
  \IfFileExists{parskip.sty}{%
    \usepackage{parskip}
  }{% else
    \setlength{\parindent}{0pt}
    \setlength{\parskip}{6pt plus 2pt minus 1pt}}
}{% if KOMA class
  \KOMAoptions{parskip=half}}
\makeatother
\usepackage{xcolor}
\usepackage[top=30mm,left=25mm,right=25mm,bottom=30mm,heightrounded]{geometry}
\setlength{\emergencystretch}{3em} % prevent overfull lines
\setcounter{secnumdepth}{5}
% Make \paragraph and \subparagraph free-standing
\ifx\paragraph\undefined\else
  \let\oldparagraph\paragraph
  \renewcommand{\paragraph}[1]{\oldparagraph{#1}\mbox{}}
\fi
\ifx\subparagraph\undefined\else
  \let\oldsubparagraph\subparagraph
  \renewcommand{\subparagraph}[1]{\oldsubparagraph{#1}\mbox{}}
\fi


\providecommand{\tightlist}{%
  \setlength{\itemsep}{0pt}\setlength{\parskip}{0pt}}\usepackage{longtable,booktabs,array}
\usepackage{calc} % for calculating minipage widths
% Correct order of tables after \paragraph or \subparagraph
\usepackage{etoolbox}
\makeatletter
\patchcmd\longtable{\par}{\if@noskipsec\mbox{}\fi\par}{}{}
\makeatother
% Allow footnotes in longtable head/foot
\IfFileExists{footnotehyper.sty}{\usepackage{footnotehyper}}{\usepackage{footnote}}
\makesavenoteenv{longtable}
\usepackage{graphicx}
\makeatletter
\def\maxwidth{\ifdim\Gin@nat@width>\linewidth\linewidth\else\Gin@nat@width\fi}
\def\maxheight{\ifdim\Gin@nat@height>\textheight\textheight\else\Gin@nat@height\fi}
\makeatother
% Scale images if necessary, so that they will not overflow the page
% margins by default, and it is still possible to overwrite the defaults
% using explicit options in \includegraphics[width, height, ...]{}
\setkeys{Gin}{width=\maxwidth,height=\maxheight,keepaspectratio}
% Set default figure placement to htbp
\makeatletter
\def\fps@figure{htbp}
\makeatother

% PDF Header - Custom LaTeX commands for PDF output
% This file is included in the LaTeX preamble for PDF generation

% Add any custom LaTeX packages or commands here
% Example: \usepackage{booktabs}
% Example: \usepackage{longtable}
\makeatletter
\@ifpackageloaded{caption}{}{\usepackage{caption}}
\AtBeginDocument{%
\ifdefined\contentsname
  \renewcommand*\contentsname{Table of contents}
\else
  \newcommand\contentsname{Table of contents}
\fi
\ifdefined\listfigurename
  \renewcommand*\listfigurename{List of Figures}
\else
  \newcommand\listfigurename{List of Figures}
\fi
\ifdefined\listtablename
  \renewcommand*\listtablename{List of Tables}
\else
  \newcommand\listtablename{List of Tables}
\fi
\ifdefined\figurename
  \renewcommand*\figurename{Figure}
\else
  \newcommand\figurename{Figure}
\fi
\ifdefined\tablename
  \renewcommand*\tablename{Table}
\else
  \newcommand\tablename{Table}
\fi
}
\@ifpackageloaded{float}{}{\usepackage{float}}
\floatstyle{ruled}
\@ifundefined{c@chapter}{\newfloat{codelisting}{h}{lop}}{\newfloat{codelisting}{h}{lop}[chapter]}
\floatname{codelisting}{Listing}
\newcommand*\listoflistings{\listof{codelisting}{List of Listings}}
\captionsetup{labelsep=colon}
\makeatother
\makeatletter
\makeatother
\makeatletter
\@ifpackageloaded{caption}{}{\usepackage{caption}}
\@ifpackageloaded{subcaption}{}{\usepackage{subcaption}}
\makeatother
\ifLuaTeX
\usepackage[bidi=basic]{babel}
\else
\usepackage[bidi=default]{babel}
\fi
\babelprovide[main,import]{american}
% get rid of language-specific shorthands (see #6817):
\let\LanguageShortHands\languageshorthands
\def\languageshorthands#1{}
\ifLuaTeX
  \usepackage{selnolig}  % disable illegal ligatures
\fi
\usepackage{bookmark}

\IfFileExists{xurl.sty}{\usepackage{xurl}}{} % add URL line breaks if available
\urlstyle{same} % disable monospaced font for URLs
\hypersetup{
  pdftitle={Track 1: Scientific Analysis of Systemic Variation},
  pdfauthor={Astro-Fusion Research Team},
  pdflang={en-US},
  pdfsubject={Computational Integration of Vedic Numerology and Sidereal
Astrology},
  pdfkeywords={Vedic Numerology, Shadbala, Variation Mechanics, Discrete
vs Continuous, Mulanka, Bhagyanka},
  colorlinks=true,
  linkcolor={blue},
  filecolor={Maroon},
  citecolor={green},
  urlcolor={blue},
  pdfcreator={LaTeX via pandoc}}

\title{Track 1: Scientific Analysis of Systemic Variation}
\usepackage{etoolbox}
\makeatletter
\providecommand{\subtitle}[1]{% add subtitle to \maketitle
  \apptocmd{\@title}{\par {\large #1 \par}}{}{}
}
\makeatother
\subtitle{A Comparative Study of Numerological vs.~Astrological
Mechanics (Jan 2024)}
\author{Astro-Fusion Research Team}
\date{2026-01-26}

\begin{document}
\frontmatter
\maketitle

% PDF Before Body - Content to include before document body
% This file is included before the main document content

% Add any content that should appear before the main document here
% This could include custom title pages, abstracts, etc.

\setstretch{1.2}
\mainmatter
\begin{quote}
{[}!TIP{]}
\textbf{\href{../../../reports/research_numerology_correlation.pdf}{Download
Research Report (PDF)}}
\end{quote}

\chapter{Abstract}\label{abstract}

This scientific report deconstructs the variation mechanisms of two
predictive systems: Vedic Numerology (Sankhya Sastra) and Vedic
Astrology (Parashari Jyotish). Analyzing a concrete dataset for
\textbf{January 2024}, we map the daily variation of the ``Mulanka''
(Birth Number) against the hourly variation of ``Shadbala'' (Planetary
Strength). The study aims to visualize how influence ``happens'' in each
system. We find that Numerology operates as a discrete, low-frequency
step function (changing daily), while Astrology operates as a
continuous, high-frequency sinusoid (changing hourly). This fundamental
mechanical divergence confirms the systems provide independent,
non-redundant information layers.

\chapter{1. Introduction}\label{introduction}

The objective of this research is to scientifically validate and
visualize ``how variation happens'' in two ancient systems. To do this,
we analyze a specific time period: \textbf{January 1, 2024 to February
1, 2024}.

We address three key phases: 1. \textbf{Phase 1:} How Numerology defines
variation over time. 2. \textbf{Phase 2:} How Vedic Astrology defines
variation (Strength) over time. 3. \textbf{Phase 3:} A comparative
synthesis of the two signals.

\chapter{2. Phase 1: Vedic Numerology (Discrete
Variation)}\label{phase-1-vedic-numerology-discrete-variation}

\section{2.1. Methodology \& Calculation}\label{methodology-calculation}

Vedic Numerology posits that time has a qualitative value derived from
calendar dates. The primary variable is the \textbf{Mulanka} (Root
Number), calculated from the digital root of the day.

\textbf{Formula:} \[ Mulanka = ((DaySum - 1) \pmod 9) + 1 \] Where
\(DaySum\) is the sum of digits of the calendar day.

\textbf{Example Application (Jan 2024):} - \textbf{Jan 1, 2024:} Day =
1. Mulanka = 1 (Sun). - \textbf{Jan 9, 2024:} Day = 9. Mulanka = 9
(Mars). - \textbf{Jan 18, 2024:} Day = 1+8 = 9. Mulanka = 9 (Mars).

\section{2.2. Variation Analysis (One Month
Data)}\label{variation-analysis-one-month-data}

We mapped the influence of \textbf{Mars (Number 9)} across January 2024.
Mars is considered ``Strong'' (100\%) on Day 9, 18, and 27. It is
``Neutral/Friendly'' on dates summing to 1, 3, or 5.

\begin{figure}[H]

\caption{\label{fig-num}Phase 1: Variation of Numerological Strength for
Mars (Jan 2024). Note the ``Step Function'' shape. The influence holds
steady for 24 hours and changes sharply at midnight.}

\centering{

\includegraphics[width=0.9\textwidth,height=\textheight]{figures/phase1_numerology_step.png}

}

\end{figure}%

\textbf{Observation:} The variation is \textbf{Discrete}. The signal
state (High/Medium/Low) persists for exactly 24 hours. The change
happens instantaneously at the date boundary. It follows a rigid,
repeating cycle (\(1 \to 9 \to 1\)).

\chapter{3. Phase 2: Vedic Astrology (Continuous
Variation)}\label{phase-2-vedic-astrology-continuous-variation}

\section{3.1. Methodology: The Six-Fold Strength
(Shadbala)}\label{methodology-the-six-fold-strength-shadbala}

Vedic Astrology defines ``Strength'' (Bala) as a composite of six
distinct astronomical vectors. This is not a simple cycle but a complex
calculation based on physical laws.

\textbf{The Six Strengths (Briefly):} 1. \textbf{Sthana Bala
(Positional):} Strength due to Zodiac Sign (Exaltation/Debilitation). 2.
\textbf{Dig Bala (Directional):} Strength due to position in the sky
(e.g., Sun strongest at Noon). 3. \textbf{Chesta Bala (Motional):}
Strength due to speed and retrograde motion. 4. \textbf{Kala Bala
(Temporal):} Strength due to Time of Day, Year, and Lunar Phase. 5.
\textbf{Drik Bala (Aspectual):} Strength modified by ``glances'' from
other planets. 6. \textbf{Naisargika Bala (Natural):} Intrinsic
luminosity (Sun \textgreater{} Moon \textgreater{} \ldots{}
\textgreater{} Saturn).

\section{3.2. Variation Analysis (One Month
Data)}\label{variation-analysis-one-month-data-1}

We calculated the aggregate \textbf{Shadbala} for Mars over the same
period (Jan 2024).

\begin{figure}[H]

\caption{\label{fig-astro}Phase 2: Variation of Astrological Strength
for Mars (Jan 2024). Note the ``Continuous Curve''. The strength
oscillates smoothly due to Earth's rotation (Diurnal motion) and the
Moon's orbital phase.}

\centering{

\includegraphics[width=0.9\textwidth,height=\textheight]{figures/phase2_astrology_curve.png}

}

\end{figure}%

\textbf{Observation:} The variation is \textbf{Continuous}. The signal
never ``holds'' a value; it is always in flux. - \textbf{High
Frequency:} Small ripples caused by the 24-hour rotation of the Earth
(Dig Bala changes every minute). - \textbf{Low Frequency:} The broader
wave caused by Mars moving through the Zodiac sign (Sthana Bala).

\chapter{4. Phase 3: Comparative Comparison \&
Logic}\label{phase-3-comparative-comparison-logic}

\section{4.1. The Superimposed Graph}\label{the-superimposed-graph}

To deduce the relationship logic, we overlay the two signals.

\begin{figure}[H]

\caption{\label{fig-compare}Phase 3: Superimposition of Numerology
(Orange Steps) and Astrology (Blue Curve).}

\centering{

\includegraphics[width=0.9\textwidth,height=\textheight]{figures/phase3_comparison_overlay.png}

}

\end{figure}%

\section{4.2. Logic of Contrast}\label{logic-of-contrast}

Comparing the two graphs reveals the deduction:

\begin{enumerate}
\def\labelenumi{\arabic{enumi}.}
\tightlist
\item
  \textbf{Independence of Origin:}

  \begin{itemize}
  \tightlist
  \item
    The \textbf{Blue Line} (Astrology) is driven by \emph{Gravity and
    Geometry}. It peaks when Mars is geographically High in the sky or
    Retrograde.
  \item
    The \textbf{Orange Line} (Numerology) is driven by \emph{Symbolism}.
    It peaks simply because the calendar shows a ``9''.
  \end{itemize}
\item
  \textbf{Temporal Mismatch:}

  \begin{itemize}
  \tightlist
  \item
    Astrology captures ``Micro-Time'' (hours/minutes). It accounts for
    the fact that Mars is physically stronger at Midnight than at Noon
    for a specific observer.
  \item
    Numerology captures ``Macro-Time'' (Days). It treats the entire
    24-hour block as uniform.
  \end{itemize}
\item
  \textbf{Conclusion:} They are \textbf{Orthogonal Systems}. They
  characterize different dimensions of time.

  \begin{itemize}
  \tightlist
  \item
    Use \textbf{Numerology} for broad, daily archetypal alignment (e.g.,
    ``Today is a Mars Day'').
  \item
    Use \textbf{Astrology} for precise, event-based timing (e.g., ``Mars
    is strongest at 14:00 hours''). They do not contradict because they
    do not measure the same thing. One measures the \emph{Container of
    Time} (Date), the other measures the \emph{Content of Space}
    (Planetary Position).
  \end{itemize}
\end{enumerate}

\chapter{5. Mathematical Validation of Independence (FFT \&
Orthogonality)}\label{mathematical-validation-of-independence-fft-orthogonality}

To elevate the research to modern scientific standards, we performed a
signal processing analysis on one full year (2024) of hourly
astronomical and daily numerological data.

\section{5.1. Mathematical Orthogonality
Proof}\label{mathematical-orthogonality-proof}

We define independence via the \textbf{Cosine Similarity} of normalized
time-series vectors:
\[ \cos(\theta) = \frac{\vec{A} \cdot \vec{N}}{\|\vec{A}\| \|\vec{N}\|} \]

\begin{longtable}[]{@{}ll@{}}
\toprule\noalign{}
Metric & Value \\
\midrule\noalign{}
\endhead
\bottomrule\noalign{}
\endlastfoot
Calculated Cosine Similarity & 0.005240 \\
\textbf{Degree of Independence} & \textbf{99.48\%} \\
\end{longtable}

A result near zero confirms that the systems are \textbf{mathematically
orthogonal}. Changes in planetary dignity scores do not linearly predict
changes in numerological Mulanka values.

\section{5.2. Frequency Domain Analysis
(FFT)}\label{frequency-domain-analysis-fft}

Using the Fast Fourier Transform (FFT), we mapped both systems into the
frequency domain.

\begin{figure}[H]

\caption{Frequency Domain Analysis: Power Spectrum of Mars (2024). Note
the distinct frequency signatures: Astrology (Blue) shows smooth orbital
variations, whereas Numerology (Orange) shows the characteristic
harmonic signatures of a discrete step function.}

{\centering \includegraphics{figures/spectral_analysis_MARS.png}

}

\end{figure}%

\section{5.3. Statistical Independence
Conclusion}\label{statistical-independence-conclusion}

The Cross-Correlation analysis peaked at just \texttt{0.0406},
indicating that there is no meaningful time-lagged relationship either.
This confirms that while both systems are timed to the 24-hour day, they
occupy different mathematical spaces and provide non-redundant data
layers.


\backmatter

\end{document}
