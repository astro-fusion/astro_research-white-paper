% Options for packages loaded elsewhere
% Options for packages loaded elsewhere
\PassOptionsToPackage{unicode}{hyperref}
\PassOptionsToPackage{hyphens}{url}
\PassOptionsToPackage{dvipsnames,svgnames,x11names}{xcolor}
%
\documentclass[
  american,
  11pt,
  a4paper,
  oneside,
  openany]{article}
\usepackage{xcolor}
\usepackage[top=30mm,left=25mm,right=25mm,bottom=30mm,heightrounded,top=25mm,bottom=25mm]{geometry}
\usepackage{amsmath,amssymb}
\setcounter{secnumdepth}{5}
\usepackage{iftex}
\ifPDFTeX
  \usepackage[T1]{fontenc}
  \usepackage[utf8]{inputenc}
  \usepackage{textcomp} % provide euro and other symbols
\else % if luatex or xetex
  \usepackage{unicode-math} % this also loads fontspec
  \defaultfontfeatures{Scale=MatchLowercase}
  \defaultfontfeatures[\rmfamily]{Ligatures=TeX,Scale=1}
\fi
\usepackage[]{libertine}
\ifPDFTeX\else
  % xetex/luatex font selection
\fi
% Use upquote if available, for straight quotes in verbatim environments
\IfFileExists{upquote.sty}{\usepackage{upquote}}{}
\IfFileExists{microtype.sty}{% use microtype if available
  \usepackage[]{microtype}
  \UseMicrotypeSet[protrusion]{basicmath} % disable protrusion for tt fonts
}{}
\usepackage{setspace}
\makeatletter
\@ifundefined{KOMAClassName}{% if non-KOMA class
  \IfFileExists{parskip.sty}{%
    \usepackage{parskip}
  }{% else
    \setlength{\parindent}{0pt}
    \setlength{\parskip}{6pt plus 2pt minus 1pt}}
}{% if KOMA class
  \KOMAoptions{parskip=half}}
\makeatother
% Make \paragraph and \subparagraph free-standing
\makeatletter
\ifx\paragraph\undefined\else
  \let\oldparagraph\paragraph
  \renewcommand{\paragraph}{
    \@ifstar
      \xxxParagraphStar
      \xxxParagraphNoStar
  }
  \newcommand{\xxxParagraphStar}[1]{\oldparagraph*{#1}\mbox{}}
  \newcommand{\xxxParagraphNoStar}[1]{\oldparagraph{#1}\mbox{}}
\fi
\ifx\subparagraph\undefined\else
  \let\oldsubparagraph\subparagraph
  \renewcommand{\subparagraph}{
    \@ifstar
      \xxxSubParagraphStar
      \xxxSubParagraphNoStar
  }
  \newcommand{\xxxSubParagraphStar}[1]{\oldsubparagraph*{#1}\mbox{}}
  \newcommand{\xxxSubParagraphNoStar}[1]{\oldsubparagraph{#1}\mbox{}}
\fi
\makeatother

\usepackage{color}
\usepackage{fancyvrb}
\newcommand{\VerbBar}{|}
\newcommand{\VERB}{\Verb[commandchars=\\\{\}]}
\DefineVerbatimEnvironment{Highlighting}{Verbatim}{commandchars=\\\{\}}
% Add ',fontsize=\small' for more characters per line
\usepackage{framed}
\definecolor{shadecolor}{RGB}{241,243,245}
\newenvironment{Shaded}{\begin{snugshade}}{\end{snugshade}}
\newcommand{\AlertTok}[1]{\textcolor[rgb]{0.68,0.00,0.00}{#1}}
\newcommand{\AnnotationTok}[1]{\textcolor[rgb]{0.37,0.37,0.37}{#1}}
\newcommand{\AttributeTok}[1]{\textcolor[rgb]{0.40,0.45,0.13}{#1}}
\newcommand{\BaseNTok}[1]{\textcolor[rgb]{0.68,0.00,0.00}{#1}}
\newcommand{\BuiltInTok}[1]{\textcolor[rgb]{0.00,0.23,0.31}{#1}}
\newcommand{\CharTok}[1]{\textcolor[rgb]{0.13,0.47,0.30}{#1}}
\newcommand{\CommentTok}[1]{\textcolor[rgb]{0.37,0.37,0.37}{#1}}
\newcommand{\CommentVarTok}[1]{\textcolor[rgb]{0.37,0.37,0.37}{\textit{#1}}}
\newcommand{\ConstantTok}[1]{\textcolor[rgb]{0.56,0.35,0.01}{#1}}
\newcommand{\ControlFlowTok}[1]{\textcolor[rgb]{0.00,0.23,0.31}{\textbf{#1}}}
\newcommand{\DataTypeTok}[1]{\textcolor[rgb]{0.68,0.00,0.00}{#1}}
\newcommand{\DecValTok}[1]{\textcolor[rgb]{0.68,0.00,0.00}{#1}}
\newcommand{\DocumentationTok}[1]{\textcolor[rgb]{0.37,0.37,0.37}{\textit{#1}}}
\newcommand{\ErrorTok}[1]{\textcolor[rgb]{0.68,0.00,0.00}{#1}}
\newcommand{\ExtensionTok}[1]{\textcolor[rgb]{0.00,0.23,0.31}{#1}}
\newcommand{\FloatTok}[1]{\textcolor[rgb]{0.68,0.00,0.00}{#1}}
\newcommand{\FunctionTok}[1]{\textcolor[rgb]{0.28,0.35,0.67}{#1}}
\newcommand{\ImportTok}[1]{\textcolor[rgb]{0.00,0.46,0.62}{#1}}
\newcommand{\InformationTok}[1]{\textcolor[rgb]{0.37,0.37,0.37}{#1}}
\newcommand{\KeywordTok}[1]{\textcolor[rgb]{0.00,0.23,0.31}{\textbf{#1}}}
\newcommand{\NormalTok}[1]{\textcolor[rgb]{0.00,0.23,0.31}{#1}}
\newcommand{\OperatorTok}[1]{\textcolor[rgb]{0.37,0.37,0.37}{#1}}
\newcommand{\OtherTok}[1]{\textcolor[rgb]{0.00,0.23,0.31}{#1}}
\newcommand{\PreprocessorTok}[1]{\textcolor[rgb]{0.68,0.00,0.00}{#1}}
\newcommand{\RegionMarkerTok}[1]{\textcolor[rgb]{0.00,0.23,0.31}{#1}}
\newcommand{\SpecialCharTok}[1]{\textcolor[rgb]{0.37,0.37,0.37}{#1}}
\newcommand{\SpecialStringTok}[1]{\textcolor[rgb]{0.13,0.47,0.30}{#1}}
\newcommand{\StringTok}[1]{\textcolor[rgb]{0.13,0.47,0.30}{#1}}
\newcommand{\VariableTok}[1]{\textcolor[rgb]{0.07,0.07,0.07}{#1}}
\newcommand{\VerbatimStringTok}[1]{\textcolor[rgb]{0.13,0.47,0.30}{#1}}
\newcommand{\WarningTok}[1]{\textcolor[rgb]{0.37,0.37,0.37}{\textit{#1}}}

\usepackage{longtable,booktabs,array}
\usepackage{calc} % for calculating minipage widths
% Correct order of tables after \paragraph or \subparagraph
\usepackage{etoolbox}
\makeatletter
\patchcmd\longtable{\par}{\if@noskipsec\mbox{}\fi\par}{}{}
\makeatother
% Allow footnotes in longtable head/foot
\IfFileExists{footnotehyper.sty}{\usepackage{footnotehyper}}{\usepackage{footnote}}
\makesavenoteenv{longtable}
\usepackage{graphicx}
\makeatletter
\newsavebox\pandoc@box
\newcommand*\pandocbounded[1]{% scales image to fit in text height/width
  \sbox\pandoc@box{#1}%
  \Gscale@div\@tempa{\textheight}{\dimexpr\ht\pandoc@box+\dp\pandoc@box\relax}%
  \Gscale@div\@tempb{\linewidth}{\wd\pandoc@box}%
  \ifdim\@tempb\p@<\@tempa\p@\let\@tempa\@tempb\fi% select the smaller of both
  \ifdim\@tempa\p@<\p@\scalebox{\@tempa}{\usebox\pandoc@box}%
  \else\usebox{\pandoc@box}%
  \fi%
}
% Set default figure placement to htbp
\def\fps@figure{htbp}
\makeatother



\ifLuaTeX
\usepackage[bidi=basic]{babel}
\else
\usepackage[bidi=default]{babel}
\fi
% get rid of language-specific shorthands (see #6817):
\let\LanguageShortHands\languageshorthands
\def\languageshorthands#1{}
\ifLuaTeX
  \usepackage[english]{selnolig} % disable illegal ligatures
\fi


\setlength{\emergencystretch}{3em} % prevent overfull lines

\providecommand{\tightlist}{%
  \setlength{\itemsep}{0pt}\setlength{\parskip}{0pt}}



 


% PDF Header - Custom LaTeX commands for PDF output
% This file is included in the LaTeX preamble for PDF generation

% Add any custom LaTeX packages or commands here
% Example: \usepackage{booktabs}
% Example: \usepackage{longtable}
\makeatletter
\@ifpackageloaded{caption}{}{\usepackage{caption}}
\AtBeginDocument{%
\ifdefined\contentsname
  \renewcommand*\contentsname{Table of contents}
\else
  \newcommand\contentsname{Table of contents}
\fi
\ifdefined\listfigurename
  \renewcommand*\listfigurename{List of Figures}
\else
  \newcommand\listfigurename{List of Figures}
\fi
\ifdefined\listtablename
  \renewcommand*\listtablename{List of Tables}
\else
  \newcommand\listtablename{List of Tables}
\fi
\ifdefined\figurename
  \renewcommand*\figurename{Figure}
\else
  \newcommand\figurename{Figure}
\fi
\ifdefined\tablename
  \renewcommand*\tablename{Table}
\else
  \newcommand\tablename{Table}
\fi
}
\@ifpackageloaded{float}{}{\usepackage{float}}
\floatstyle{ruled}
\@ifundefined{c@chapter}{\newfloat{codelisting}{h}{lop}}{\newfloat{codelisting}{h}{lop}[chapter]}
\floatname{codelisting}{Listing}
\newcommand*\listoflistings{\listof{codelisting}{List of Listings}}
\captionsetup{labelsep=colon}
\makeatother
\makeatletter
\makeatother
\makeatletter
\@ifpackageloaded{caption}{}{\usepackage{caption}}
\@ifpackageloaded{subcaption}{}{\usepackage{subcaption}}
\makeatother
\usepackage{bookmark}
\IfFileExists{xurl.sty}{\usepackage{xurl}}{} % add URL line breaks if available
\urlstyle{same}
\hypersetup{
  pdftitle={Temporal Discontinuity Between Vedic Numerology and Astrology: A Quantitative Analysis of System Independence},
  pdfauthor={Bishal Ghimire},
  pdflang={en-US},
  pdfsubject={Computational Integration of Vedic Numerology and Sidereal
Astrology},
  pdfkeywords={Vedic Numerology, Vedic Astrology, Temporal
Discontinuity, Shadbala, Nakshatra, Cosine Similarity, Swiss Ephemeris},
  colorlinks=true,
  linkcolor={blue},
  filecolor={Maroon},
  citecolor={green},
  urlcolor={blue},
  pdfcreator={LaTeX via pandoc}}


\title{Temporal Discontinuity Between Vedic Numerology and Astrology: A
Quantitative Analysis of System Independence}
\usepackage{etoolbox}
\makeatletter
\providecommand{\subtitle}[1]{% add subtitle to \maketitle
  \apptocmd{\@title}{\par {\large #1 \par}}{}{}
}
\makeatother
\subtitle{Demonstrating Fundamental Differences in Temporal Dynamics
Between Discrete and Continuous Predictive Systems}
\author{Astro-Fusion Research Team}
\date{February 01, 2026}
\begin{document}
\maketitle
\begin{abstract}
This study presents a rigorous quantitative investigation into the
temporal relationship between two ancient Vedic predictive systems:
Vedic Numerology (Anka Jyotish) and Vedic Astrology (Parashari Jyotish).
Despite sharing a common mythological lineage where numbers are mapped
to planetary deities, our computational analysis reveals a fundamental
temporal discontinuity. Using the Swiss Ephemeris (DE440) for
sub-arcsecond precision astronomical calculations, we demonstrate that
Numerology operates on a discrete, low-frequency temporal grid (changing
\textasciitilde73 times annually), while Astrology functions as a
high-frequency continuous system (Moon changes Nakshatra
\textasciitilde27 times monthly). Mathematical analysis reveals 99.48\%
orthogonality between the normalized time series, and Nakshatra-Day
mapping shows synchronization rates (11.23\%) statistically
indistinguishable from random chance (11.11\%). These findings establish
that despite mythological connections, Vedic Numerology and Vedic
Astrology should be treated as distinct, non-interchangeable predictive
frameworks providing independent information dimensions.
\end{abstract}

% PDF Before Body - Content to include before document body
% This file is included before the main document content

% Add any content that should appear before the main document here
% This could include custom title pages, abstracts, etc.

\renewcommand*\contentsname{Table of contents}
{
\hypersetup{linkcolor=}
\setcounter{tocdepth}{3}
\tableofcontents
}
\listoffigures
\listoftables

\setstretch{1.2}
\section{Introduction}\label{introduction}

\subsection{The Convergence Question}\label{the-convergence-question}

The intersection of mathematics and mythology forms the backbone of
ancient predictive sciences. In the Vedic tradition, the cosmos is
viewed not as a random assembly of matter but as a conscious,
interconnected system governed by \textbf{Grahas} (planets) which act as
agents of karma. Two primary systems evolved to interpret these
influences:

\begin{enumerate}
\def\labelenumi{\arabic{enumi}.}
\tightlist
\item
  \textbf{Vedic Astrology (Jyotish)}: Relies on continuous astronomical
  position of celestial bodies
\item
  \textbf{Vedic Numerology (Anka Jyotish)}: Abstracts movements into
  discrete integer values based on calendar dates
\end{enumerate}

A fundamental question persists: \textbf{Are these systems measuring the
same underlying reality, or do they represent independent information
dimensions?}

\subsection{Research Problem}\label{research-problem}

Practitioners often assume that if a person is in a ``Sun period'' in
numerology (e.g., a date summing to 1), the astrological Sun must also
be strong or prominent. \textbf{This study challenges that assumption.}
We propose that the algorithms driving these systems are fundamentally
mismatched in the time domain, leading to a ``Temporal Discontinuity''
where a planet can be numerologically ``King'' while astrologically
``Debilitated.''

\subsection{Objectives}\label{objectives}

\begin{enumerate}
\def\labelenumi{\arabic{enumi}.}
\tightlist
\item
  Quantify the temporal resolution differences between systems
\item
  Measure mathematical independence using signal processing techniques
\item
  Test synchronization between Nakshatra Lords and Mulanka planets
\item
  Establish whether these systems provide redundant or independent
  information
\end{enumerate}

\section{Mythological and Archetypal
Foundations}\label{mythological-and-archetypal-foundations}

To understand why these systems are often conflated, one must examine
their shared mythological roots. Each number in Vedic numerology
represents a planetary deity's energy pattern:

\begin{longtable}[]{@{}llll@{}}
\caption{Vedic Numerology Planetary
Associations}\label{tbl-associations}\tabularnewline
\toprule\noalign{}
Number & Planet & Sanskrit & Mythological Role \\
\midrule\noalign{}
\endfirsthead
\toprule\noalign{}
Number & Planet & Sanskrit & Mythological Role \\
\midrule\noalign{}
\endhead
\bottomrule\noalign{}
\endlastfoot
1 & Sun & Surya & Soul (Atman), King, Ego \\
2 & Moon & Chandra & Mind (Manas), Queen, Emotions \\
3 & Jupiter & Brihaspati & Guru of Devas, Wisdom, Expansion \\
4 & Rahu & - & North Node, Illusion (Maya) \\
5 & Mercury & Budha & Prince, Intellect (Buddhi) \\
6 & Venus & Shukra & Guru of Asuras, Desire (Kama) \\
7 & Ketu & - & South Node, Liberation (Moksha) \\
8 & Saturn & Shani & Judge (Karmakaraka), Discipline \\
9 & Mars & Mangal & Commander, Energy (Shakti) \\
\end{longtable}

This shared planetary pantheon creates an apparent connection, but the
\textbf{mathematical algorithms} generating planetary influences differ
fundamentally.

\section{Methodology}\label{methodology}

\subsection{Data Sources and Tools}\label{data-sources-and-tools}

\begin{itemize}
\tightlist
\item
  \textbf{Astronomical Engine}: Swiss Ephemeris v2.10 (DE440/DE441 JPL
  ephemeris)
\item
  \textbf{Ayanamsa}: Lahiri (Chitra Paksha) for sidereal correction
\item
  \textbf{Study Period}: January 1, 2024 to December 31, 2024 (365 days)
\item
  \textbf{Astrological Resolution}: 2-hour intervals (4,380 data points
  per planet)
\item
  \textbf{Numerological Resolution}: Daily values (365 data points)
\item
  \textbf{Reference Location}: New Delhi, India (28.6°N, 77.1°E)
\end{itemize}

\subsection{Numerology Algorithm: Modulo-9
Arithmetic}\label{numerology-algorithm-modulo-9-arithmetic}

Vedic numerology uses a base-9 system. The \textbf{Mulanka} (Root
Number) is the most rapidly changing daily indicator, derived from the
day of the month using digital root summation:

\[
Mulanka = ((Day - 1) \bmod 9) + 1
\]

This function is a \textbf{discrete step function} that holds a constant
integer value for 24 hours.

\subsubsection{Example Calculations}\label{example-calculations}

\begin{itemize}
\tightlist
\item
  \textbf{January 1}: Day = 1 → Mulanka = 1 (Sun)
\item
  \textbf{January 9}: Day = 9 → Mulanka = 9 (Mars)
\item
  \textbf{January 18}: Day = 18 → 1+8 = 9 → Mulanka = 9 (Mars)
\item
  \textbf{January 27}: Day = 27 → 2+7 = 9 → Mulanka = 9 (Mars)
\end{itemize}

\subsection{Astrology Algorithm: Continuous Celestial
Mechanics}\label{astrology-algorithm-continuous-celestial-mechanics}

\subsubsection{Planetary Position
Calculation}\label{planetary-position-calculation}

Planetary longitude \(\lambda_p(t)\) at time \(t\) involves:

\begin{enumerate}
\def\labelenumi{\arabic{enumi}.}
\tightlist
\item
  \textbf{Heliocentric Calculation}: Position vector from Sun to Planet
\item
  \textbf{Geocentric Conversion}: Adjustment for Earth's position
\item
  \textbf{Sidereal Adjustment}: Subtraction of Ayanamsa
\end{enumerate}

\[
\lambda_{sidereal}(t) = \lambda_{tropical}(t) - \alpha_{Lahiri}(t)
\]

where \(\alpha_{Lahiri} \approx 24°\) in 2024.

\subsubsection{Shadbala (Six-Fold
Strength)}\label{shadbala-six-fold-strength}

Planetary strength \(\sigma_p\) is computed as:

\[
\sigma_p = \sum_{i=1}^{6} w_i \cdot B_i
\]

The six components are:

\begin{enumerate}
\def\labelenumi{\arabic{enumi}.}
\tightlist
\item
  \textbf{Sthana Bala}: Positional strength (sign placement)
\item
  \textbf{Dig Bala}: Directional strength (house position)
\item
  \textbf{Kala Bala}: Temporal strength (day/night, hora)
\item
  \textbf{Chesta Bala}: Motional strength (speed, retrograde)
\item
  \textbf{Naisargika Bala}: Natural luminosity
\item
  \textbf{Drik Bala}: Aspectual strength (benefic/malefic glances)
\end{enumerate}

\subsubsection{Nakshatra Calculation}\label{nakshatra-calculation}

The Moon traverses the zodiac in \textasciitilde27.3 days. The zodiac is
divided into 27 \textbf{Nakshatras} (Lunar Mansions) of 13°20' each:

\[
N = \lfloor \frac{\lambda_{Moon}}{13.333°} \rfloor
\]

Each Nakshatra has a ruling planet in the sequence: Ketu → Venus → Sun →
Moon → Mars → Rahu → Jupiter → Saturn → Mercury (repeating).

\subsection{Statistical Methods}\label{statistical-methods}

\subsubsection{Cosine Similarity for
Independence}\label{cosine-similarity-for-independence}

Vector independence is quantified using cosine similarity:

\[
\cos(\theta) = \frac{\vec{A} \cdot \vec{N}}{\|\vec{A}\| \|\vec{N}\|}
\]

where \(\vec{A}\) is the normalized astrology time series and
\(\vec{N}\) is the normalized numerology time series.

\subsubsection{Cross-Correlation
Analysis}\label{cross-correlation-analysis}

Time-lagged relationships are tested via cross-correlation:

\[
R(\tau) = \frac{\sum_t (A_t - \bar{A})(N_{t+\tau} - \bar{N})}{\sigma_A \sigma_N}
\]

\section{Results}\label{results}

\subsection{Temporal Resolution
Comparison}\label{temporal-resolution-comparison}

The fundamental observation is a \textbf{120:1 frequency mismatch}:

\begin{longtable}[]{@{}llll@{}}
\caption{Temporal Resolution
Comparison}\label{tbl-resolution}\tabularnewline
\toprule\noalign{}
System & Resolution & Annual Changes & Data Points (2024) \\
\midrule\noalign{}
\endfirsthead
\toprule\noalign{}
System & Resolution & Annual Changes & Data Points (2024) \\
\midrule\noalign{}
\endhead
\bottomrule\noalign{}
\endlastfoot
Vedic Astrology & 2-hour & 4,380 & 52,560 (all planets) \\
Vedic Numerology & 24-hour & 73 & 365 \\
\textbf{Ratio} & - & \textbf{60:1} & \textbf{144:1} \\
\end{longtable}

Astrology captures ``Micro-Time'' (hours/minutes), while Numerology
captures ``Macro-Time'' (days). This creates an inherent barrier to
synchronization.

\subsection{Phase 1: Numerological Variation
Analysis}\label{phase-1-numerological-variation-analysis}

Mapping the influence of Mars (Number 9) across January 2024 reveals the
\textbf{step function} nature of numerology:

\begin{itemize}
\tightlist
\item
  \textbf{Strong (100\%)}: Days 9, 18, 27
\item
  \textbf{Neutral}: Days with digital roots 1, 3, 5
\item
  \textbf{Weak}: Other days
\end{itemize}

The signal state persists for exactly 24 hours, changing instantaneously
at the date boundary.

\subsection{Phase 2: Astrological Variation
Analysis}\label{phase-2-astrological-variation-analysis}

In contrast, Mars's Shadbala score exhibits \textbf{continuous
variation}:

\begin{itemize}
\tightlist
\item
  \textbf{High Frequency Ripples}: 24-hour rotation of Earth (Dig Bala
  changes every minute)
\item
  \textbf{Low Frequency Waves}: Mars moving through zodiac signs (Sthana
  Bala)
\item
  \textbf{Medium Frequency}: Lunar phase influences on Kala Bala
\end{itemize}

The signal never ``holds'' a value; it is always in flux.

\subsection{Phase 3: Mathematical
Orthogonality}\label{phase-3-mathematical-orthogonality}

\subsubsection{Cosine Similarity
Results}\label{cosine-similarity-results}

\begin{longtable}[]{@{}lll@{}}
\caption{Orthogonality Analysis}\label{tbl-orthogonality}\tabularnewline
\toprule\noalign{}
Metric & Value & Interpretation \\
\midrule\noalign{}
\endfirsthead
\toprule\noalign{}
Metric & Value & Interpretation \\
\midrule\noalign{}
\endhead
\bottomrule\noalign{}
\endlastfoot
Cosine Similarity & 0.0052 & Near-zero correlation \\
\textbf{Independence} & \textbf{99.48\%} & Mathematically orthogonal \\
\end{longtable}

Changes in planetary dignity scores do not linearly predict changes in
numerological Mulanka values.

\subsubsection{Cross-Correlation
Analysis}\label{cross-correlation-analysis-1}

\begin{longtable}[]{@{}ll@{}}
\caption{Cross-Correlation Results}\label{tbl-crosscorr}\tabularnewline
\toprule\noalign{}
Metric & Value \\
\midrule\noalign{}
\endfirsthead
\toprule\noalign{}
Metric & Value \\
\midrule\noalign{}
\endhead
\bottomrule\noalign{}
\endlastfoot
Maximum Correlation & 0.0406 \\
Lag at Maximum & 0 days \\
Significance & Not significant \\
\end{longtable}

No meaningful time-lagged relationship exists between the systems.

\subsection{Nakshatra-Day Synchronization
Test}\label{nakshatra-day-synchronization-test}

We tested whether the Mulanka (Root Number) planet of the day matches
the Nakshatra Lord (ruler of Moon's constellation):

\begin{longtable}[]{@{}ll@{}}
\caption{Nakshatra-Day Synchronization
Test}\label{tbl-sync}\tabularnewline
\toprule\noalign{}
Statistic & Value \\
\midrule\noalign{}
\endfirsthead
\toprule\noalign{}
Statistic & Value \\
\midrule\noalign{}
\endhead
\bottomrule\noalign{}
\endlastfoot
Total Days Analyzed & 365 \\
Exact Matches & 41 \\
\textbf{Match Rate} & \textbf{11.23\%} \\
Expected by Chance & 11.11\% (1/9) \\
Chi-Square p-value & 0.89 \\
\end{longtable}

\textbf{Result}: The synchronization rate is statistically
indistinguishable from random chance, confirming no inherent causal link
between the Gregorian date number and actual lunar position.

\subsection{Frequency Domain Analysis
(FFT)}\label{frequency-domain-analysis-fft}

Fast Fourier Transform analysis reveals distinct frequency signatures:

\begin{itemize}
\tightlist
\item
  \textbf{Astrology (Blue)}: Smooth orbital variations with peaks at
  synodic periods
\item
  \textbf{Numerology (Orange)}: Harmonic signatures characteristic of
  discrete step functions
\end{itemize}

The spectral profiles occupy different frequency bands, confirming
temporal incompatibility.

\section{Discussion}\label{discussion}

\subsection{Logic of Contrast}\label{logic-of-contrast}

Comparing the two systems reveals fundamental differences:

\subsubsection{Independence of Origin}\label{independence-of-origin}

\begin{itemize}
\tightlist
\item
  \textbf{Astrology}: Driven by \emph{Gravity and Geometry}. Peaks when
  Mars is high in the sky or retrograde.
\item
  \textbf{Numerology}: Driven by \emph{Symbolism}. Peaks simply because
  the calendar shows a ``9''.
\end{itemize}

\subsubsection{Temporal Mismatch}\label{temporal-mismatch}

\begin{itemize}
\tightlist
\item
  \textbf{Astrology}: Captures ``Micro-Time'' (hours/minutes). Mars is
  physically stronger at midnight than noon for a specific observer.
\item
  \textbf{Numerology}: Captures ``Macro-Time'' (Days). The entire
  24-hour block is treated as uniform.
\end{itemize}

\subsubsection{Orthogonal Systems}\label{orthogonal-systems}

They are \textbf{Orthogonal Systems} characterizing different dimensions
of time:

\begin{itemize}
\tightlist
\item
  Use \textbf{Numerology} for broad, daily archetypal alignment (``Today
  is a Mars Day'')
\item
  Use \textbf{Astrology} for precise, event-based timing (``Mars is
  strongest at 14:00 hours'')
\end{itemize}

They do not contradict because they do not measure the same thing. One
measures the \emph{Container of Time} (Date), the other measures the
\emph{Content of Space} (Planetary Position).

\subsection{The Illusion of
Synchronization}\label{the-illusion-of-synchronization}

When practitioners observe apparent correlations, they are likely
experiencing \textbf{apophenia}---the tendency to perceive meaningful
patterns in random data. Our Monte Carlo-equivalent analysis (comparing
observed match rates to chance) demonstrates that any perceived
synchronization falls within the null distribution.

\subsection{Implications for Practice}\label{implications-for-practice}

\begin{enumerate}
\def\labelenumi{\arabic{enumi}.}
\tightlist
\item
  \textbf{No Mutual Reinforcement}: A ``Strong'' astrological Jupiter
  does not automatically enhance a ``3-day'' in numerology
\item
  \textbf{Complementary Information}: The systems provide independent
  data layers
\item
  \textbf{Responsible Interpretation}: Practitioners should not conflate
  numerological and astrological strength
\end{enumerate}

\section{Conclusions}\label{conclusions}

This study establishes definitive evidence for the temporal
discontinuity between Vedic Numerology and Vedic Astrology:

\begin{enumerate}
\def\labelenumi{\arabic{enumi}.}
\tightlist
\item
  \textbf{Frequency Mismatch}: 120:1 ratio in temporal resolution
  creates inherent incompatibility
\item
  \textbf{Mathematical Orthogonality}: 99.48\% independence between
  normalized time series
\item
  \textbf{Random Synchronization}: Nakshatra-Day matching (11.23\%)
  equals random chance (11.11\%)
\item
  \textbf{Distinct Spectral Signatures}: FFT analysis confirms different
  frequency domains
\item
  \textbf{No Cross-Correlation}: Maximum lag correlation of 0.0406 is
  not significant
\end{enumerate}

\textbf{Final Conclusion}: Despite shared planetary archetypes, Vedic
Numerology and Vedic Astrology should be treated as \textbf{distinct,
non-interchangeable predictive frameworks} providing independent
information about temporal quality.

\section{References}\label{references}

\phantomsection\label{refs}

\section{Appendix A: Detailed Algorithm
Specifications}\label{appendix-a-detailed-algorithm-specifications}

\subsection{Mulanka Calculation
(Python)}\label{mulanka-calculation-python}

\begin{Shaded}
\begin{Highlighting}[]
\KeywordTok{def}\NormalTok{ calculate\_mulanka(day: }\BuiltInTok{int}\NormalTok{) }\OperatorTok{{-}\textgreater{}} \BuiltInTok{int}\NormalTok{:}
    \CommentTok{"""}
\CommentTok{    Calculate Vedic Mulanka from day of month.}
\CommentTok{    }
\CommentTok{    Args:}
\CommentTok{        day: Day of month (1{-}31)}
\CommentTok{    Returns:}
\CommentTok{        Mulanka number (1{-}9)}
\CommentTok{    """}
    \ControlFlowTok{return}\NormalTok{ ((day }\OperatorTok{{-}} \DecValTok{1}\NormalTok{) }\OperatorTok{\%} \DecValTok{9}\NormalTok{) }\OperatorTok{+} \DecValTok{1}
\end{Highlighting}
\end{Shaded}

\subsection{Nakshatra Lord Lookup}\label{nakshatra-lord-lookup}

\begin{Shaded}
\begin{Highlighting}[]
\NormalTok{NAKSHATRA\_LORDS }\OperatorTok{=}\NormalTok{ [}
    \StringTok{"Ketu"}\NormalTok{, }\StringTok{"Venus"}\NormalTok{, }\StringTok{"Sun"}\NormalTok{, }\StringTok{"Moon"}\NormalTok{, }\StringTok{"Mars"}\NormalTok{,}
    \StringTok{"Rahu"}\NormalTok{, }\StringTok{"Jupiter"}\NormalTok{, }\StringTok{"Saturn"}\NormalTok{, }\StringTok{"Mercury"}
\NormalTok{]}

\KeywordTok{def}\NormalTok{ get\_nakshatra\_lord(nakshatra\_index: }\BuiltInTok{int}\NormalTok{) }\OperatorTok{{-}\textgreater{}} \BuiltInTok{str}\NormalTok{:}
    \CommentTok{"""Get ruling planet of a Nakshatra (0{-}26 index)."""}
    \ControlFlowTok{return}\NormalTok{ NAKSHATRA\_LORDS[nakshatra\_index }\OperatorTok{\%} \DecValTok{9}\NormalTok{]}
\end{Highlighting}
\end{Shaded}

\section{Appendix B: Data
Availability}\label{appendix-b-data-availability}

All analysis data and code are available at:
https://github.com/astro-fusion/astro\_research-white-paper

Specific files: -
\texttt{use\_cases/numerology/research\_paper/numerology\_astrology\_correlation.qmd}
- \texttt{src/vedic\_astrology\_core/time\_series.py} -
\texttt{docs/research/track\_1\_numerology\_vs\_astrology/}




\end{document}
