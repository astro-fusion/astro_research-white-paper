% Options for packages loaded elsewhere
\PassOptionsToPackage{unicode}{hyperref}
\PassOptionsToPackage{hyphens}{url}
\PassOptionsToPackage{dvipsnames,svgnames,x11names}{xcolor}
%
\documentclass[
  11pt,
  a4paper,
  oneside,
  openany]{article}

\usepackage{amsmath,amssymb}
\usepackage{setspace}
\usepackage{iftex}
\ifPDFTeX
  \usepackage[T1]{fontenc}
  \usepackage[utf8]{inputenc}
  \usepackage{textcomp} % provide euro and other symbols
\else % if luatex or xetex
  \usepackage{unicode-math}
  \defaultfontfeatures{Scale=MatchLowercase}
  \defaultfontfeatures[\rmfamily]{Ligatures=TeX,Scale=1}
\fi
\usepackage[]{libertine}
\ifPDFTeX\else
    % xetex/luatex font selection
\fi
% Use upquote if available, for straight quotes in verbatim environments
\IfFileExists{upquote.sty}{\usepackage{upquote}}{}
\IfFileExists{microtype.sty}{% use microtype if available
  \usepackage[]{microtype}
  \UseMicrotypeSet[protrusion]{basicmath} % disable protrusion for tt fonts
}{}
\makeatletter
\@ifundefined{KOMAClassName}{% if non-KOMA class
  \IfFileExists{parskip.sty}{%
    \usepackage{parskip}
  }{% else
    \setlength{\parindent}{0pt}
    \setlength{\parskip}{6pt plus 2pt minus 1pt}}
}{% if KOMA class
  \KOMAoptions{parskip=half}}
\makeatother
\usepackage{xcolor}
\usepackage[top=30mm,left=25mm,right=25mm,bottom=30mm,heightrounded,top=25mm,bottom=25mm]{geometry}
\setlength{\emergencystretch}{3em} % prevent overfull lines
\setcounter{secnumdepth}{5}
% Make \paragraph and \subparagraph free-standing
\ifx\paragraph\undefined\else
  \let\oldparagraph\paragraph
  \renewcommand{\paragraph}[1]{\oldparagraph{#1}\mbox{}}
\fi
\ifx\subparagraph\undefined\else
  \let\oldsubparagraph\subparagraph
  \renewcommand{\subparagraph}[1]{\oldsubparagraph{#1}\mbox{}}
\fi


\providecommand{\tightlist}{%
  \setlength{\itemsep}{0pt}\setlength{\parskip}{0pt}}\usepackage{longtable,booktabs,array}
\usepackage{calc} % for calculating minipage widths
% Correct order of tables after \paragraph or \subparagraph
\usepackage{etoolbox}
\makeatletter
\patchcmd\longtable{\par}{\if@noskipsec\mbox{}\fi\par}{}{}
\makeatother
% Allow footnotes in longtable head/foot
\IfFileExists{footnotehyper.sty}{\usepackage{footnotehyper}}{\usepackage{footnote}}
\makesavenoteenv{longtable}
\usepackage{graphicx}
\makeatletter
\def\maxwidth{\ifdim\Gin@nat@width>\linewidth\linewidth\else\Gin@nat@width\fi}
\def\maxheight{\ifdim\Gin@nat@height>\textheight\textheight\else\Gin@nat@height\fi}
\makeatother
% Scale images if necessary, so that they will not overflow the page
% margins by default, and it is still possible to overwrite the defaults
% using explicit options in \includegraphics[width, height, ...]{}
\setkeys{Gin}{width=\maxwidth,height=\maxheight,keepaspectratio}
% Set default figure placement to htbp
\makeatletter
\def\fps@figure{htbp}
\makeatother

% PDF Header - Custom LaTeX commands for PDF output
% This file is included in the LaTeX preamble for PDF generation

% Add any custom LaTeX packages or commands here
% Example: \usepackage{booktabs}
% Example: \usepackage{longtable}
\makeatletter
\@ifpackageloaded{caption}{}{\usepackage{caption}}
\AtBeginDocument{%
\ifdefined\contentsname
  \renewcommand*\contentsname{Table of contents}
\else
  \newcommand\contentsname{Table of contents}
\fi
\ifdefined\listfigurename
  \renewcommand*\listfigurename{List of Figures}
\else
  \newcommand\listfigurename{List of Figures}
\fi
\ifdefined\listtablename
  \renewcommand*\listtablename{List of Tables}
\else
  \newcommand\listtablename{List of Tables}
\fi
\ifdefined\figurename
  \renewcommand*\figurename{Figure}
\else
  \newcommand\figurename{Figure}
\fi
\ifdefined\tablename
  \renewcommand*\tablename{Table}
\else
  \newcommand\tablename{Table}
\fi
}
\@ifpackageloaded{float}{}{\usepackage{float}}
\floatstyle{ruled}
\@ifundefined{c@chapter}{\newfloat{codelisting}{h}{lop}}{\newfloat{codelisting}{h}{lop}[chapter]}
\floatname{codelisting}{Listing}
\newcommand*\listoflistings{\listof{codelisting}{List of Listings}}
\captionsetup{labelsep=colon}
\makeatother
\makeatletter
\makeatother
\makeatletter
\@ifpackageloaded{caption}{}{\usepackage{caption}}
\@ifpackageloaded{subcaption}{}{\usepackage{subcaption}}
\makeatother
\ifLuaTeX
\usepackage[bidi=basic]{babel}
\else
\usepackage[bidi=default]{babel}
\fi
\babelprovide[main,import]{american}
% get rid of language-specific shorthands (see #6817):
\let\LanguageShortHands\languageshorthands
\def\languageshorthands#1{}
\ifLuaTeX
  \usepackage{selnolig}  % disable illegal ligatures
\fi
\usepackage{bookmark}

\IfFileExists{xurl.sty}{\usepackage{xurl}}{} % add URL line breaks if available
\urlstyle{same} % disable monospaced font for URLs
\hypersetup{
  pdftitle={Rigorous Statistical Falsification of Exogenous Planetary Predictors in Global Gold Markets},
  pdfauthor={Computational Astrology Research Group},
  pdflang={en-US},
  pdfsubject={Computational Integration of Vedic Numerology and Sidereal
Astrology},
  pdfkeywords={Financial Econometrics, Signal Processing, Pseudoscience
Demarcation, Granger Causality, Spectral Analysis},
  colorlinks=true,
  linkcolor={blue},
  filecolor={Maroon},
  citecolor={green},
  urlcolor={blue},
  pdfcreator={LaTeX via pandoc}}

\title{Rigorous Statistical Falsification of Exogenous Planetary
Predictors in Global Gold Markets}
\author{Computational Astrology Research Group}
\date{January 30, 2026}

\begin{document}
\maketitle
\begin{abstract}
\textbf{Abstract.} The Efficient Market Hypothesis (EMH) posits that
asset prices reflect all available information, rendering them
unpredictable through historical data. Conversely, ``Financial
Astrology'' claims that deterministic planetary cycles influence market
psychology and price action. This study applies rigorous econometric
signal processing---including Augmented Dickey-Fuller stationarity
tests, Lomb-Scargle spectral analysis, and Vector Autoregression (VAR)
with Bonferroni-corrected Granger Causality tests---to falsify the
hypothesis that planetary positions provide unique information gain for
forecasting XAU/USD (Gold) spot prices. Analyzing 25 years of daily
market data against high-precision geocentric ephemerides, we
demonstrate that apparent correlations fail to exceed the threshold of
statistical significance when adjusted for multiple hypothesis testing.
These findings reinforce the EMH and characterize perceived astrological
influence as apophenia.
\end{abstract}

% PDF Before Body - Content to include before document body
% This file is included before the main document content

% Add any content that should appear before the main document here
% This could include custom title pages, abstracts, etc.

\setstretch{1.2}
\section{Introduction}\label{introduction}

The transformation of raw environmental data into actionable economic
intelligence is a cornerstone of modern quantitative finance. While
macroeconomic indicators (inflation, interest rates) are well-studied,
the statistical validity of alternative cyclic predictors, such as
astronomical phenomena, remains a subject of contentious debate. This
tension---between the \emph{process of discovery} in data science and
the \emph{product of rigor} in scientific reporting---necessitates a
robust methodological framework for evaluation.

This study applies high-dimensional spectral analysis and Granger
causality tests to falsify the hypothesis that planetary orbital
mechanics influence XAU/USD spot prices. By treating ``Financial
Astrology'' not as a mystic art but as a testable signal processing
claim, we subject it to ``Severe Testing'' criteria: establishing a null
hypothesis of zero effect and only rejecting it in the face of
overwhelming statistical evidence.

\section{Methodology}\label{methodology}

\subsection{Data Acquisition and
Preprocessing}\label{data-acquisition-and-preprocessing}

We utilized daily closing prices for Gold (XAU/USD) sourced from COMEX
via Yahoo Finance, spanning the period from January 1, 2000, to the
present. As financial time series are inherently non-stationary, raw
prices were transformed into Logarithmic Returns to stabilize variance
and approximate continuous compounding:

\[ R_t = \ln(P_t) - \ln(P_{t-1}) \]

Planetary positions were calculated using the Swiss Ephemeris (DE440), a
high-precision numerical integration of the solar system. To align the
continuous celestial data with the discrete trading calendar (which
excludes weekends and holidays), planetary positions were sampled at
12:00 UTC on valid trading days only (see
\texttt{src/data/align\_astro\_data.py}).

\subsection{Statistical Framework}\label{statistical-framework}

\subsubsection{Stationarity Testing}\label{stationarity-testing}

A prerequisite for most econometric inferences is stationarity. We
employed the \textbf{Augmented Dickey-Fuller (ADF)} test to verify that
our differenced target variable (Gold Log Returns) and our engineered
features (Planetary Sine/Cosine components) do not possess a unit root.

\subsubsection{Spectral Analysis}\label{spectral-analysis}

To detect potential cyclic signals in the unevenly sampled financial
data, we utilized the \textbf{Lomb-Scargle Periodogram}, which avoids
the spectral leakage issues associated with the standard Fast Fourier
Transform (FFT) on non-contiguous data.

\subsubsection{Granger Causality}\label{granger-causality}

Predictive precedence was evaluated using a \textbf{Vector
Autoregression (VAR)} model. The null hypothesis (\(H_0\)) states that
past values of planetary positions do not contain information that
significantly reduces the forecast error of Gold returns. To mitigate
the ``p-hacking'' risk inherent in testing multiple planets, we applied
the \textbf{Bonferroni Correction}, adjusting our significance threshold
(\(\alpha = 0.05\)) by dividing it by the number of independent
hypotheses tested.

\section{Results}\label{results}

\subsection{Stationarity Validation}\label{stationarity-validation}

The ADF test results (Table Table~\ref{tbl-stationarity}) confirm that
the Log-Return transformation successfully induced stationarity in the
Gold price series.

\begin{longtable}[]{@{}lrrl@{}}

\caption{\label{tbl-stationarity}Stationarity Test Results}

\tabularnewline

\toprule\noalign{}
Variable & ADF Statistic & ADF p-value & Stationary \\
\midrule\noalign{}
\endhead
\bottomrule\noalign{}
\endlastfoot
Gold Log Returns & -31.7893 & 0 & Yes \\
Mars\_Sin & -14.1846 & 0 & No \\
Mars\_Cos & -14.1444 & 0 & No \\
Saturn\_Sin & -7.3582 & 0 & No \\
Saturn\_Cos & 10.8858 & 1 & No \\

\end{longtable}

\subsection{Spectral Analysis}\label{spectral-analysis-1}

The Lomb-Scargle periodogram (Figure~\ref{fig-periodogram}) reveals the
spectral power density of Gold returns.

\begin{figure}[H]

\caption{\label{fig-periodogram}Lomb-Scargle Periodogram of Gold
Log-Returns. The red dashed line indicates the 1\% False Alarm
Probability (FAP) threshold.}

\centering{

\includegraphics[width=0.9\textwidth,height=\textheight]{artifacts/fig_periodogram.pdf}

}

\end{figure}%

If planetary cycles were driving prices, we would expect significant
peaks at known synodic periods (e.g., \textasciitilde29.5 days for Moon,
\textasciitilde88 days for Mercury). The absence of such consistent
peaks above the noise floor suggests a lack of periodic deterministic
forcing.

\subsection{Granger Causality \&
Prediction}\label{granger-causality-prediction}

The VAR analysis tested whether planetary variables Granger-cause Gold
returns. The results, summarized below, show that after Bonferroni
correction:

\begin{longtable}[]{@{}lcccc@{}}
\toprule\noalign{}
Planet & Best Lag & F-Statistic & p-value & Significant? \\
\midrule\noalign{}
\endhead
\bottomrule\noalign{}
\endlastfoot
\textbf{Mars} & 1 & 2.6134 & 0.0735 & No \\
\textbf{Saturn} & 1 & 0.7444 & 0.4751 & No \\
\textbf{Jupiter} & 1 & 0.6917 & 0.5008 & No \\
\end{longtable}

\subsection{Event Prediction (Molchan
Diagram)}\label{event-prediction-molchan-diagram}

The Molchan Diagram (Figure~\ref{fig-molchan}) evaluates the binary
classification skill of using Mars' speed variations to predict extreme
volatility events.

\begin{figure}[H]

\caption{\label{fig-molchan}Molchan Diagram for Mars Speed vs.~Gold
Volatility. The trajectory hugs the diagonal (random guessing),
indicating no information gain over chance.}

\centering{

\includegraphics[width=0.7\textwidth,height=\textheight]{artifacts/fig_molchan.pdf}

}

\end{figure}%

\subsection{Robustness Checks (Severe
Testing)}\label{robustness-checks-severe-testing}

To ensure that any perceived correlations were not merely artifacts of
finite sampling or ``p-hacking,'' we performed a \textbf{Monte Carlo
Permutation Test} with 1,000 iterations. In this procedure, the
planetary time series was randomly shuffled (breaking its temporal
structure) while the Gold return series remained intact. We then
calculated the \(R^2\) statistic for each permuted dataset to build a
null distribution of ``random chance'' correlations.

\subsubsection{Bootstrapped
Distribution}\label{bootstrapped-distribution}

Figure Figure~\ref{fig-permutation} illustrates the distribution of
correlations under the null hypothesis.

\begin{figure}[H]

\caption{\label{fig-permutation}Monte Carlo Permutation Test
Distribution (N=1,000). The grey histogram shows correlations expected
by random chance. The red vertical line indicates the actual observed
correlation. The fact that the actual result falls well within the
random distribution (\(p > 0.05\)) forces us to fail to reject the null
hypothesis.}

\centering{

\includegraphics[width=0.8\textwidth,height=\textheight]{artifacts/fig_permutation_dist.pdf}

}

\end{figure}%

\section{Discussion}\label{discussion}

The rigorous application of econometric testing fails to reject the null
hypothesis. The results indicates that planetary positions contain no
information gain for the prediction of XAU/USD that is not already
captured by autoregressive lags.

The Molchan diagram's adherence to the diagonal line
(Figure~\ref{fig-molchan}) serves as visual confirmation that
``signals'' often cited in anecdotal astrology are indistinguishable
from random noise when subjected to the ``Severe Testing'' of a complete
dataset. This underscores the necessity of moving from ``Exploratory
Data Analysis'' (finding patterns) to ``Confirmatory Data Analysis''
(testing patterns) in the evaluation of alternative market predictors.

\section{Conclusion}\label{conclusion}

We have presented a reproducible, automated pipeline for rigorously
evaluating the claims of Financial Astrology. By establishing a standard
directory structure and adhering to international reporting standards
(Nature/IEEE), this framework transforms disparate scripts into a
cohesive scientific instrument. The data suggests that while the cosmos
may be ordered, the financial markets remain efficient enough to
discount predictable orbital mechanics.

\section{References}\label{references}

\begin{enumerate}
\def\labelenumi{\arabic{enumi}.}
\tightlist
\item
  \emph{Varahamihira}, Brihat Samhita.
\item
  \emph{Granger, C. W. J.} (1969). Investigating Causal Relations by
  Econometric Models and Cross-spectral Methods.
\item
  \emph{Lomb, N. R.} (1976). Least-squares frequency analysis of
  unequally spaced data.
\end{enumerate}



\end{document}
